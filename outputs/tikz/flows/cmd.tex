% black magic
\makeatletter
\newcommand{\setdy}[3]{%
  \pgfpointdiff{\pgfpointanchor{#1}{center}}{\pgfpointanchor{#2}{center}}
  \edef#3{0.5*\the\pgf@y}}
\newlength{\cy}
\newcommand{\setcy}[1]{%
  \pgfpointanchor{#1}{center}
  \setlength{\cy}{\pgf@y}}
\makeatother
\def\pgfmathsetglobalmacro#1#2{%
  \pgfmathparse{#2}%
  \global\let#1\pgfmathresult}
% pie charts
\newlength{\rad}\setlength{\rad}{6mm}
\newlength{\dx}\setlength{\dx}{0.5mm}
\newcommand{\pie}[5]{
  \node[pie=2*#2*\rad-\dx] (#1) at (#3,#4){};
  \node[pie=2*#2*\rad+\dx] at (#3,#4){};
  \pgfmathsetmacro{\cuma}{0.0}
  \pgfmathsetmacro{\cumb}{0.0}
  \foreach \prop/\clr in {#5}{
    \pgfmathsetglobalmacro{\cumb}{\cumb + \prop}
    \fill[color=\clr] (#3,#4)
    -- ([shift={(#3,#4)}] 90-\cuma*360:#2*\rad) arc (90-\cuma*360:90-\cumb*360:#2*\rad)
    -- cycle;
    \pgfmathsetglobalmacro{\cuma}{\cuma + \prop}
  }
}
% flows
\newcommand{\flow}[6]{
  \coordinate(c) at ($(#1)!0.5!(#2)$);
  \setdy{#1}{#2}{\dy}
  \setcy{c}
  \draw[#4,wide=#5*\turnover pt,domain=-#3:+#3] plot(
    {-\dy/tan(#3) + ((\dy/sin(#3) - (#6+0.5*#5-0.5)*\turnover pt))*cos(\x)},
    { \cy         + ((\dy/sin(#3) - (#6+0.5*#5-0.5)*\turnover pt))*sin(\x)}
  );
  \setlength{\cy}{0.1\rad}
  \pgfmathsetmacro{\sign}{\dy/abs(\dy)}
  \pgfmathsetmacro{\r}{-\dy/tan(#3) + \dy/sin(#3)}
  \draw[fill=black] ($(c) + (\r pt + 0.55*\turnover pt, 0)$)
    -- ++(-\cy,0) -- ++(\cy,\sign*2\cy) -- ++(\cy,-\sign*2\cy) -- cycle;
}
\newcommand{\flows}[6]{
  \pgfmathsetmacro{\cum}{0}
  \foreach \clr/\val in {S3/#6,T3/#5,I3/#4}{%
    \flow{#1}{#2}{#3}{\clr}{\val}{\cum}
    \pgfmathsetglobalmacro{\cum}{\cum + \val}
  }
  \flow{#1}{#2}{#3}{black}{0.1}{-0.1}
%  \flow{#1}{#2}{#3}{black}{0.1}{+1.0}
}