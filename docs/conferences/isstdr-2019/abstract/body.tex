\section{Background}
Heterogeneity in the risks of STI/HIV acquisition and transmission are central to core group theory. We examined
the influence of population turnover among risk groups on group-specific STI prevalence and the contribution of
unmet needs among the core group to onward transmission.
\section{Methods}
We developed an analytical approach to modeling risk group turnover that leverages demographic data and
ensures constant relative risk group size. A deterministic model of STI transmission without disease-attributable
mortality incorporated this turnover approach with three risk groups, including: a core group with the highest rates
of partner change, a multiple-partnerships group, and a low-risk group. We varied the duration within the core
group (3 to 33 years) via turnover among all groups and duration of infectiousness (5 years to lifetime) via a
uniform treatment rate. We then compared the influence of turnover on group-specific STI prevalence at different
treatment rates. We also calibrated to group-specific STI prevalence with and without turnover, and compared the
fitted partner change rates and transmission population attributable fraction (tPAF) of the core group to cumulative
STI infections in the total population.
\section{Methods}
Across the range of turnover and treatment parameters explored, turnover consistently decreased STI prevalence
in the core group. In the low-risk group, turnover increased prevalence under low treatment rate, but had the
opposite effect under high treatment rate. When calibrating to the same STI prevalence, fitted core group partner
change rates were higher with turnover than without. Using these fitted parameters, models with turnover then
consistently projected a higher tPAF of the core group versus models without.
\section{Conclusion}
Modeling of risk group turnover can influence the projected group-specific STI prevalence and fitted risk
parameters. Models without turnover may underestimate the contribution of core groups in STI epidemics, and
thus the impact of interventions prioritizing these populations.
