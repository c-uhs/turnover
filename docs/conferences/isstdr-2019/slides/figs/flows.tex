% colors
\newcommand{\colorf}{\colorx}
% document
\newcommand{\drawflownodes}{
  \clip (-2.5,-1.6) rectangle (+2.5,0.5);
  \node(h) [tbox=\colorx] at (0.0,0.0)        {\scriptsize{High}};
  \node(l) [tbox=\colorx, below = 0.5cm of h] {\scriptsize{Low}};
  \coordinate (m) at ($(h)!0.5!(l)$);
  \coordinate[right = 1cm of m](mh);
  \coordinate[left  = 1cm of m](ml);
}
\newcommand{\drawflowflow}[3]{
  \draw[arrow=\colorf,line width=#3,fill=none,-{Latex[length=4mm]}] (#1) -| (m#1) |- (#2);
}
\newcommand{\drawflowhigh}[1]{\drawflowflow{h}{l}{#1}}
\newcommand{\drawflowlow} [1]{\drawflowflow{l}{h}{#1}}
\newcommand{\drawflowpiehigh}[1]{
  \node(ph) [pie, right of = mh] {\includegraphics[width=1.25cm]{flow-#1-h}};
  \node(ch) [callout] at (ph) {};
  \draw[callout,fill=\colorf] (mh) -- (ch.165) -- (ch.195) -- cycle;
}
\newcommand{\drawflowpielow}[1]{
  \node(pl) [pie, left of = ml] {\includegraphics[width=1.25cm]{flow-#1-l}};
  \node(cl) [callout] at (pl) {};
  \draw[callout,fill=\colorf] (ml) -- (cl.15 ) -- (cl.345) -- cycle;
}
