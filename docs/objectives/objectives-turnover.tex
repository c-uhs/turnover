\documentclass{article}
\usepackage[margin=1.5cm,top=1cm]{geometry}
\setlength{\parskip}{6pt}
\setlength{\parindent}{0pt}
%%%%%%%%%%%%%%%%%%%%%%%%%%%%%%%%%%%%%%%%%%%%%%%%%%%%%%%%%%%%%%%%%%%%%%%%%%%%%%%%%%%%%%%%%%%%%%%%%%%%
Modelling Epidemic Risk Group Dynamics
\usepackage{authblk}
\author[1]{Jesse~Knight}
\author[2]{Stefan~Baral}
\author[2]{Sheree~Schwartz}
\author[1]{Linwei~Wang}
\author[1]{Huiting~Ma}
\author[3]{Katherine~Young}
\author[3]{Harry~Hausler}
\author[1,4,5,6]{Sharmistha~Mishra}
\affil[1]{MAP Centre for Urban Health Solutions, Unity Health Toronto}
\affil[2]{TB HIV Care, South Africa}
\affil[3]{Deptartment of Epidemiology, Johns Hopkins Bloomberg School of Public Health}
\affil[4]{Department of Medicine, Division of Infectious Disease, University of Toronto}
\affil[5]{Institute of Health Policy, Management and Evaluation, Dalla Lana School of Public Health, University of Toronto}
\affil[6]{Instituof Medical Sciences, University of Toronto}
\renewcommand\Affilfont{\itshape\small}
\date{2019 April 05}
%%%%%%%%%%%%%%%%%%%%%%%%%%%%%%%%%%%%%%%%%%%%%%%%%%%%%%%%%%%%%%%%%%%%%%%%%%%%%%%%%%%%%%%%%%%%%%%%%%%%
\begin{document}
  \maketitle
  \subsection*{Research Questions}
  \begin{enumerate}
    \item How can rates of entry into and turnover between risk groups
    be chosen to ensure steady-state risk group sizes?
    \item How can these rates be informed by commonly available data sources?
    \item How are the dynamics of SIR epidemics influenced by the magnitude of turnover
    (from the highest risk group)?
  \end{enumerate}
  \subsection*{Objectives}
  \begin{enumerate}
    \item Formalize a mathematical framework for risk group demographics
    \item Describe methods for deriving risk group demographic parameters from common data sources
    \item Illustrate differences in modelled projections
    for different implementations of risk group demographics,
    using an example SIR system
    \begin{enumerate}
      \item Impact of structure on prevalence:
      \begin{enumerate}
        \item Number of risk groups
        \item Inclusion of population growth
        \item Inclusion of turnover
      \end{enumerate}
      \item Impact of rates of turnover \& treatment on:
      \begin{enumerate}
        \item Overall incidence \& prevalence
        \item Sub-group incidence \& prevalence
      \end{enumerate}
      \item Impact of turnover on fitted model outputs:
      \begin{enumerate}
        \item TPAF of high risk group
      \end{enumerate}
    \end{enumerate}
  \end{enumerate}
\end{document}
