\documentclass{article}
\usepackage[margin=1.5cm,top=1cm]{geometry}
\setlength{\parskip}{6pt}
\setlength{\parindent}{0pt}
%%%%%%%%%%%%%%%%%%%%%%%%%%%%%%%%%%%%%%%%%%%%%%%%%%%%%%%%%%%%%%%%%%%%%%%%%%%%%%%%%%%%%%%%%%%%%%%%%%%%
Contribution of unmet needs of individuals at high risk
may be underestimated in epidemic models without risk turnover:\linebreak[1]
a mechanistic modelling analysis
\usepackage{authblk}
\author[1]{Jesse~Knight}
\author[1]{Linwei~Wang}
\author[1]{Huiting~Ma}
\author[2]{Katherine~Young}
\author[2]{Harry~Hausler}
\author[3]{Sheree~Schwartz}
\author[3]{Stefan~Baral}
\author[1]{Sharmistha~Mishra}
\affil[1]{MAP Centre for Urban Health Solutions, Unity Health Toronto}
\affil[2]{TB HIV Care, South Africa}
\affil[3]{Dept.~Epidemiology, Johns Hopkins Bloomberg School of Public Health}
\renewcommand\Affilfont{\itshape\small}
\date{2019 April 05}
%%%%%%%%%%%%%%%%%%%%%%%%%%%%%%%%%%%%%%%%%%%%%%%%%%%%%%%%%%%%%%%%%%%%%%%%%%%%%%%%%%%%%%%%%%%%%%%%%%%%
\begin{document}
  \maketitle
  \subsection*{Research Questions}
  \begin{enumerate}
    \item How can rates of entry into and turnover between risk groups
    be chosen to ensure steady-state risk group sizes?
    \item How can these rates be informed by commonly available data sources?
    \item How are the dynamics of STI/HIV epidemics influenced by the magnitude of turnover
    (from the highest risk group)?
  \end{enumerate}
  \subsection*{Objectives}
  \begin{enumerate}
    \item Formalize a mathematical framework for risk group demographics
    \item Describe methods for deriving risk group demographic parameters from common data sources
    \item Illustrate differences in modelled projections
    for different implementations of risk group demographics,
    using an example sexually transmitted infection
    \begin{enumerate}
      \item Impact of structure on prevalence:
      \begin{enumerate}
        \item Number of risk groups
        \item Inclusion of population growth
        \item Inclusion of turnover
      \end{enumerate}
      \item Impact of rates of turnover \& treatment on:
      \begin{enumerate}
        \item Overall incidence \& prevalence
        \item Sub-group incidence \& prevalence
        \item TPAF after fitting to group-specific prevalence
      \end{enumerate}
    \end{enumerate}
  \end{enumerate}
\end{document}
