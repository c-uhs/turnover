% !TEX TS-program = pdflatex
% !TEX encoding = UTF-8 Unicode


\documentclass[12pt]{article} 

\usepackage[utf8]{inputenc}

%%% PAGE DIMENSIONS
\usepackage{geometry} % to change the page dimensions
\geometry{a4paper}

\usepackage{graphicx} % support the \includegraphics command and options

%%% PACKAGES
\usepackage{booktabs} % for much better looking tables
\usepackage{array} % for better arrays (eg matrices) in maths
\usepackage{paralist} % very flexible & customisable lists (eg. enumerate/itemize, etc.)
\usepackage{verbatim} % adds environment for commenting out blocks of text & for better verbatim
\usepackage{subfig} % make it possible to include more than one captioned figure/table in a single float
% These packages are all incorporated in the memoir class to one degree or another...

%%% HEADERS & FOOTERS
\usepackage{fancyhdr} % This should be set AFTER setting up the page geometry
\pagestyle{fancy} % options: empty , plain , fancy
\renewcommand{\headrulewidth}{0pt} % customise the layout...
\lhead{}\chead{}\rhead{}
\lfoot{}\cfoot{\thepage}\rfoot{}

%%% SECTION TITLE APPEARANCE
\usepackage{sectsty}
\allsectionsfont{\sffamily\mdseries\upshape} % (See the fntguide.pdf for font help)
% (This matches ConTeXt defaults)

%%% ToC (table of contents) APPEARANCE
\usepackage[nottoc,notlof,notlot]{tocbibind} % Put the bibliography in the ToC
\usepackage[titles,subfigure]{tocloft} % Alter the style of the Table of Contents
\renewcommand{\cftsecfont}{\rmfamily\mdseries\upshape}
\renewcommand{\cftsecpagefont}{\rmfamily\mdseries\upshape} % No bold!

%%% END Article customizations

%%% The "real" document content comes below...

\title{Approach to modeling turnover in sexual risk and implications for epidemics of sexually transmited infections}
\author{Jesse Knight, etc.}
%\date{} % Activate to display a given date or no date (if empty),
         % otherwise the current date is printed 

\begin{document}
\maketitle


\section{Introduction}\label{s:intro}
Core group theory has long underpinned the study of epidemics of 
sexually transmitted infections (STI). The theory posits that heterogeneity 
in acquistion and transmission risk are necessary and sufficient for an STI
 epidemic to emerge and persist. This heterogenetiy is often demarcated by 
identifing potential cores, comprised of subpopulations or geographies, 
where onward transmission risks are the highest such that the core's 
unmet STI prevention and treatment needs sustain local epidemics.
  %~\cite{Yorke1978}. + Gesnick (spatial cores)
\par
Mathematical models of STI transmission include heterogenetiy in risk 
by stratfying the modeled population by features such as the partner 
change rate and different levels of mixing between subgroups by 
partnership type. % can cite key STI or HIV modeling papers
The implications of including heterogenetiy include higher basic 
repoductive ratios, R0, and often lower overall STI prevalence compared 
with assumptions of homogeneity if the latter still results in R0 greater than 1. 
% cite Boily 1997; Anderson & May; Watmough, VanDeDriesse
R0 and overall STI prevalence are further influenced by mixing between subgroups.
 % cite Boily 1997 
Thus, Models with more than two risk groups
are increasingly relevant for exploring epidemic nuance
and for aligning model outputs with programmatic decision support
-- i.e.\ prioritization specific interventions for specific risk groups. % \cite{?}
\par
Less often included in STI transmission models 
and less discussed is the influence of movement of 
individuals between risk groups, which we herein refer to as ``turnover''.%
For example, a period of higher risk could represent the average duration
in formal sex work, which is often associated with larger number of sexual partners
as paid clients and other STI-associated vulnerabilities. % cite Watts; 
It could also represent periods of higher partner change outside
the context of formal sex work. % cite
Stigum et al modeled movement between risk groups as a form
of ``migration'' and showed that [describe] and thus, had nearly as large
an influence on overall STI prevalence as sexual mixing between
subgroups. % Stigum
It has also been shown that rates of movement between risk groups
can play an important role during estimation of intervention impact
following model fitting to calibration targets%~\cite{Eaton2014}. - talk about how played a role /direction
% succintly here or in discussion

% i think useful to put this into the introduction -
% \footnote{In early works, such as %\cite{Stigum1994},
%  movement of individuals between risk groups is often called ``migration'';
% in order to avoid confusion with population entry / exit,
% we prefer the term ``turnover''.}
% This turnover of individuals between risk groups has a similar effect to
% sexual mixing between groups, making it
% an important feature to include in representative models%~\cite{Stigum1994}.
\par
Yet, implementations of risk groups and turnover in recent models vary widely,
from no modeled risk groups%~\cite{Estill2012,Barnighausen2012}
to seven risk groups with highly context-specific turnover. % Boily et al JAIDS 2015]
A common challenge in structuring and parameterizing
STI models are considerations on how best to incorporate 
turnover and duration of periods of risk using available
data. Models require parameters on transition rates between
risk groups, but must also contend with considerations such as
stability in the relative size of risk groups over time.

% above, it might be best to keep the challenges brief and 
% keep content below for discussion.
First, estimating the rates of movement between groups
directly from cross-sectional survey data is difficult,
and typically requires strong assumptions. % but does this paper address these assumptions?
% i..e we should make sure to place intro content specific to solutions presented; and
% then talk about other main challenges we did not address, etc. in discussion
Second, ensuring the relative sizes of risk groups do not vary
dramatically over time requires
careful selection rates of turnover among groups,
or other compensatory parameters. % this is good but see shorter edit above for consideration

%below I think helpful to just list a few approaches previously used in specific terms
Prior works have generally solved this problem \textit{ad hoc},
without providing a generalized approach, % don't think this sentence re: providing a generalized approach is needed 
% b/c prior work were not a priori trying to develop a generalized approach
while some simply rely on a ``burn-in'' period, % suggests we already mentioned what other models did?
which permits equilibriation of risk group sizes due to turnover dynamics
before introduction of the infection.
\par
We therefore draw on prior work to propose a unified framework for
defining and parameterizing risk group dynamics.
We present such a framework here,
and draw direct links to modeling assumptions and relevant sources of data. % not sure I follow this sentence?
Building on previous work by%~\citet{Stigum1994},
we then leverage this framework to explore the influence of % in general, we do not write 'impact' as it has specific
% connotations in epidemic modeling (i.e. intervention impact), so best to say influence...
several risk group implementations on model outputs (incidence, prevalence) % i thought it was influence of inclusion/exclusion
% of turnover, and rates of turnover? would revise the objectives to tell the reader what exactly will be examined...
in a representative model. % model of an STI (for later, re: perhaps saying S-I and S-I-S explicilty)




\subsection{A subsection}

More text.

\end{document}
