% JK: We discussed reframing the introduction
%     around influence of turnover on TPAF of the high risk group.
%     However, I found it difficult to jump into that idea
%     without introducing at least the idea of heterogeneity.
%     Plus, I really like the idea of starting from the "basics"
%     (with core group theory), as it emphasizes we are working with
%     the fundamentals of epidemic dynamics,
%     which we're kind of arguing should include turnover.
%     I did beef up the section on TPAF in para 3 though,
%     and reframed the results in the last para
%     to put understanding TPAF as the motivation for other experiments.
%     Does that work?
Core group theory has long underpinned the study of
epidemics of sexually transmitted infections (STI).
The theory posits that heterogeneity in
acquisition and transmission risk are
sometimes necessary and sometimes sufficient for
an STI epidemic to emerge and persist.
This heterogeneity is often demarcated by identifying potential cores,
comprised of sub-populations or geographies,
where risks of acquisition and onward transmission are the highest,
such that the core's unmet STI prevention and treatment needs
sustain local epidemics \citep{Yorke1978,Gesink2011}.
Such cores often comprise so-called key populations, such as
female sex workers,
men who have sex with men,
and transgender people.
\par
Mathematical models of STI transmission include heterogeneity in risk
by stratifying the modelled population by features such as
the partner change rate,
levels of sexual mixing between subgroups, and
partnership types \citep{Mishra2012}.
The implications of including heterogeneity,
as compared to assumptions of homogeneity, include
higher basic reproductive ratios $R_0$, and
lower overall STI prevalence
(provided the latter still results in $R_0 > 1$) \citep{Boily1997}.
$R_0$ and overall STI prevalence are further influenced by
mixing between subgroups \citep{Stigum1994,Boily1997}.
\par % TPAF >>
Mathematical models with multiple risk groups can then be used to quantify
the contribution of high-risk groups to overall transmission.
For example, the \textit{transmission population attributable fraction} (TPAF)
of the group can be computed \citep{Mishra2012}.
TPAF is defined as: the fraction of all new infections that stem,
directly and indirectly, from a failure to prevent infection
in a particular risk group.
The TPAF can be used to help guide ``prioritized'' or ``targeted'' interventions
for groups at highest risk.
\par
In most transmission models, individuals in a particular risk group
are assumed to remain in that group for their entire life.
Rarely discussed or included in transmission models
is the movement of individuals between risk groups,
which we refer to as ``turnover''.
For example, consider a high risk group representing sex work,
for which there are a larger number of sexual partners as paid clients,
and other STI-associated vulnerabilities \citep{Watts2010}.
Individuals may
retire from sex work but continue to be sexually active at a lower level of risk,
or enter into sex work following a period of lower risk \citep{Boily2015}.
Individuals may also enter into and exit from
a group engaging in multiple partnerships, and so on.
\par
% JK: Original SM edits asked for specifically what mechanisms
%     are described by these references re. influence of turnover on model outputs,
%     but no paper actually offers very much for this, they just say things like:
%     "strongly affected by immigration" (Stigum1994) or
%     "accounted for 92% of the variation" (Eaton2014).
Several authors have implicated risk group turnover as 
an important factor in model outputs, such as:
the predicted equilibrium prevalence \citep{Stigum1994,Eaton2014};
the fraction of transmissions occurring during acute infection \citep{Zhang2012};
the basic reproductive number $R_0$ \citep{Henry2015}; and
the level of universal treatment required to achieve epidemic control \citep{Henry2015}.
Yet, implementations of risk groups and turnover in recent models vary widely.
In works by \citeauthor{Koopman1997},
rates of movement between two risk groups
are balanced analytically based on the size of the groups;
\citet{Boily2015} use a 100-year burn-in period
to equilibrate a complex system of turnover transitions;
some authors only model turnover in the direction of high to low risk,
\citep{Stigum1994,Eaton2014}.
These various approaches to turnover are contrasted with
implementations of model features like assortative sexual mixing,
which typically follow ``standard'' methods,
such as that proposed by \citet{Nold1980}.
\par
Perhaps widespread and consistent implementation of turnover dynamics in STI models
is made difficult by the following two challenges.
First, it is not clear how epidemiological data can be used to
inform rates of turnover among risk groups.
Second, as shown by \citet{Boily2015},
naive selection of turnover rates can result in imbalanced flows,
causing risk groups to initially change size over time.
\par
As a solution to these problems,
we introduce a unified framework for
parameterizing risk group turnover, based on available data.
We develop this framework in Section~\ref{s:system}.
In Sections~\ref{s:exp}~and~\ref{s:results},
we sought to examine the influence of turnover
on the estimated TPAF of a high risk group,
using an illustrative STI model.
However, to understand this influence of turnover on TPAF,
we first investigated the influence of turnover on
group-specific incidence and prevalence
(Experiment~1).
Similarly, since the uncertain model parameters are often fitted
so that the model reproduces observed prevalence data,
we also examined the influence of turnover on fitted parameters
(Experiment~2).
The influence of turnover on TPAF was then explored,
before and after model fitting
(Experiment~3).
We discuss the implications of these experiments
in Section~\ref{s:discussion}.