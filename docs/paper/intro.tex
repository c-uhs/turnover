Heterogeneity in transmission risk is a consistent characteristic of
% HM: I feel we might want to explain "heterogeneity" a little more?
epidemics of sexually transmitted infections (STI) \citep{Anderson1991}.
This heterogeneity is often demarcated by identifying
specific populations whose risks of acquisition and onward transmission of STI are the highest,
such that their specific unmet prevention and treatment needs
can sustain local epidemics of STI \citep{Yorke1978}.
An important indicator in the appraisal of STI epidemics is
the contribution of high risk groups to the overall epidemic \citep{Case2012}.
% SB: Rather than saying important, would explain. Ie, try to avoid editorializing in the
% introduction and just provide empiric evidence as to why this is the case.
\par
Indicators of contribution are used to help prioritize or target interventions
% SB: Why quotations? I see that you are using this as the link to the last sentence of the last
% paragraph but would avoid this. Suspense is not really ideal here... :)
% HM: do not understand this sentience.
to reach groups at highest risk \citep{Case2012,Shubber2014}.
Transmission models are increasingly being used to quantify
contribution by simulating counterfactuals where transmission 
% SB: Also don’t think quotations are ideal. Could just explain contribution as the traditional
% PAF and that also sets up nicely for why the tPAF is an advance.
between specific subgroups is stopped, and
the relative difference in cumulative infections in the total population
over various time-periods is measured \citep{Mishra2016,Mukandavire2018}.
This is often referred to as the
\textit{transmission population attributable fraction} (TPAF)
% SB: Really think the PAF as an old concept needs to be introduced in a sentence and then can
% include the TPAF as to why it is an advance for infectious diseases.
and counterfactuals have included
setting susceptibility to zero and/or setting 
infectiousness to zero to calculate the TPAF \citep{Mishra2012}.
The TPAF is then interpreted as
the fraction of all new infections that stem, directly and indirectly, from
a failure to prevent acquisition and/or
a failure to provide effective treatment in a particular risk group \citep{Mishra2016}.
\par
An epidemiologic phenomenon that is often overlooked in transmission models,
but is well-described in the context of sexual behaviour,
is the movement between risk groups \citep{Watts2010}.
% SB: Again, editorializing and is setting up potentially adversarial relationships with reviewers.
%     I just don’t think you need the sentence and can still highlight the need for turnover in models.
Such movement is often referred to in the STI epidemiology literature as
``turnover'' \citep{Watts2010}.
For example, turnover may reflect entry and retirement from
formal sex work -- a period of time in an individual's sexual life-course
that represents a greater risk of STI susceptibility and 
onward transmission \citep{Watts2010}. Similarly, individuals 
may have a larger number of sexual partnerships or experience
greater STI-associated vulnerabilities during specific times
within their sexual life-course \citep{Marston2006}.
\par
Risk group turnover has been shown to 
influence the predicted equilibrium prevalence of an STI \citep{Stigum1994,Zhang2012};
the fraction of transmissions occurring during acute HIV infection \citep{Zhang2012};
the basic reproductive number $R_0$ of HIV \citep{Henry2015}; and
the coverage of antiretroviral therapy required to achieve HIV epidemic control \citep{Henry2015}.
% SB: Can cite our recent Lancet HIV piece...
Yet how, and the extent to which, turnover influences the TPAF remains unknown.
% SB: I would really frame the specific element of what you aim to study towards
%     the last paragraph of intro.
\par
Implementations of risk group turnover in compartmental transmission models also vary widely.
In works by \citeauthor{Koopman1997} and \citeauthor{Stigum1994},
rates of movement between two risk groups
were balanced analytically based on the size of the groups;
\citet{Boily2015} used a 100-year burn-in period
to equilibrate a complex system of demographic transitions before 
introduction of HIV into the modelled system;
others restricted the system to unidirectional turnover -- e.g. from high to low risk
\citep{Eaton2014}.
% LW: Can u explain a bit more about this approach?
%     Not clear what exactly burn-in period mean here.
%     Given lots of argument below was based on avoiding the need for a burn-in period,
%     it is necessary to let readers know what it is and what is the limitation in this approach.
% SB: Would break this sentence up and likely not have it be its own paragraph
\par
Challenges in implementing turnover include
incorporation of data-driven epidemiologic constraints.
For example, data may suggest
that the relative sizes of specific populations in the model,
such as the population of sex workers,
have remained constant over time. % \cite{TBD} % Type 1
Data may also suggest that heterogeneity in risk behaviour
of individuals entering into the model (reflecting coital debut)
may be different from the risk behaviour
of individuals already in the model (reflecting sexually active adults). % \cite{TBD} % Type 2
Data may indicate the average duration
of a high risk period of one's sexual life-course \citep{Watts2010}, % Type 3
or how their behaviour changes following that period. % \cite{TBD} % Type 4
Such data should be reflected in implementations of turnover,
but it is not always clear how to do so.
Moreover, without an exact analysis of the turnover implementation,
a burn-in period may be required to equilibrate the system,
resulting in potentially unwanted changes to the initial group sizes.
\par
In this paper, we aim to examine the influence of risk group turnover
on the TPAF of an illustrative STI without STI-attributable mortality.
First, we propose a solution to the challenges outlined above
of including epidemiologic data while avoiding the need for a burn-in period.
That is, we introduce a unified framework for
parameterizing risk group turnover, based on available data.
% LW: Find this sentence a bit too abstract.
%     Wondering if can provide a bit more detail here. Like the one sentence summary of other
%     approaches mentioned above:
%     e.g., In works by Koopman et al. and Stigum et al., rates of movement between two risk groups
%     were balanced analytically based on the size of the groups
%     Then in ur approach, what was done? What was the available data?
We then examine the mechanisms by which turnover
influences group-specific STI incidence and prevalence
(Experiment~1).
We then examine how inclusion/exclusion of turnover influences
the values parameters related to heterogeneity during model fitting
(Experiment~2).
% LW: Should experiment be split into 2?
%     To me, the first part is a continuation of Experiment 1
%     and the latter is a continuation of experiment 2.
%     Also, in the write-up, might be clearer for the readers if we rearrange the experiments
%     (1, 3(first half); 2, 3 (second half)
Finally, we compare the TPAF of the highest risk group estimated
from two different settings: one with and one without turnover,
and from two models that reproduce the same setting:
one model with and one model without turnover
(Experiment~3).
