Heterogeneity in transmission risk is a consistent characteristic of
epidemics of sexually transmitted infections (STI) \citep{Anderson1991}.
This heterogeneity is often demarcated by identifying
specific populations whose risks of acquisition and onward transmission of STI are the highest,
such that their specific unmet prevention and treatment needs
can sustain local epidemics of STI \citep{Yorke1978}.
Increased risk can be conferred in many ways, including
a higher number of sexual partners,
reduced condom use,
stigma during access to care.
% HM: I feel we might want to explain "heterogeneity" a little more?
% JK: how about the new content above?
The contribution of high risk groups to the overall epidemic
can then be used as an indicator in the appraisal of STI epidemics,
helping to guide intervention priorities
\citep{Shubber2014,Mishra2016}.
% SB: Rather than saying important, would explain. Ie, try to avoid editorializing in the
% introduction and just provide empiric evidence as to why this is the case.
\par
Traditionally, contribution has been quantified using the
\textit{population attributable fraction} (PAF):
the proportion of new infections which are
attributable to disproportionate risk of transmission in a given population,
such as in the modes of transmission model \citep{Case2012,Mishra2014a}.
% SB: Also don’t think quotations are ideal. Could just explain contribution as the traditional
%     PAF and that also sets up nicely for why the tPAF is an advance.
% SB: Really think the PAF as an old concept needs to be introduced in a sentence and then can
%     include the TPAF as to why it is an advance for infectious diseases.
% JK: Is this better now?
However, transmission models are increasingly being used to quantify
contribution over longer time horizons
using the \textit{transmission population attributable fraction} (TPAF).
The TPAF is estimated by
simulating counterfactual scenarios where transmission
between specific subgroups is stopped, and
the relative difference in cumulative infections in the total population
over various time-periods is measured \citep{Mishra2016,Mukandavire2018}.
Transmission can be stopped by
setting susceptibility and/or infectiousness to zero in the model \citep{Mishra2012}.
The TPAF is then interpreted as
the fraction of all new infections that stem, directly and indirectly, from
a failure to prevent acquisition and/or to provide effective treatment
in a particular risk group \citep{Mishra2016}.
\par
An epidemiologic phenomenon that is sometimes missing from transmission models,
but is well-described in the context of sexual behaviour,
is the movement of individuals between risk groups.
% SB: Again, editorializing and is setting up potentially adversarial relationships with reviewers.
%     I just don’t think you need the sentence and can still highlight the need for turnover in models.
% JK: I agree it is hard to express this thought in a friendly / objective way,
%     but I really think its worth noting.
Such movement is often referred to in the STI epidemiology literature as
\textit{turnover} \citep{Watts2010}.
For example, turnover may reflect entry into or retirement from formal sex work,
or other periods associated with higher STI susceptibility and onward transmission
due to more partners and/or vulnerabilities
\citep{Marston2006,Watts2010}.
Risk group turnover has been shown to
influence the predicted equilibrium prevalence of an STI \citep{Stigum1994,Zhang2012};
the fraction of transmissions occurring during acute HIV infection \citep{Zhang2012};
the basic reproductive number $R_0$ \citep{Henry2015}; and
the coverage of antiretroviral therapy required to achieve HIV epidemic control \citep{Henry2015}.
% SB: Can cite our recent Lancet HIV piece...
% JK: This one? "The disconnect between individual-level ..."
%     I coubldn't see if it commented on turnover though
Yet how, and the extent to which, turnover influences TPAF remains unknown.
% SB: I would really frame the specific element of what you aim to study towards
%     the last paragraph of intro.
% JK: Just trying to tie it in to the previous two paragpraphs here
Implementations of risk group turnover in compartmental transmission models also vary widely.
In some simple models, the rates of movement between two risk groups
is balanced analytically based on the sizes of the groups
\citep{Koopman1997,Stigum1994}.
In more complex models, a stabilization period is used
to equilibrate turnover dynamics,
during which time the sizes of risk groups may deviate
from their initial values \citep{Boily2015}.
Other models only consider unidirectional turnover
-- e.g. from high to low risk \citep{Eaton2014}.
% LW: Can u explain a bit more about this approach?
%     Not clear what exactly burn-in period mean here.
%     Given lots of argument below was based on avoiding the need for a burn-in period,
%     it is necessary to let readers know what it is and what is the limitation in this approach.
% SB: Would break this sentence up and likely not have it be its own paragraph
% JK: Hopefully this is clearer now?
\par
Challenges in implementing turnover include
incorporation of data-driven epidemiologic constraints.
For example, data may suggest
that the relative sizes of specific populations in the model,
such as the population of sex workers,
have remained constant over time. % \cite{TBD} % Type 1
Data may also suggest that heterogeneity in risk behaviour
of individuals entering into the model (reflecting coital debut)
may be different from the risk behaviour
of individuals already in the model (reflecting sexually active adults). % \cite{TBD} % Type 2
Data may indicate the average duration
of a high risk period of one's sexual life-course \citep{Watts2010}, % Type 3
or how their behaviour changes following that period. % \cite{TBD} % Type 4
Such data should be reflected in implementations of turnover,
but it is not always clear how to do so.
Moreover, without an exact analysis of the turnover implementation,
a burn-in period may be required to equilibrate the system,
resulting in potentially unwanted changes to the initial group sizes.
% JK: I'm struggling to find citations for these ^ @SM do you have any suggestions?
\par
In this paper, we aim to examine the influence of risk group turnover
on the TPAF of a high risk group
in an illustrative STI without STI-attributable mortality.
First, we propose a solution to the challenges outlined above
of parameterizing turnover based on epidemiologic data
while avoiding the need for an equilibration period.
% LW: Find this sentence a bit too abstract.
%     Wondering if can provide a bit more detail here. Like the one sentence summary of other
%     approaches mentioned above:
%     e.g., In works by Koopman et al. and Stigum et al., rates of movement between two risk groups
%     were balanced analytically based on the size of the groups
%     Then in ur approach, what was done? What was the available data?
% JK: Since this is no longer a major focus of the paper,
%     I think it works to just leave it as one sentence. Thoughts?
We then examine the mechanisms by which turnover
influences group-specific STI prevalence
(Experiment~1).
Next, we examine how inclusion/exclusion of turnover influences
the values of heterogeneity-related parameters inferred during model fitting
(Experiment~2).
Finally, we compare the TPAF of the highest risk group
estimated by two models after fitting to same setting:
one model with and one model without turnover
(Experiment~3).