% JK: We discussed reframing the introduction
%     around influence of turnover on TPAF of the high risk group.
%     However, I found it difficult to jump into that idea
%     without introducing at least the idea of heterogeneity.
%     Plus, I really like the idea of starting from the "basics"
%     (with core group theory), as it emphasizes we are working with
%     the fundamentals of epidemic dynamics,
%     which we're kind of arguing should include turnover.
%     I did beef up the section on TPAF in para 3 though,
%     and reframed the results in the last para
%     to put understanding TPAF as the motivation for other experiments.
%     Does that work?  
% SM: Yes, but it seems less succint and especially for a journal like Epidemics
%     - seems like it takes a while to get to the point of the paper.
%     I have tried to shorten it and also removed mention of core-group theory
%     since we do not really tackle that in explicit terms in the experiment or discussion.
%     Just seemed like a lot of terms/concepts that we were not necessarily revisiting.
%     i.e. all focus is on the unified approach + TPAF.
%     So I am not sure we are adding much by talking about core-group theory
%     when we do not really challenge or add to core-group theory
%     in our key findings/key messages (even though that is one way we could have taken the paper).
%     I think its best to not introduce more terms
%     - but keep the introduction as short and sweet as possible!
%     An intro should mirror the discussion.
% JK: Yes, makes sense! Thanks for all your work with this,
%     and will do my best to integrate the "move to discussion" content
%     with those revisions.
Heterogeneity in transmission risk are critical features of 
epidemics of sexually transmitted infections (STI)~\cite{Anderson1991}.
This heterogeneity is often demarcated by identifying
sub-populations whose risks of acquisition and onward transmission are the highest,
such that their unmet STI prevention and treatment needs
can sustain local epidemics \citep{Yorke1978}.
An important indicator in the appraisal of STI epidemics is
the contribution of high-risk groups to the overall epidemic \citep{Case2012}.
\par
Indicators of ``contribution'' are used to help prioritize or target interventions
to reach groups at highest risk \citep{Case2012,Shubber2014}.
Transmission models are increasingly being used to quantify
``contribution'' by simulating counterfactuals where transmission 
from and/or to specific subgroups is stopped, and 
the relative difference in cumulative infections in the total population
over various time-periods is measured \citep{Mishra2016,Mukandavire2018}.
This is often referred to as the
\textit{transmission population attributable fraction} (TPAF) 
and counterfactuals have included
setting susceptibility to zero and/or setting 
infectiousness to zero to calculate the TPAF \citep{Mishra2012}.
The TPAF is then interpreted as
the fraction of all new infections that stem, directly and indirectly, from
a failure to prevent acquisition and/or
a failure to provide effective treatment in a particular risk group \citep{Mishra2016}.
\par
An epidemiological phenomenon that is often overlooked in transmission models,
but is well-described in the context of sexual behaviour,
is the movement between risk groups \citep{Watts2010}.
Such movement is often referred in the STI epidemiology literature as
``turnover'' \citep{Watts2010}.
For example, turnover may reflect entry and retirement from
formal sex work -- a period of time in an individual's sexual life-course
that represents a greater risk of STI susceptibility and 
onward transmission \citep{Watts2010}. Similarly, individuals 
may have a larger number of sexual partnerships or experience
greater STI-associated vulnerabilities during specific times
within their sexual life-course \citep{Marston2006}.
% JK: Original SM edits asked for specifically what mechanisms
%     are described by these references re. influence of turnover on model outputs,
%     but no paper actually offers very much for this, they just say things like:			
%     "strongly affected by immigration" (Stigum1994) or
%     "accounted for 92% of the variation" (Eaton2014).
% SM: interesting, and got it. then your paper provides 'mechanistic' reasons
%     and so this addition should be highlighted in the discussion section.
% JK: sounds good!
\par
Risk group turnover has been shown to 
influence the predicted equilibrium prevalence of an STI \citep{Stigum1994,Eaton2014};
the fraction of transmissions occurring during acute HIV infection \citep{Zhang2012};
the basic reproductive number $R_0$ of HIV \citep{Henry2015}; and
the level of universal treatment required to achieve HIV epidemic control \citep{Henry2015}.
Yet how, and the extent to which, turnover influences the TPAF remains unknown.
\par
Implementations of risk group turnover in compartmental transmission models also vary widely.
In works by \citeauthor{Koopman1997} and \citeauthor{Stigum1994},
rates of movement between two risk groups
were balanced analytically based on the size of the groups;
\citet{Boily2015} used a 100-year burn-in period
to equilibrate a complex system of demographic transitions before 
introduction of HIV into the modelled system;
others restricted the system to unidirectional turnover -- e.g. from high to low risk
\citep{Eaton2014}.
\par
Challenges in implementing turnover include
incorporation of data-driven epidemiological constraints.
For example, data may suggest
that the relative sizes of subgroups in the model
(such as the population of sex workers)
have remained constant over time. % \cite{TBD} % Type 1
Data may also suggest that heterogeneity in risk behaviour
of individuals entering into the model (reflecting sexual debut)
may be different from the risk behaviour
of individuals already in the model (reflecting sexually active adults). % \cite{TBD} % Type 2
Data may indicate the average duration
of a high-risk period of one's sexual life-course \citep{Watts2010}, % Type 3
or how their behaviour changes following that period. % \cite{TBD} % Type 4
% SM: list the epidemiological constraints discussed in the system
%     [in same order as in system] to foreshadow the unified approach.
%     cite each if possible or leave as [REF] for now,
%     and I will help you find the right refs when paper is with co-authors.
% JK: It's difficult to introduce Constraints 2 and 4 without
%     having introduced the system (as in Section 2),
%     but above I gave it a go.
Such data should be reflected in implementations of turnover,
but it is not always clear how to do so.
Moreover, without an exact analysis of the turnover implementation,
a burn-in period may be required to equilibrate the system,
resulting in potentially unwanted changes to the initial group sizes.
% Second, as shown by \citet{Boily2015}, naive selection of turnover rates can result in imbalanced flows,		
% causing risk groups to initially change size over time.
% SM: i am unclear why this is referred to as 'niavie',
%     and we do want risk groups to change over time?
%     i.e. this rationale as a 'challenge' is unclear/vauge.
%     Also - try to be careful in your tone of language as it reads as editorializing
%     (e.g. niavie --> what does that mean? what is not niavie then?).
%     Try to be really precise in your writing.
%     And clear/transparent - e.g. I also think you have to be really careful as
%     all the papers cited above (except Henry, Stigum) include STI-attributable mortality.
%     Flows may have to imbalanced because we do not want to replace 1:1
%     - that creates a massive artifact re: resupply of susceptibles
%     (as shown by some work Anna and Nasheed did already).
% JK: Sorry! Did not mean to imply any degredation by the word "naive" -- more like "Naive Bayes"
%     (in fact the Boily model is most complex.)
%     But you're right, bad choice of word...
%     I meant to suggest that the challenge of balancing risk groups
%     (even before HIV-attributable mortality) was not analytically solved,
%     but instead the burn-in period was used.
%     Hopefully, I've made a bit more clear why
%     the burn-in period might cause problems though:
%     "resulting in potentially unwanted changes to the initial group sizes".
\par
In this paper, we aim to examine the influence of risk group turnover
on the TPAF of an illustrative STI without STI-attributable mortality.
First, we propose a solution to the challenges outlined above
of including epidemiological data while avoiding the need for a burn-in period.
That is, we introduce a unified framework for
parameterizing risk group turnover, based on available data.
We then examine the mechanisms by which turnover
influences group-specific STI incidence and prevalence
(Experiment~1).
We then examine how inclusion/exclusion of turnover influences
the values parameters related to heterogeneity during model fitting
(Experiment~2).
Finally, we compare the TPAF of the highest-risk groups estimated
from two different settings: one with and one without turnover,
and from two models that reproduce the same setting:
one model with and one model without turnover
(Experiment~3).