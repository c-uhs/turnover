Core group theory has long underpinned the study of
epidemics of sexually transmitted infections (STI).
The theory posits that heterogeneity in
acquisition and transmission risk are sometimes necessary and sometimes sufficient for
an STI epidemic to emerge and persist.
This heterogeneity is often demarcated by identifying potential cores,
comprised of sub-populations or geographies,
where risks of acquisition and onward transmission are the highest,
such that the core's unmet STI prevention and treatment needs
sustain local epidemics
\citep{Yorke1978,Gesink2011}.
\par
Mathematical models of STI transmission include heterogeneity in risk
by stratifying the modelled population by features such as
the partner change rate,
levels of sexual mixing between subgroups, and
partnership types
\citep{Mishra2012}.
The implications of including heterogeneity,
as compared to assumptions of homogeneity, include
higher basic reproductive ratios $R_0$, and
lower overall STI prevalence
(provided the latter still results in $R_0 > 1$)
\citep{Boily1997}.
$R_0$ and overall STI prevalence are further influenced by
mixing between subgroups \citep{Stigum1994,Boily1997}.
Models with two or more risk groups
are increasingly relevant in applied epidemic modelling,
such as to help prioritize specific interventions for particular risk groups
\citep{Mishra2012}.
\par
Rarely discussed or included in STI transmission models
is the movement of individuals between risk groups,
which we refer to as ``turnover''.
For example, consider a high risk group representing sex work,
for which there are a larger number of sexual partners as paid clients,
and other STI-associated vulnerabilities
\citep{Watts2010}.
Individuals may
retire from sex work but continue to be sexually active at a lower level of risk,
or enter into sex work following a period of lower risk
\citep{Boily2015}.
Similar transitions are also possible for
individuals engaging in multiple partnerships, and so on.
\par
%% JK: I know original SM edits asked for specifically what mechanisms
%% are described by these references re. influence of turnover on model outputs,
%% but neither paper actually offers very much for this, they just say things like:
%% "strongly affected by immigration" (Stigum1994) or
%% "accounted for 92% of the variation" (Eaton2014)
%% I'm thinking this can be described late in the Systems > Prior Work section?
Several authors have implicated turnover as 
an important factor in model outputs, such as:
the predicted equilibrium prevalence \citep{Stigum1994,Eaton2014};
the fraction of transmissions occurring during acute infection \citep{Zhang2012};
the basic reproductive number $R_0$ \citep{Henry2015}; and
the level of universal treatment required to achieve epidemic control \citep{Henry2015}.
Yet, implementations of risk groups and turnover in recent models vary widely.
% from no risk groups,
% to several risk groups but no turnover,
% to several risk groups with complicated systems of turnover \citep{Boily2015}.
Some authors only model turnover in the direction of high to low risk,
controlled by a single parameter \citep{Stigum1994,Eaton2014}.
In works by \citeauthor{Koopman1997},
rates of movement between two risk groups
are balanced analytically based on the size of the groups,
while \citet{Boily2015} use a 100-year burn-in period
to equilibrate a complex system of turnover transitions.
These various approaches to turnover are contrasted with
implementations of model features like assortative sexual mixing,
which typically follow ``standard'' methods,
such as that proposed by \citet{Nold1980}.
\par
Perhaps widespread and consistent implementation of turnover dynamics in STI models
is made difficult by the following two challenges.
First, it is not obvious how epidemiological data can be used to
inform rates of turnover among risk groups.
Second, as shown by \citet{Boily2015},
naive selection of turnover rates can result in imbalanced flows,
causing risk groups to initially change size over time;
it is not clear how to avoid this problem in all cases.
\par
Therefore, we propose a unified framework for
parameterizing risk group dynamics,
including risk heterogeneity, population growth, and turnover.
We develop this framework in Section \ref{s:system}.
Additionally, while some authors have shown the influence
of turnover on overall incidence and prevalence
\citep{Stigum1994,Zhang2012,Henry2015},
the mechanisms and influence of turnover on
group-specific incidence and prevalence remain unclear.
In Sections \ref{s:exp} and \ref{s:results},
we illustrate these mechanisms using a representative SIR model.
Finally, epidemic models are often used to help prioritize interventions
following calibration to a specific context.
Since turnover can affect model calibration,
we also explore some potential implications of failing to model turnover dynamics
when they truly exist in Sections \ref{s:exp} and \ref{s:results}.
We discuss the results of these experiments in Section \ref{s:discussion}.