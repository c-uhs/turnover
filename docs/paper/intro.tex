% JK: We discussed reframing the introduction
%     around influence of turnover on TPAF of the high risk group.
%     However, I found it difficult to jump into that idea
%     without introducing at least the idea of heterogeneity.
%     Plus, I really like the idea of starting from the "basics"
%     (with core group theory), as it emphasizes we are working with
%     the fundamentals of epidemic dynamics,
%     which we're kind of arguing should include turnover.
%     I did beef up the section on TPAF in para 3 though,
%     and reframed the results in the last para
%     to put understanding TPAF as the motivation for other experiments.
%     Does that work?  

% SM: Yes, but it seems less succint and especially for a journal like Epidemics - seems like it takes a while to get to the point of the paper. I have tried to shorten it and also removed mention of core-group theory since we do not really tackle that in explicit terms in the experiment or discussion. Just seemed like a lot of terms/concepts that we were not necessarily revisiting. i.e. all focus is on the unified approach + TPAF. So I am not sure we are adding much by talking about core-group theory when we do not really challenge or add to core-group theory in our key findings/key messages (even though that is one way we could have taken the paper). I think its best to not introduce more terms - but keep the introduction as short and sweet as possible!
% An intro should mirror the discussion.

Heterogeneity in
acquisition and transmission risk are critical features of 
epidemics of sexually transmitted infections (STI). %cite e.g. Garnett ; Anderson & May textbook
This heterogeneity is often demarcated by identifying
sub-populations where risks of acquisition and onward transmission are the highest,
such that their unmet STI prevention and treatment needs
can sustain local epidemics \citep{Yorke1978,Gesink2011}.		%since not talking about core, remove Gesink citation [only discussed epi-core not transmission core]
An important indicator in the appraisal of STI epidemics is
the contribution of high-risk groups to the overall epidemic.  %cite: Case et al. PMID: 23226895;

\par  %Move (and can shorten) this paragraph to Discussion. 
%Mathematical models of STI transmission include heterogeneity in risk
%by stratifying the modelled population by features such as %check spelling of modelled - one l or two l re: consistency check
%the partner change rate,
%levels of sexual mixing between subgroups, and
%partnership types \citep{Mishra2012}.				%I don't think this is the appropriate citation. such strata came way before our 2012 paper. go back Garnet or Anderson & May.
%The implications of including heterogeneity,
%as compared to assumptions of homogeneity, include
%higher basic reproductive ratios $R_0$, and
%lower overall STI prevalence
%(provided the latter still results in $R_0 > 1$) \citep{Boily1997}.	
%$R_0$ and overall STI prevalence are further influenced by
%patterns of mixing (or who has sex with whom) between subgroups \citep{Stigum1994,Boily1997}.

\par % TPAF >>
Indicators of ``contribution'' are used to help ``prioritize'' or ``target'' interventions
for groups at highest risk. %cite: Case et al. PMID: 23226895; Shubber et al. PMID: 24962034
Transmission models are increasingly being used to quantify
``contribution'' by simulating counterfactuals where transmission 
from and/or to specific subgroups is blocked, and 
the relative difference in cumulative infections in the total population over 
various time-periods is measured. %cite: 'data needs annals epi paper' PMID: 27421700 ; Vickerman Senegal paper PMID: 30033604 
This is often referred to as the \textit{transmission population attributable fraction} (TPAF) 
and counterfactuals have included
setting susceptibility to zero and/or setting 
infectiousness to zero to calculate the TPAF \citep{Mishra2012}.
The TPAF is then interpreted as
the fraction of all new infections that stem,
directly and indirectly, from a failure to prevent acquisition and/or 
a failure to provide effective treatment in a particular risk group.		%cite: 'data needs annals epi paper' PMID: 27421700 					
\par
An epidemiological phenomena that is often overlooked in transmission models, but
is well-described in the context of sexual behaviour 
is the movement between risk groups.						%Watts PMID: 21098061 
Such movement 
is often referred in the STI epidemiology literature as ``turnover''.  %cite (but check if its the best/appropriate citation as cannot remember if they used the term 'turnover'): Watts PMID: 21098061 
For example, turnover may reflect entry and retirement from	% in general, would try to avoid writing papers like we are a teaching a course. appropriate for seminar' paper but less so for an original research paper.
formal sex work - a period of time in an individual's sexual-life course
that represents a greater risk of STI susceptibility and 
onward transmission \citep{Watts2010}. Similarly, individuals 
may have a larger number of sexual partnerships or experience
greater STI-associated vulnerabilities during specific times
within their sexual life-course. % cite epi paper like this
% JK: Original SM edits asked for specifically what mechanisms
%     are described by these references re. influence of turnover on model outputs,
%     but no paper actually offers very much for this, they just say things like:			
%     "strongly affected by immigration" (Stigum1994) or
%     "accounted for 92% of the variation" (Eaton2014).
%SM interesting, and got it. then your paper provides 'mechanistic' reasons and so this addition should be highlighted in the discussion section.
Risk group turnover has been shown to 
influence the predicted equilibrium prevalence of an STI \citep{Stigum1994,Eaton2014};
the fraction of transmissions occurring during acute infection \citep{Zhang2012};		% of an STI?
the basic reproductive number $R_0$ \citep{Henry2015}; and						% in general, or of an STI?
the level of universal treatment required to achieve epidemic control \citep{Henry2015}.	% in general, or of an STI?
Yet how, and the extent to which, turnover influence the TPAF remains unknown.

Implementations of risk group turnover in compartmental transmission models vary widely.  %individual-based models actually do capture this, but in different ways so lets be specific to the type of models we are limited to talking about / have examined
In works by \citeauthor{Koopman1997},
rates of movement between two risk groups
were balanced analytically based on the size of the groups;
\citet{Boily2015} used a 100-year burn-in period
to equilibrate a complex system of demographic transitions before 
introduction of HIV into the modelled system;
others restricted the system to unidirectional turnover - e.g. from high to low risk,
\citep{Stigum1994,Eaton2014}.
%These various approaches to turnover are contrasted with			%SM not needed here - does not add to rationale for the paper. Can put in discussion section.
%implementations of model features like assortative sexual mixing,
%which typically follow ``standard'' methods,
%such as that proposed by \citet{Nold1980}.
Challenges in implementing turnover include various epidemiological 
constraints that may need to be imposed as well as the data informing such constraints.
For example, the relative size of subgroups may need to remain constant when data 
suggest the size (for example of the population of sex workers) has remained stable over time. % cite
Other constraints include the duration of a high-risk period of one's sexual life-course.  %list the epidemiological constraints discussed in the system [in same order as in system] to foreshadow the unified approach. % cite each if possible or leave as [REF] for now, and I will help you find the right refs when paper is with co-authors
A related problem is that in trying to address each subgroup's period of higher risk
(for example, formal sex work), models end up with imbalanced flows which [SM: what is the problem then?]
can make it difficult to generalize as number of groups expand - require more parameters??? require a burn-in period???

%Second, as shown by \citet{Boily2015}, naive selection of turnover rates can result in imbalanced flows,		
% causing risk groups to initially change size over time. 	% i am unclear why this is referred to as 'niavie', and we do want risk groups to change over time? i.e. this rationale as a 'challenge' is unclear/vauge. Also - try to be careful in your tone of language as it reads as editorializing (e.g. niavie --> what does that mean? what is not niavie then?). I recommend writing in more objective and precise language and to avoid sounding negative but rather - just state the facts clearly. I also think you have to be really careful as all the papers cited above (except Henry, Stigum), include HIV-attributable mortality and so the flows will necessarily be imbalanced because we do not want to replace 1:1 - that creates a massive artifact re: resupply of susceptibles (as shown by some work Anna and Nasheed did already).

\par
In this paper, we aim to examine the influence of risk group turnover 
on the TPAF of an illustrative sexually transmitted infection. First, 
we propose a solution to the challenge of capturing the above
epidemiological constraints while allowing for initial conditions (i.e. no burn-in period)
and flexibility as the number of risk groups expands. That is, we
introduce a unified framework for
parameterizing risk group turnover, based on available data.
We then examine the mechanisms by which turnover 
influences group-specific STI incidence and prevalence
(Experiment~1); and how inclusion/exclusion of turnover influences
the values of other parameters related to heterogeneity during model fitting
(Experiment~2). Finally, we compare the TPAF of the highest-risk groups 
estimated from (i) two different settings (one with and one without turnover), 
and (ii) two models that reproduce one setting (one model with and one model without turnover).