% TODO: update for SIR, not STI focus
Core group theory has long underpinned the study of
epidemics of sexually transmitted infections (STI).
The theory posits that heterogeneity in
acquisition and transmission risk are sometimes necessary and sometimes sufficient for
an STI epidemic to emerge and persist.
This heterogeneity is often demarcated by identifying potential cores,
comprised of sub-populations or geographies,
where onward transmission risks are the highest,
such that the core's unmet STI prevention and treatment needs
sustain local epidemics~\citep{Yorke1978,Gesink2011}.
\par
Mathematical models of STI transmission include heterogeneity in risk
by stratifying the modelled population by features such as
the partner change rate, and levels of sexual mixing between subgroups, and partnership types%
~\cite{}. % TBD
The implications of including heterogeneity,
as compared to assumptions of homogeneity, include
higher basic reproductive ratios $R_0$, and
lower overall STI prevalence
(provided the latter still results in $R_0 > 1$)~\citep{Boily1997}.
% Anderson & May; Watmough, VanDeDriesse
$R_0$ and overall STI prevalence are further influenced by
mixing between subgroups~\citep{Boily1997}.
Thus, models with more than two risk groups
are increasingly relevant for exploring epidemic nuance
and for aligning model outputs with programmatic decision support
-- i.e.\ prioritization of specific interventions for specific risk groups~\cite{}. % TBD
\par
Less often discussed or included in STI transmission models
is the influence of movement of individuals between risk groups,
which we herein refer to as ``turnover''.
For example, a period of higher risk could represent
the average duration in sex work,
which is often associated with larger number of sexual partners as paid clients,
and other STI-associated vulnerabilities~\cite{}. % Watts
It could also represent periods of higher partner change
outside the formal sex work.~\cite{}. % TBD
\citeauthor{Stigum1994} modelled movement between risk groups as a form of ``migration'',
and showed that % TBD - expand; how turnover effect similar to proportional mixing
and thus, had nearly as large an influence on overall STI prevalence
as sexual mixing between sub-groups~\citep{Stigum1994}.
It has also been shown that rates of movement between risk groups
can play an important role during estimation of intervention impact,
following model fitting to calibration targets,
since % TBD - expand
~\citep{Eaton2014}.
\par
Yet, implementation of risk groups and turnover in recent models vary widely,
from no modelled risk groups to
seven risk groups with highly context-specific turnover~\citep{Boily2015}.
A common challenge in structuring and parameterizing STI models
are considerations on how to best incorporate
turnover and duration of periods of risk
using available data.
Models require parameters on transition rates between risk groups,
but must also content with considerations such as
stability in the relative size of risk groups over time.
\par
First, estimating the rates of movement between groups
directly from cross-sectional survey data is difficult,
and typically requires strong assumptions.
% TBD - ensure we address this challenge in paper else move this to discussion
Second, ensuring the relative sizes of risk groups
do not vary dramatically over time requires
careful selection of rates of turnover among groups,
or other compensatory parameters.
\par
% TBD - introduce specific previous approaches
\par
% TBD - introduce specific objectives
%We therefore draw on prior work to propose a unified framework for
%defining and parameterizing risk group dynamics.

% NEW:
% synonyms:
% - episodic risk\dfrac{num}{den}
% - migration (Stigam)

%
% OLD:
%
%Building on previous work,
%we then leverage this framework to explore the influence of
%Core group theory has long underpinned modelling of STI epidemics.
%It generally describes maintenance of an epidemic by
%high levels of exposure in a small subset of individuals,
%relative to a larger general population with less exposure~\cite{Yorke1978}.
%When considering one or more core groups
%-- i.e.\ a population composed of several risk-stratified groups --
%the epidemic characteristics are known to depend on
%the group sizes, relative exposures, and rates of sexual mixing between groups.
%Less discussed is a similar dependence on
%movement of individuals between risk groups, which we call ``turnover''.%
%\footnote{In early works, such as \cite{Stigum1994},
%  movement of individuals between risk groups is often called ``migration'';
%  in order to avoid confusion with population entry / exit,
%  we prefer the term ``turnover''.}
%This turnover of individuals between risk groups has a similar effect to
%sexual mixing between groups, making it
%an important feature to include in representative models~\cite{Stigum1994}.
%\par
%Models with more than two risk groups
%are increasingly relevant for exploring epidemic nuance
%and for aligning model outputs with programmatic decision support
%-- i.e.\ prioritization specific interventions for specific risk groups. % \cite{?}
%Yet, implementations of risk groups and turnover in recent models vary widely,
%from no modelled risk groups~\cite{Estill2012,Barnighausen2012}
%to seven risk groups with highly context-specific turnover [[best Mishra 7-groups citation]].
%In fact, many HIV models still do not consider turnover,
%despite the fact that rates of movement between risk groups
%can play an important role during estimation of intervention impact
%following model fitting to calibration targets~\cite{Eaton2014}.
%\par
%Two major challenges exist to implementing turnover,
%which may help explain its inconsistent usage.
%First, unlike behavioural parameters,
%estimating the rates of movement between groups
%directly from cross-sectional survey data is difficult,
%and typically requires strong assumptions.
%Second, ensuring the relative sizes of risk groups do not vary
%dramatically over time requires
%careful selection rates of turnover among groups,
%or other compensatory parameters.
%Prior works have generally solved this problem \textit{ad hoc},
%without providing a generalized approach,
%while some simply rely on a ``burn-in'' period,
%which permits equilibriation of risk group sizes due to turnover dynamics
%before introduction of the infection.
%\par
%We therefore expect that a unified framework for
%defining and parameterizing risk group dynamics
%would be of great use.
%We present such a framework here,
%and draw direct links to modelling assumptions and relevant sources of data.
%Building on previous work by~\citet{Stigum1994},
%we then leverage this framework to explore the impact of
%several risk group implementations on model outputs (incidence, prevalence)
%in a representative model.

% < move to intro >
% The overall influence of risk group turnover in an epidemic is not straightforward.
% On one hand, movement of infected individuals from high to low risk groups
% increases disease penetration in the lower risk groups; % \cite{}
% however, turnover also decreases the duration of risk exposure
% among individuals in the higher risk groups. % \cite{}
% </ move to intro >
% Moreover, since the average exposure experienced by each group
% is directly affected by the duration of infectiousness $\delta_I$, we also explore
% the sensitivity to treatment rate $\tau = {\delta_I}^{-1}$ on this result.