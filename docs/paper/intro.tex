Heterogeneity in transmission risk is a consistent characteristic of
epidemics of sexually transmitted infections (STI) \citep{Anderson1991}.
This heterogeneity is often demarcated by identifying
specific populations whose risks of acquisition and onward transmission of STI are the highest,
such that their specific unmet prevention and treatment needs
can sustain local epidemics of STI \citep{Yorke1978}.
% That is, disproportionate risks experienced by a smaller subset of a population,
% such as individuals engaged in sex work,
% could sustain onward transmission.
% JK: The above sentence (commented) feels a bit repetitive? Can we remove?
Disproportionate risk can be conferred in several ways at the
individual-level (higher number of sexual partners), 
partnership-level (reduced condom use within specific partnership types), 
or structural-level (stigma as a barrier to accessing prevention and treatment services)
\citep{Baral2013}.
The contribution of high risk groups to the overall epidemic
can then be used as an indicator in the appraisal of STI epidemics,
helping to guide intervention priorities
\citep{Shubber2014,Mishra2016}.
% SB: Rather than saying important, would explain. Ie, try to avoid editorializing in the
% introduction and just provide empiric evidence as to why this is the case.
\par
Traditionally, contribution to an epidemic was quantified using either:\ %
the classic \textit{population attributable fraction} (PAF)
% SM: not really. the PAF is from epidemiology - and is not exactly what the MOT model generates.
% JK: Gotcha.
via the relative risk of incident infections within a risk group
versus the rest of the population
and the relative size of the risk group \citep{Hanley2001};
or the distribution of new infections across subsets of a population
\citep{Case2012,Mishra2014}.
So when small risk groups experience disproportionately higher rate of
incident infections -- e.g. 5 percent of a population acquire 30 percent
of STI infections -- contribution is interpreted as 5 percent of the population contributing to
30 percent of all infections \citep{Pruss-Ustun2013}.
% SB: Also don’t think quotations are ideal. Could just explain contribution as the traditional
%     PAF and that also sets up nicely for why the tPAF is an advance.
% SB: Really think the PAF as an old concept needs to be introduced in a sentence and then can
%     include the TPAF as to why it is an advance for infectious diseases.
% JK: Is this better now?
% SM: Not really accurate :) --> the PAF is different from the tPAF but *not just because of time-period*.
%     PAF is generated from observational data and does not account for indirect/onward/secondary transmission.
%     The MOT modeled esiimates come from a binomial probablity term to generate incident estimates,
%     not traditional PAF equations per se which work with empirical prevalent or incident estiamtes.
However, the classic PAF does not account for chains of (indirect) transmission, and has been
shown to underestimate the contribution of some higher-risk groups to cumulative
STI infections, especially over time \citep{Mishra2014}.
Thus, transmission models are increasingly being used to quantify
contribution by accounting for indirect transmission and estimating
the \textit{transmission population attributable fraction} (tPAF).
The tPAF is estimated by
simulating counterfactual scenarios where transmission
between specific subgroups is stopped, and
the relative difference in cumulative infections in the total population
over various time-periods is measured \citep{Mishra2014,Mukandavire2018}.
Transmission can be stopped by
setting susceptibility and/or infectiousness to zero in the model \citep{Mishra2014}.
The tPAF is then interpreted as
the fraction of all new infections that stem, directly and indirectly, from
a failure to prevent acquisition and/or to provide effective treatment
in a particular risk group \citep{Mishra2016,Mukandavire2018,Maheu-Giroux2017}.
\par
There is limited evidence on how model structure 
might influence the tPAF of higher risk groups
\citep{Mishra2016,Mukandavire2018,Maheu-Giroux2017},
% SM: cite: look at most of the papers that estimated tPAF
%     (including mine, MC's, Mukandavire, Maheu-Giroux https://www.ncbi.nlm.nih.gov/pubmed/29137859),
%     and you will see that they did not explicitly ask this question - so cite all of them here.
%     you will see that they did look at other potential sources of heterogeneity though
%     (like size of the highest-risk groups, etc.).
%     Also - if any of them looked at sensitivity of their parameters related to turn-over
%     (e.g. duration in sex work) -- what did they find?
%     could be good to mention in discussion section of paper.
especially movement of individuals between risk groups,
an epidemiologic phenomenon  that is well-described
in the context of sexual behaviour \citep{Watts2010}.
% SB: Again, editorializing and is setting up potentially adversarial relationships with reviewers.
%     I just don’t think you need the sentence and can still highlight the need for turnover in models.
% JK: I agree it is hard to express this thought in a friendly / objective way,
%     but I really think its worth noting.
% SM: on re-reading the intro, I think Stef is right.
%     We are using circular logic to say that it is important without first showing evidence
%     -- which we aim to actually do.
%     see edits. I think it could be good to read more from the tPAF papers :).
Such movement is often referred to in the STI epidemiology literature as
\textit{turnover} \citep{Watts2010}.
For example, turnover may reflect entry into or retirement from formal sex work,
or other periods associated with higher STI susceptibility and onward transmission
due to more partners and/or vulnerabilities
\citep{Marston2006,Watts2010}.
Risk group turnover has been shown to
influence the predicted equilibrium prevalence of an STI \citep{Stigum1994,Zhang2012};
the fraction of transmissions occurring during acute HIV infection \citep{Zhang2012};
the basic reproductive number $R_0$ \citep{Henry2015}; and
the coverage of antiretroviral therapy required to achieve HIV epidemic control \citep{Henry2015}.
Yet how, and the extent to which, turnover influences tPAF has yet to be examined.
\par
There is variability in how turnover has been previously implemented
\citep{Stigum1994,Koopman1997,Eaton2014,Boily2015},
in large part because of four main assumptions or epidemiologic
constraints surrounding movement between risk groups.
For example, in the context of turnover, the relative 
size of specific populations in the model 
may be constrained to remain constant over time
\citep{Stigum1994,Koopman1997,Eaton2014},
such as the proportion of individuals who sell sex.
% SM: cite one of the models that tried to do this
%     data source for paid sex among males = Demographic Health Surveys.
%     check out some DHS reports in same country over time);
%     data sources for FSW are repeated mapping/enumeration surveys in same location).
%     But in reading the introduction a few times, I would suggest mentioning the possible constraints here,
%     and then listing the data sources in the methods (and discussion).
Second, some individuals may enter into high risk groups at an early age,
and subsequently settle into lower risk groups;
thus the distribution of risks among individuals entering into the transmission model
may be assumed to be different from
the distribution of risks among individuals already in the transmission model
\citep{Eaton2014}.
% SM: I do not understand this sentence. I tried to edit it but it is still unclear to me.
%     Please revise for clarity and simplicity - use more words or 2 sentences if needed.
%     SM: cite on of the prior models if they did this
% JK: how does it read now?
% SM: Re: data sources = perhaps Transitions and then the published data/studies on size of FSW population in Kenya?
%     or perhaps what about what you have from Eswatini?
Third, turnover may be constrained to reflect the average duration of time spent 
within a given risk group \citep{Boily2015},
such as duration engaged in formal sex work \citep{Watts2010}.
Finally, turnover could reflect data on how sexual behaviour changes
following exit from a given risk group \citep{Boily2015}.
% SM: I could not understand the sentence so did not know how to edit.
%     Do we mean by keeping track of former sex worker for example? like we did in Boily2015?
% JK: Pretty much - so what proportion of retiring FSW enter into low risk vs medium risk, etc.
%     after retirement.
Most prior models used some combination of these constraints,
based on their specific data or research question,
% and implementation issues, such as
% the need for a burn-in period to establish a demographic steady-state before introducing infection
% \citep{Boily2015}.
% JK: The above idea is pretty advanced and not adding much here in the intro I think.
%     So I moved it to the discussion only.
but to date there is no unified approach to modelling turnover.
\par
In this study, we explored the mechanisms by which turnover
may influence the tPAF of a high risk group
using an illustrative STI model
with treatment-induced immunity and without STI-attributable mortality.
First, we developed a unified approach to
implementing turnover based on epidemiologic constraints.
We then sought the following objectives:
1)~understand the mechanisms by which turnover
influences group-specific STI prevalence and ratios of prevalence between risk groups;
2)~examine how inclusion/exclusion of turnover in a model influences
the level of risk heterogeneity inferred during model fitting; and
3)~examine how inclusion/exclusion of turnover in a model influences
the estimated tPAF of the highest risk group
after model fitting to a particular setting.
