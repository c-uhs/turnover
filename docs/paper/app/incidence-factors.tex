Substituting the proportional mixing definition of $\rho_{ik}$ into
the incidence equation, Eq.~(\ref{eq:foi}), we have:
\begin{align}
\lambda_i
  &= C_i \sum_k \rho_{ik} \beta \thinspace \frac{\mathcal{I}_k}{\mathcal{X}_k}
    \nonumber\\
  &= C_i \beta \sum_k
    \frac{C_k \mathcal{X}_k}{\sum_{\mathrm{k}}C_{\mathrm{k}}\mathcal{X}_{\mathrm{k}}}
    \frac{\mathcal{I}_k}{\mathcal{X}_k}
    \nonumber\\
  &= C_i \beta \underbrace{
      \frac{\sum_k C_k \mathcal{I}_k}{\sum_{\mathrm{k}}C_{\mathrm{k}}\mathcal{X}_{\mathrm{k}}}
    }_{f}
\end{align}
We can factor the term $f$ as:
\begin{align}
  f
  &= \frac{\sum_k C_k \mathcal{I}_k}{\sum_{\mathrm{k}}C_{\mathrm{k}}\mathcal{X}_{\mathrm{k}}}
     \nonumber\\
  &= \frac{\sum_k C_k \mathcal{I}_k}{\sum_k \mathcal{I}_k}
       \cdot
     \frac{\sum_k \mathcal{I}_k}{\sum_k \mathcal{X}_k}
       \cdot
     \frac{\sum_k \mathcal{X}_k}{\sum_k C_k \mathcal{X}_k}
\intertext{which we recognize as the following terms:}
  &= \hat{C}_{\mathcal{I}} \cdot \hat{\mathcal{I}} \cdot \hat{C}^{-1}
\end{align}
Namely,
\begin{enumerate}
  \item $\hat{C}_{\mathcal{I}}$ is the average number of partners among infectious individuals
  \item $\hat{\mathcal{I}}$ is the proportion of the population who are infectious (overall prevalence)
  \item $\hat{C}$ is the average number of partners among all individuals (constant)
\end{enumerate}
Therefore, only two non-constant factors control incidence per susceptible:
1) the average number of partners among infectious individuals $\hat{C}_{\mathcal{I}}$, and
2) overall prevalence $\hat{\mathcal{I}}$.
The product of these factors $\hat{C}_{\mathcal{I}}\,\hat{\mathcal{I}}$,
scaled by $\beta \, C_i / \hat{C}$,
then gives $\lambda_i$.
In fact, the incidence in each group individually is proportional to
incidence overall, as $C_i$ is only factor depending on $i$.