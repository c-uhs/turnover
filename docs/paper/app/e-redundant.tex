Whenever it is assumed that risk groups do not change size,
$G$ rows of the form shown in Eq.~(\ref{eq:eg-basis})
are added to $\bm{b}$ and $A$:
\begin{equation}\tag{\ref{eq:eg-basis}}
\bm{b} = \left[\begin{array}{c}
\nu x_1 \\ \nu x_2 \\ \nu x_3
\end{array}\right];\qquad
A = \left[\begin{array}{ccccccccc}
 \nu  & \cdot & \cdot & -x_1  & -x_1  &  x_2  & \cdot &  x_3  & \cdot \\
\cdot &  \nu  & \cdot &  x_1  & \cdot & -x_2  & -x_2  & \cdot &  x_3  \\
\cdot & \cdot &  \nu  & \cdot &  x_1  & \cdot &  x_2  & -x_3  & -x_3  \\
\end{array}\right]
\end{equation}
After multiplying by $\bm{\theta}$, these $G$ rows can be row-reduced by summing to obtain:
\begin{equation}
\begin{aligned}
\left[ \nu x_1 + \nu x_2 + \nu x_3 \right] &= 
\left[ \nu e_1 + \nu e_2 + \nu e_3
+ 0\,\phi_{12} + 0\,\phi_{13} + 0\,\phi_{21} + 0\,\phi_{23} + 0\,\phi_{31} + 0\,\phi_{32}
\right]\\
\nu \left[ x_1 + x_2 + x_3 \right] &= 
\nu \left[ e_1 + e_2 + e_3 \right]
\end{aligned}
\end{equation}
which therefore implies that $\sum_{i} x_i = \sum_{i} e_i$,
or equivalently $\sum_{i} \hat{x}_i = \sum_{i} \hat{e}_i = 1$.
Thus, it is redundant to specify all $G$ elements of $\hat{e}$,
as the final element will be dictated by constant group size constraints.