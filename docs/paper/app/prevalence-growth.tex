The aim of this derivation is to relate the equilibrium prevalence $P$
to the rate of population entry $\nu$
for a simplified epidemic model.
A homogeneous ($G = 1$) SI model is assumed.
The expression for prevalence is given by:
\begin{equation}
P = \frac{\mathcal{I}}{\mathcal{S}+\mathcal{I}}
\end{equation}
At equilibrium, prevalence is unchanging.
The expression for $\frac{d}{dt}P$ can be obtained by the quotient rule:
\begin{equation}
\frac{d}{dt}P = \frac{\Big( \frac{d}{dt}\mathcal{I} \Big) \Big( \mathcal{S}+\mathcal{I} \Big)
  - \Big( \mathcal{I} \Big) \Big( \frac{d}{dt} (\mathcal{S}+\mathcal{I}) \Big) }
{{\Big(\mathcal{S}+\mathcal{I}\Big)}^{2}}\\
\end{equation}
which simplifies to:
\begin{equation}
\frac{d}{dt}P = \frac{ \mathcal{S} \lambda - \mathcal{I} \nu } {\mathcal{S} + \mathcal{I}}
\end{equation}
Separating terms in the numerator, this can be further re-written as follows:
\begin{equation}
\frac{d}{dt}P = \lambda (1 - P) - \nu P
\end{equation}
Moreover, considering the simplified system,
the force of infection can be written as
$\lambda = \beta \frac{\mathcal{I}}{\mathcal{S}+\mathcal{I}} = \beta P$.
Thus, the rate of change of prevalence is given by:
\begin{equation}
\frac{d}{dt}P = \beta P (1 - P) - \nu P
\end{equation}
Now, at equilibrium, $\frac{d}{dt}P = 0$, and so $\beta P_{eq} (1 - P_{eq}) = \nu P_{eq}$,
which simplifies to:
\begin{equation}
\frac{\nu}{\beta} = (1 - P_{eq})
\end{equation}
Therefore, as the population growth rate $\nu$ increases,
the equilibrium prevalence $P_{eq}$ must decrease.
