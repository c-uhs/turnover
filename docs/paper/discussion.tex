Epidemic models have rarely considered turnover of individuals among risk groups.
%Perhaps this is because it is seen as either infeasible or inconsequential to do so.
%In this work, we have hopefully shown that it is neither.
%\par
In this work we have presented a new framework for modelling such turnover.
This framework subsumes and generalizes previous approaches to this task,
and clarifies requisite data and assumptions.
We then used this framework to simulate turnover in
an illustrative model of an STI epidemic,
and explored how and why turnover influences the following model outputs:
group-specific equilibrium prevalence and incidence;
inferred heterogeneity during model fitting; and
projected TPAF of a high-risk group.
% --------------------------------------------------------------------------------------------------
\paragraph{Turnover framework}
Our framework uses a flexible system of linear equations to resolve
the rates of turnover $\phi$ and
the proportions of individuals entering into each risk group upon entry into the model $\bm{\hat{e}}$.
This approach has several advantages.
First, as shown in Section~\ref{ss:prev-approach},
this formulation subsumes previous implementations of turnover,
and thereby hopefully clarifies relationships among them.
Second, it facilitates parameterization of turnover among any number of risk groups
in a scalable, flexible way.
Finally, the framework makes the parameterization of turnover systems more transparent,
through explicit incorporation of constraints as rows in the linear system.
We have also summarized the major constraints which could be used,
so that future modellers may choose among these
based on available data and epidemic context.
It is our hope that this framework will facilitate
inclusion of turnover in future epidemic models.
\par
The major data needs implicated by this approach are then as follows.
First, the proportion of total individuals in each risk group $\bm{\hat{x}}$ is required.
This can generally be obtained from representative surveys of the modelled population,
and from key populations surveys for under-represented risk groups.
Similarly, the risk behaviours of individuals immediately after entering the model
can be used to specify elements in $\bm{\hat{e}}$.
Such behaviours could be approximated using data from the youngest age group
in the above surveys,
or using surveys among young people specifically.
Third, it is useful to know the average duration of individuals within each risk group $\bm{\delta}$.
Among key populations, survey questions like
``For how many years have you been a sex worker?''
can be used to obtain estimates of duration in sex work \citep{Watts2010}.
Among the general population,
longitudinal data on individual-level changes in risk behaviour
can be used to estimate duration in each risk group,
although cross-sectional surveys unfortunately rarely capture such information.
Finally, similar data, if available, can also be used to estimate
the rates of transition between specific risk groups ($\phi$).
% --------------------------------------------------------------------------------------------------
\paragraph{Influence of turnover on incidence \& prevalence}
In Experiment 1, we showed that turnover can have a large influence on
the equilibrium STI incidence and prevalence
projected among different risk groups, as well as overall.
In general, turnover yields a net movement of infectious individuals
from high to low risk.
In an analogy with heat transfer mechanics,
incidence represents infection ``conduction'',
whereas turnover represents infection ``convection'',
and both processes represent important mechanisms of infection spread.
As such, it would be important to simulate turnover in epidemic models
if such dynamics are known or assumed to be present in reality.
\par
Turnover also acts to homogenize the risk experienced by individuals in the model.
Recall from core group theory that
risk heterogeneity is sometimes necessary for persistence of an STI epidemic.
Therefore, homogenization of risk through turnover
could make epidemic control easier to achieve,
compared to a setting with less or no turnover,
as demonstrated by \citet{Henry2015}.
We corroborate this hypothesis in Figure~\ref{fig:surface-prevalence-all},
where we see that increasing turnover reduces
the universal treatment rate $\tau$ required to achieve zero prevalence ($R_0 < 1$).
% --------------------------------------------------------------------------------------------------
\paragraph{Implications for interventions}
Mathematical models are often used to quantify
the contribution of high-risk groups to overall transmission,
in order to help prioritize interventions.
For example, the transmission population attributable fraction (TPAF)
represents the maximum proportion of infections which could be averted
with a perfect intervention in one risk group.
In Experiment 3.1, we showed that the TPAF of a high risk group
was lower in a setting with turnover than in a setting without.
This was attributable to a higher prevalence ratio
in the absence of the homogenizing effects of turnover,
resulting in more transmissions from the high risk group.
Thus, reaching the high risk group with interventions
would be more important in a setting without turnover.
\par
However, in many epidemic models,
the values of uncertain parameters are inferred
by fitting to group-specific prevalence data
before analysis of interventions.
In Experiment 2, we showed that
the heterogeneity in risk-associated parameters inferred via model fitting
was higher in a model with turnover than in a model without.
As a result, the TPAF of the high risk group
was higher in the model with turnover than in the model without (Experiment 3.2).
Therefore, if turnover dynamics present in reality
are not simulated in the associated transmission model,
the TPAF of high risk groups may be underestimated.
That is, if risk group turnover is not simulated in fitted models,
the impact of interventions prioritizing high risk groups
may be systematically underestimated.
% --------------------------------------------------------------------------------------------------
\paragraph{Limitations}
There are several limitations to this work,
and the results shown here
are specific to the model structure and assumptions employed.
For example,
we assumed proportional contact formation across risk groups,
whereas a degree of so-called ``assortative mixing''
by risk group is often assumed.
We briefly explored the potential influence of assortative mixing
on the results of Experiment 3 in Appendix~\ref{aaa:what-if-asso}.
We found that the TPAF of the high risk group
would likely be further underestimated under assortative mixing
if turnover is omitted from fitted models.
We also did not consider disease-attributable mortality,%
\footnote{Since disease-attributable mortality
  disproportionately affects higher risk groups,
  risk groups may change size after including this dynamic in the model.
  The proposed turnover framework does not yet include mechanisms to
  re-balance such changes to group sizes.
  In fact, such re-balancing would imply the existence of ``market'' forces
  to maintain specific risk group sizes,
  and it's not clear if such forces exist,
  or if they dominate individual-level factors
  driving movement between risk groups.}
or reinfection.
However, following similar analyses in
Appendices~\ref{aaa:what-if-mort}~and~\ref{aaa:what-if-sirs}, respectively,
we found that inclusion of disease-attributable mortality or reinfection would
also result in greater underestimation of high risk group TPAF
by models without turnover.
That is, when
assortative mixing, disease-attributable mortality, or reinfection are considered,
fitted models without turnover would likely underestimate
the importance of reaching high risk groups with interventions
even more than shown in the original Experiment~3 results.
\par
We have also not explored:
more complex or infection-specific natural disease history,
which may include periods of reduced or increased infectivity;
interactions between risk group turnover and age stratifications,
which are likely important;
or sensitivity of the results shown to other model parameters,
such as the number and sizes of risk groups $\bm{\hat{x}}$,
the probability of transmission $\beta$,
and population entry and exit rates $\nu$~and~$\mu$.
Finally, in examining the influence of turnover on STI prevalence and incidence,
we have focused on systems at equilibrium,
motivated by our analysis of TPAF.
Turnover likely also has an important influence on
temporal dynamics of infection transmission.
For example, \citet{Henry2015} derived
the analytical relationship between turnover
and the basic reproduction number $R_0$,
and showed that $R_0$ decreases monotonically with turnover.
These limitations and gaps in our analysis thus represent
interesting opportunities for future work.