Under construction!
% % JK: from \ref{ss:res-turnover}
% These trends echo previous observations \citep{Zhang2012,Henry2015}
% but highlight the dependence of the threshold effect on the duration of infectiousness.
% Furthermore, this implies that the rate of treatment required to achieve epidemic control
% is affected by turnover.
% Namely, since turnover acts to erode the core-group effect,
% the rate of treatment required to achieve epidemic control will be
% lower with turnover than without.
% Models which fail to capture turnover dynamics might therefore be liable to
% overestimate the treatment rate required to achieve epidemic control.
% % repetitive wording here ...

% Comparisons with Zhang2012 and Henry2015
% - incidence is related to R0, but only for prevalence ~ 0
%   - also: contributions here help explain by sub-group analysis
% - speak to early infections
%   - 

% noting results w.r.t. fitted models: should probably fit rates of turnover
% in conjunction with other parameters
% methods here help ensure group sizes remain fixed despite changes in turnover rates

% hard to capture turnover parameters from cross-sectional surveys
% suggestions on how to do this?

% Henry2015 note that major qualitative results regarding effect of turnover
% were not changed by assumed proportionate vs assortative patterns of sexual mixing.

% re. exp:1.3
%It is worth comparing these results with the findings of \cite{Henry2015},
%who observed that increasing turnover consistently decreased $R_0$ in a model of HIV.%
%\footnote{\cite{Henry2015} use a parameter called the ``re-selection rate'' $\omega$
%  to control the magnitude of turnover;
%  this rate is functionally equivalent to
%  a uniform scaling factor of the turnover rates $\phi$ used here.}
%\citeauthor{Henry2015} highlight the reduction of
%contact rates among infected individuals via turnover in their explanation of
%the effect of turnover on $R_0$.
%However, they also assume that acute infections are more infectious than chronic,
%complicating the relationship between turnover and $R_0$.
%Such an assumption is clearly plausible for many infections,
%but it's not clear whether their conclusions regarding the effect of turnover on $R_0$
%hold without this assumption.

% LIMITATIONS:
% - proportional mixing only
% - 

% Conduction vs convection: parallels from other fields

%% SM (from exp): that we restricted our analyses to proportionate mixing is an important limitation
%% to talk about in discussion (and reference work on assortative vs. proportionate mixing from Boily, Garnett, etc.)
%% and in limitations, remember to hypothesize (and cite if available) how this assumption could influence the findings

%% SM: in the discussion - point out that duration in the highest risk group is the most common empiric estimate we would have available, vis a vis sex work for example and cite Watts et al

%% SM: move this to discussion --> "TPAF estimates the proportion of cumulative new infections which are attributable to
%% prevention gaps among a specific population.%
%% this information should be in the methods ---> \footnote{To estimate TPAF of the high risk group,
%%  transmission ``from'' the high risk group is turned off, not ``to''.}