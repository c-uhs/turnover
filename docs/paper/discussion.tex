We developed a new framework for modelling turnover of individuals among risk groups,
based on a flexible combination of data-driven constraints.
We then used this framework to explore the influence of turnover on
the contribution of the highest risk group to onward transmission in an
illustrative STI epidemic without STI-attributable mortality.
We found that failure to model turnover when simulating settings with turnover
could result in underestimation of
the contribution of the highest risk group to onward transmission.
% SM: in original research papers, usually the summary opening paragraph should
%     succintly (1-2 line at most) say what we found, not what we did. Otherwise reads as repetitive.
% JK: A lot of papers in Epidemics journal seem to start discussion with paragraphs like:
%     "Currently, estimation of X is performed using method / model Y.
%     But method Y has problems a, b, c.
%     Our objective was to overcome these challenges,
%     and we did with a new proposed method Z."
%     That's why I was hoping to highlight again the fact that turnover is not often modelled here.
% --------------------------------------------------------------------------------------------------
\paragraph{Influence of turnover on incidence \& prevalence}
% SM: check journal style if subheadings are common/used in Discussion section.
%     usually not in most journals in the health-field, but might be ok for Epidemics & ID Modeling.
%     I personally would remove the subheadings.
% JK: It seems they are used in Epidemics journal about 1/2 the time.
%     I would prefer to keep them, as I find they help organize the discussion points clearly.
%     Just curious, how come you would prefer to remove them?
We found that turnover influences the overall equilibrium STI incidence and prevalence
in a pattern consistent with previous works \citep{Stigum1994,Zhang2012,Henry2015}.
However, unlike previous works, we also illustrated the influence of turnover on
group-specific prevalence, and demonstrated mechanistically how this occurs.
% Our findings are not new and indeed, are analogous to heat transfer mechanics [REF].
% SM: please cite (even the foundational textbook, but ideally a paper that explains
%     the mechanisms and terms conduction/convection.
%     A note that PhD examiners are particularly picky about citations
%     (I have seen students get major corrections for under-citing in the last few exams I have chaired).
% For example, incidence can be thought of as infection ``conduction'',
% SM: if putting anything in quotation marks in Intro or Discussion, must be cited,
%     unless you are using that term specifically for the study (e.g. turnover).
%     Is there a reason for quotation marks? What other meaning could those terms in quotation marks have?
% whereas turnover would represent infection ``convection'',
% and both processes represent important mechanisms of infection spread.
% SM: this is good, but I am worried that if people do not know heat transfer or fluid mechanics,
%     then the 'so what' will go missing.
%     I tried to edit to convey the more 'cross-disciplinary' nature of your work here
%     by trying to frame it within the wider literature.
% As such, it would be important to simulate turnover in epidemic models	
% if such dynamics are known or assumed to be present in reality
% SM: this is not a clear conclusion that can be drawn from the findings discussed in the paragraph.
%     What is defined as 'important'?
%     Also - a more careful version of a sentence like this is better suited for after you discuss the TPAF stuff.
% JK: Re-reading this ``conduction'' / ``convection'' anaology (and all the edits), it feels like:
%     - a sudden reference to something obscure, which needs a lot of explaining
%     - it doesn't add much to our key messages
%     - it detracts from the flow of other ideas
%     So, I'm just going to leave out for now -- thoughts?
For example, we showed that some infections among the low risk groups
were acquired by individuals during a previous period of higher risk
(see Appendix~\ref{aa:inf-phi-vs-inc} for additional results).
We can summarize this influence of turnover,
including reduced ratio of STI prevalence between the highest and lowest risk groups,
as ``reducing heterogeneity in risk'' via movement of individuals between risk groups.
% SM: use quotation marks here b/c this term has many meanings, and you use it in a new way here.
%     Perhaps as the whole group of co-authors but I wonder about using 'reduce heterogenetiy'
%     instead of 'homogenizes' after this first time use.
%     'Homogenizes' has some negative connotations and I worry about
%     folks working in LGBTQ and KP fields reading this and taking it out of context
%     because the paper is so methodological that the term gets lost in the weeds.
%     I would suggest that 'reduces/less heterogeneity' is equivalent here and less worrisome.
% SM: will be glad to never read an instructive 'recall' from you again in an original research paper :)
%     - only in your future lecture notes and if you write a seminar teaching paper or textbook!
% JK: LOL loud and clear! :)
\citet{Henry2015} demonstrated that
such reduction in risk heterogeneity through turnover
decreases the basic reproductive number and thus,
means epidemic control could be easier to achieve.
Our findings on the mechanisms by which increasing turnover
reduces the STI treatment rate to achieve zero prevalence
further supports these insights from \citet{Henry2015}.
% --------------------------------------------------------------------------------------------------
\paragraph{Implications for interventions}
Our comparison of models with and without turnover,
calibrated to the same epidemic,
showed that if turnover exists in a given setting
but it is ignored in a model,
the TPAF of high risk groups will be underestimated by the model.
% LW: I think this is a valid conclusion
This is because heterogeneity in risk
must be higher in the presence of turnover than in a model without turnover
in order to produce the same epidemic features.
Although we examined a single parameter to capture risk (frequency of partner change),
these insights would be generalizable to
any other component of susceptibility or infectivity,
because the risk per susceptible individual (force of infection) includes both
biological transmission probabilities and frequency of partner change \citep{Anderson1991}.
In the context of models with assortative mixing
(individuals are more likely to form partnerships with
other individuals in their risk group),
the difference between the TPAFs estimated by the model
with vs.\ without turnover is therefore expected to be even larger.
% SM: have not reviewed results again, so make sure this edit is correct :).
%     I also do want to review that part more carefully to see the checks on the mixing matrices :)
%     but will do that when paper is with co-authors and after my Sep 6 many deadlines.
%     Just make sure that what I wrote here is what you showed in results/appendix really clearly.
%     I remember the figure and think yes, but did not review carefully yet. 
% JK: yup - all good! In the appendix I used the usual Garnett / Nold approach
%     with epsilon = 0.5.
% JK: update: well, now this is removed from the appendix,
%     so am just suggesting at the expected trend,
%     though obviously we've explored this and shown that
%     the TPAF gap is even larger under assortative mixing.
The public health implication of models ignoring turnover
which is present in reality is that
the TPAF of high risk groups will be systematically underestimated,
potentially misguiding resources away from high risk groups.
Follow on research should quantify the size of
this potential bias in TPAFs generated from models without turnover,
and characterize the epidemiologic conditions under which the bias would be largest.
% --------------------------------------------------------------------------------------------------
\paragraph{Turnover framework}
% LW: I would keep this at the end of discussion before limitation section.
%     It just flows better with results if we discuss result implications first
%     before moving into method advance.
% JK: done.
% SM: this paragraph needs a lot of revision/work
%     and I think I should definitely review/edit the next version before submission.
%     I will not have time to edit before it goes to co-authors unfortuntely
%     - so please make a note for them to tell you whether they understood the key messages here.
%     Remember that a big part of this paragraph is written for epidemiologists/data folks
%     who are going to give modelers the data to parameterize turnover.
% JK: have given this an overhaul -- hopefully is more precise and clear now.
The proposed framework provides a flexible way to parameterize risk group turnover
based on available epidemiologic data and/or assumptions.
The approach has four advantages.
First, the framework defines how specific epidemiologic data and assumptions
can be used as constraints to help define rates of turnover
(Table~\ref{tab:constraints}).
These data and assumptions are further discussed in the next paragraph.
Second, the framework allows such constraints,
which take the form $b_k = A_k \bm{\theta}$
to be chosen and combined in a flexible way,
depending on which data are available, or which assumptions are most plausible.
While it is necessary that constraints do not conflict one another,
it is not necessary that a complete set of $G^2$ constraints be defined
(where $G$ is the number of risk groups),
since optimization techniques can be used to calculate, for example,
the smallest possible values of the parameters which satisfy the given constraints.
Third, this flexible approach also allows the framework to scale
to any number of risk groups, $G$.
Fourth, we have shown how several previous implementations of turnover
\citep{Stigum1994,Eaton2014,Henry2015}
can be recreated exactly using the proposed framework
(Appendix~\ref{aa:prev-appr}).
In so doing, we highlight which specific assumptions are
the same and which are different
across the different implementations.
\par
The major data needs of this approach include the following four categories.
First, the proportion of total individuals in each risk group $\hat{x}_i$ is required.
% SM: the System section would benefit from a table/flow-diagram of the constraints
%     and their epidemiological meaning + data needs.
% JK: have added this now in the latest version of System, and added reference here.
%     I did chose Table specifically though, vs flow-diagram,
%     since I don't think there should be a path / flow / order of constraint selection;
%     it's more so an unordered set of constraints, which just have to be sufficient
%     to yield a unique (exactly one) and consistent (no conflicts) solution \theta.
In the context of STIs, risk group size estimates may be obtained from
demographic health surveys \citep{DHS}, and from
mapping and enumeration of marginalized populations,
such as sex workers \citep{Abdul-Quader2014}.
For example, one risk group may be defined by
any engagement in casual sex within the past year,
and the corresponding proportion of the population
could be estimated from self-reported behaviours
in demographic health surveys \citep{DHS}.
% HM: overall, I think this para could be shorten and simplified. It reads a little bit long
% JK: Just trying to balance this with providing examples
Second, the proportion of individuals who enter into each risk group $\hat{e}_i$
upon entry into the model can be used to define additional constraints.
Such proportions could be obtained the same way as $\hat{x}_i$ above,
except using only data from individuals who recently became sexually active.
For example, among women who became sexually active in the past year,
what proportion also engaged in casual sex within the past year.
If data on sexual debut is not available,
then recent entry into sexual activity could be approximated using
a suitable age range.
Third, the average duration of time spent within each risk group $\delta_i$
can similarly be used to define additional constraints.
% SM: what do you mean by useful? are all 4 of these needed? or just 1? unclear from current paragraph
% JK: That's the thing about the framework (which is tricky to explain):
%     A combination of constraints is needed. Many combinations will work, but not all,
%     and the number of constraints you need (G^2) depends on the number of risk groups (G).
%     Hopefully this is more clear now in the above paragraph and in the revised System section.
Cross-sectional survey questions asked of female sex workers such as	
``for how many years have you been a sex worker?''
may be used to obtain estimates of duration in sex work,
with the recognition that such data are censured \citep{Watts2010}.
Longitudinal, or cohort studies
that track the self-reported sexual behaviour over time can also provide 
estimates of duration within variable periods of risk \citep{Fergus2007}.
Fourth, similar studies may provide data on
trajectories in risk behaviour of individuals over time,
which can be used to estimates transition rates $\phi$ or ratios of transition rates.
For example, upon retirement from sex work,
a proportion of former sex workers may enter into monogamous relationships,
while another proportion may continue to form multiple partnerships.
% LW: I would avoid the symbols (e.g., δ. x ....) in discussion section.
% JK: I like to keep them in case they are more recognizable than the definitions
%     and I don't think they take anything away if not.
% --------------------------------------------------------------------------------------------------
\paragraph{Limitations}
% JK: I changed a few references from "sex work" to "risk groups"
%     and "STI" to "infection" to keep comments a more generalized,
%     especially for Epidemics journal.
%     But let me know if you want to keep it more focused on STI & sex work.
% LW: I think underltying assumptions used in ur experiments should be highlighted especially
%     the third assumption.
% JK: Regarding "balanced" turnover?
%     = number of people moving between two groups is same in both directions.
%     I think there are already so many limitations,
%     and it really is just a way of avoiding "bias" in which turnover paths dominate.
%     So, I would prefer to leave out here...
%     I did add: health state does not affect turnover, since we had talked about that
There are six limitations of the study that are important when considering
the proposed turnover framework and the implications of our results.
First, our approach does not account for
infection-attributable mortality, such as HIV-attributable mortality.
It is well-established that HIV-attributable mortality will reduce the 
relative size of higher risk groups,
and that alone can cause an epidemic to decline \citep{Boily1997}.
As such, many models of HIV transmission
that include very small ($<3\%$ of the population)
high risk groups, such as female sex workers,
often do not constrain the relative size
of the sub-group populations to be stable over time
\citep{Pickles2013}.
% LW: Lacking a “however” statement following this limitation to justify in which case such
%     limitation is of less concern
% JK: I do mention below, after introducing both points,
%     that these should be subject of future work.
%     I know its kind of a cop-out but incorportating death in the current model
%     is difficult to explain.
% JK: re. below: I had briefly mentioned 'age' before, which maybe got lost in edits.
%     Just adding back here with comments on sex stratification,
%     as I think it is at least as important as sex,
%     but even more challenging to adapt the framework to consider it.
Second, we assumed a single-sex population and did not stratify by age.
In the context of real-world STI epidemics,
the relative size of risk groups may differ
by both sex and age,
such as the number of females who sell sex,
versus the number of males who sell sex,
versus the number of males who pay for sex (clients).
Future work on the proposed framework could incorporate
infection-attributable mortality, and age-sex stratifications into the model.
Third, we assumed that turnover rates were not affected by
the health status of individuals,
which could change the observed influence of turnover on
equilibrium STI incidence and prevalence.
While changes to health state may affect the sexual behaviour of individuals,
its not clear what overall patterns emerge at the population level.
Fourth, we did not include individuals becoming re-susceptible
-- an important feature of many STIs such as syphilis and gonorrhoea
\citep{Fenton2008}.
As shown by \citet{Fenton2008} and \citet{Pourbohloul2003},
the re-supply of susceptible individuals following STI treatment
will fuel an epidemic, and so the influence of turnover on
STI incidence, prevalence, and TPAF may be different, and warrants future study.
% HM: couple of limitations is about the unified system.
%     If we are going to focus on the model with one specific case,
%     we might want to focus more on the limitations in application.
% JK: I'm not sure what you mean exactly?
Fifth, our analyses were restricted to
equilibrium STI prevalence and incidence.
The influence of turnover at different phases of an epidemic
-- growth, mature, declining --
are expected to vary, and thus represents an important topic for future investigation.
Finally, our analyses reflected an illustrative STI epidemic
in a population with illustrative risk strata.
Future work should explore more realistic systems for specific STIs,
such as the work by \citet{Johnson2016}.
% --------------------------------------------------------------------------------------------------
\paragraph{Conclusion}
In conclusion, turnover in risk will influence
epidemic model outputs, including projected incidence, prevalence, and
measures of the contribution of high risk groups to overall STI transmission.
% JK: added "incidence & prevalence" here.
%     I know its not the #1 message of the implications,
%     but still an important part of the results
%     worth mentioning in the conclusion I think.
Turnover should therefore be considered in
transmission models with heterogeneity in risk.
% SS: This seems to undercut the paper a bit – if the data needs are allowed to come off as too daunting.
%     I wonder if before highlighting what the failure would result in, if it’s worth noting that
%     suggestions were provided of how to estimate / parameterize these data points. Perhaps not necessary,
%     but I was thinking again about the contributions you want to highlight in this paper and the methods that this
%     is advancing, and by ending with this statement it seems to downplay the methodological advance a bit by
%     making it sound difficult to implement. Feel free to ignore.
% SB: Yep, this is the balance of the big data grant. Ie, data are weak but not so weak
%     that we cant do things. So while a balanced approach is needed, it is possible!
% JK: Great points! I've revised this to highlight the proposed methods as a solution
%     to the data challenge, and generally sound more positive
The methods presented here illustrate how
epidemiologic data can be used to parametrize
turnover in epidemic models.
We hope that this will support accurate estimation of
the importance of addressing the unmet needs of high risk populations
-- including
gay men and other men who have sex with men,
transgender women,
people who use drugs, and
people of all genders who sell sex
-- to achieve population-level transmission reduction.