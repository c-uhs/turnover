Using a mechanistic modelling analyses, 
we found that turnover could be important 
when estimating the tPAF
of high risk groups to the overall epidemic.
Mechanistic insights include disentangling 
three key phenomena by which turnover 
alters infection prevalence and incidence within risk groups and 
thereby, the prevalence ratios between the highest and lowest risk groups.
Methodological contributions include an unified 
framework for modelling turnover 
using a flexible combination of data-driven constraints.
Thus, the explanatory insights and framework
have mechanistic, public health, and methodological relevance 		%re-ordered to match the order of the paragraph topics below
for the parameterization
and use of epidemic models that project the tPAF of high risk groups.

% --------------------------------------------------------------------------------------------------
\paragraph{Influence of turnover on prevalence}				% I did not see topic headings like this as subheadings in the papers I have reviewed for Scientific Reports or Epidemics. Are you sure they are allowed?

% TODO: treatment in our model ~~ early infection in Zhang2012:
% decreasing infectiousness as time since infection increases
Consistent with previous work, we found that the relationship between
turnover and overall equilibrium STI prevalence has
an inverted U-shaped pattern
\citep{Stigum1994,Zhang2012,Henry2015}.
%SM: how did Stigum, Zhang, Henry explain their U-shaped pattern? --> did they suggest/explain same as the 3 below? if not, then perhaps most accurate to say that.. e.g. "We also identified the following phenomena drove the U-shaped relationship..." or if they found and explained the same thing --> then have to tell the reader that very clearly. from the statement you had before ("However, unlike prior works we also illustrated the influence of turnover on risk group-specific prevalence, and demonstrated mechanistically how this influence occurs ") = its vague but interpretation = they did not provide a mechanistic explanation.

We identified the following phenomena drove the U-shaped relationship:
shifts in proportion susceptible especially in the high risk group (and thus, reduction in herd immunity); 
changes in the characteristics of the sexual network (via reduction in infectious partnerships with greater turnover); 
and influx of individuals in the infectious health-state into the lower risk groups.
An examination of the above three phenomena revealed an important pattern 
of prevalence ratios by rates of turnover.

However, unlike prior works %cite
we also illustrated the influence of turnover on
risk group-specific prevalence, and demonstrated mechanistically how this influence occurs.  %circular logic. 
Under low rates of turnover versus no turnover,
prevalence in all groups tended to increase, due to
reduced herd immunity in the high risk group
via loss of treated individuals.%
% JK: not sure where to put this footnote.
\footnote{
  In comparing these results to \citep{Zhang2012},
  we theorize that the acute and chronic HIV infection states use by \citeauthor{Zhang2012},
  with higher and lower levels of infectiousness, respectively,
  were analogous to our infectious and treated health states.}
Under high rates of turnover versus low rates of turnover,
prevalence in all groups tended to decrease, due to
reduced average number of partners among infectious individuals.
We also showed that the high/low risk prevalence ratio
monotonically decreased with increasing turnover
-- which we describe as the ``homogenizing effect'' of turnover --
since the net movement of infectious individuals with turnover is
from high to low risk.
Indeed, as turnover increased,
an increasing proportion of infections in the low risk group
were acquired by individuals during a previous period of higher risk
(Figure~\ref{fig:new-inf-phi-vs-lambda}).
% \citet{Henry2015} demonstrated that
% a reduction in risk heterogeneity through turnover
% decreases the basic reproductive number and thus,
% means epidemic control could be easier to achieve.
% Our findings on the mechanisms by which increasing turnover
% reduces the STI treatment rate to achieve zero prevalence
% further supports these insights from \citet{Henry2015}.
% JK: Re. above - too bad we didn't include this in the main text anymore,
%     it was a cool result. I will add it in text to the appendix. TODO
% A proportion of infections among the low risk groups
% were acquired by individuals during a previous period of higher risk
% (Figure~\ref{fig:new-inf-phi-vs-lambda}).
% JK: Same re. this too.
% --------------------------------------------------------------------------------------------------
\paragraph{Implications for interventions}
Our comparison of models with and without turnover,
calibrated to the same STI epidemic,
showed that if turnover exists in a given setting
but is ignored in a model,
the tPAF of high risk groups will be underestimated by the model.
% LW: I think this is a valid conclusion
This is because heterogeneity in risk
must be higher in the presence of turnover than in a model without turnover
in order to produce the same STI prevalence across risk groups.
Although we examined a single parameter to capture risk
(number of partners per year),
these insights would be generalizable to and multiplied by
any other component of susceptibility or infectivity,
since the risk per susceptible individual (force of infection) includes both
biological transmission probabilities and number of partners per year \citep{Anderson1991}.
In the context of models with assortative mixing
(individuals are more likely to select partners from their risk group),
the difference between the tPAFs estimated by the model
with versus without turnover is therefore expected to be even larger.
% SM: have not reviewed results again, so make sure this edit is correct :).
%     I also do want to review that part more carefully to see the checks on the mixing matrices :)
%     but will do that when paper is with co-authors and after my Sep 6 many deadlines.
%     Just make sure that what I wrote here is what you showed in results/appendix really clearly.
%     I remember the figure and think yes, but did not review carefully yet. 
% JK: yup - all good! In the appendix I used the usual Garnett / Nold approach
%     with epsilon = 0.5.
% JK: update: well, now this is removed from the appendix,
%     so am just suggesting at the expected trend,
%     though obviously we've explored this and shown that
%     the TPAF gap is even larger under assortative mixing.
The public health implication of models ignoring turnover
which is present in reality is that
the tPAF of high risk groups will be systematically underestimated,
potentially misguiding resources away from high risk groups.
Follow on research should quantify the size of
this potential bias in tPAFs generated from models without turnover,
and characterize the epidemiologic conditions under which the bias would be largest.
% --------------------------------------------------------------------------------------------------
\paragraph{Turnover framework}
% LW: I would keep this at the end of discussion before limitation section.
%     It just flows better with results if we discuss result implications first
%     before moving into method advance.
% JK: done.
% SM: this paragraph needs a lot of revision/work
%     and I think I should definitely review/edit the next version before submission.
%     I will not have time to edit before it goes to co-authors unfortuntely
%     - so please make a note for them to tell you whether they understood the key messages here.
%     Remember that a big part of this paragraph is written for epidemiologists/data folks
%     who are going to give modelers the data to parameterize turnover.
% JK: have given this an overhaul -- hopefully is more precise and clear now.
The proposed framework provides a flexible way to parameterize risk group turnover
based on available epidemiologic data and/or assumptions.
The approach has five advantages.
First, the framework defines how specific epidemiologic data and assumptions
can be used as constraints to help define rates of turnover
(Table~\ref{tab:constraints}).
These data and assumptions are further discussed in the next paragraph.
Second, the framework allows such constraints,
which take the form $b_k = A_k \bm{\theta}$
to be chosen and combined in a flexible way,
depending on which data are available, or which assumptions are most plausible.
While it is necessary that constraints do not conflict one another,
it is not necessary that a complete set of $G^2$ constraints be defined
(where $G$ is the number of risk groups),
since optimization techniques can be used to calculate, for example,
the smallest possible values of $\phi$ which satisfy the given constraints.
Third, this flexible approach also allows the framework to scale
to any number of risk groups, $G$.
Fourth, we can show how several previous implementations of turnover
\citep{Stigum1994,Eaton2014,Henry2015}
can be recreated exactly using the proposed framework
(Appendix~\ref{aa:prev-appr}).
In so doing, we highlight which specific assumptions are
the same and which are different across the different implementations.
Fifth and finally, the framework avoids the need for a burn-in period
to establish a demographic steady-state before introducing infection,
which was required in some previous models \citep{Boily2015}.
\par
The types of data which can be used by this approach include the following four categories
(Table~\ref{tab:constraints}).
First, the proportion of total individuals in each risk group $\hat{x}_i$ is required
for constraints of Type~1.
% SM: the System section would benefit from a table/flow-diagram of the constraints
%     and their epidemiological meaning + data needs.
% JK: have added this now in the latest version of System, and added reference here.
%     I did chose Table specifically though, vs flow-diagram,
%     since I don't think there should be a path / flow / order of constraint selection;
%     it's more so an unordered set of constraints, which just have to be sufficient
%     to yield a unique (exactly one) and consistent (no conflicts) solution \theta.
In the context of STIs, risk group size estimates may be obtained from
demographic health surveys \citep{DHS}, and from
mapping and enumeration of marginalized populations,
such as sex workers \citep{Abdul-Quader2014}.
For example, one risk group may be defined by
any engagement in casual sex within the past year,
and the corresponding proportion of the population
could be estimated from self-reported behaviours
in demographic health surveys \citep{DHS}.
% HM: overall, I think this para could be shorten and simplified. It reads a little bit long
% JK: Just trying to balance this with providing examples
Second, the proportion of individuals who enter into each risk group $\hat{e}_i$
upon entry into the modelled population can be used to define additional constraints of Type~2.
Such proportions could be obtained the same way as $\hat{x}_i$ above,
except using only data from individuals who recently became sexually active.
For example, among women who became sexually active for the first time in the past year,
what proportion also engaged in casual sex within the past year.
If data on sexual debut is not available,
then recent entry into sexual activity could be approximated using
a suitable age range.
Third, the average duration of time spent within each risk group $\delta_i$
can similarly be used to define Type~3 constraints.
% SM: what do you mean by useful? are all 4 of these needed? or just 1? unclear from current paragraph
% JK: That's the thing about the framework (which is tricky to explain):
%     A combination of constraints is needed. Many combinations will work, but not all,
%     and the number of constraints you need (G^2) depends on the number of risk groups (G).
%     Hopefully this is more clear now in the above paragraph and in the revised System section.
Cross-sectional survey questions asked of female sex workers such as
``for how many years have you been a sex worker?''\ %
may be used to obtain estimates of duration in sex work,
with the recognition that such data are censured \citep{Watts2010}.
Longitudinal, or cohort studies
that track the self-reported sexual behaviour over time can also provide 
estimates of duration of time spent in other risk groups \citep{Fergus2007}.
Fourth, similar studies may provide data on
trajectories in risk behaviour of individuals over time,
which can be used to estimates specific transition rates $\phi$ (Type~2 constraints)
or ratios of transition rates (Type~4 constraints).
For example, upon retirement from sex work,
twice as many former sex workers may enter into monogamous relationships,
as compared to those who continue to form multiple partnerships.
% --------------------------------------------------------------------------------------------------
\paragraph{Limitations}
There are six limitations of the study that are important when considering
the proposed turnover framework and the implications of our results.
First, our approach does not account for
infection-attributable mortality, such as HIV-attributable mortality.
It is well-established that HIV-attributable mortality will reduce the 
relative size of higher risk groups,
and that alone can cause an epidemic to decline \citep{Boily1997}.
As such, many models of HIV transmission
that include very small ($<3\%$ of the population)
high risk groups, such as female sex workers,
often do not constrain the relative size
of the sub-group populations to be stable over time
\citep{Pickles2013}.
% LW: Lacking a “however” statement following this limitation to justify in which case such
%     limitation is of less concern
% JK: I do mention below, after introducing both points,
%     that these should be subject of future work.
%     I know its kind of a cop-out but incorportating death in the current model
%     is difficult to explain.
Second, we assumed a single-sex population and did not stratify by age.
In the context of real-world STI epidemics,
the relative size of risk groups may differ
by both sex and age,
such as the number of females who sell sex,
versus the number of males who sell sex,
versus the number of males who pay for sex (clients).
Future work on the proposed framework could incorporate both
infection-attributable mortality, and age-sex stratifications into the model.
Third, we assumed that turnover rates were not affected by
the health status of individuals,
which could change the observed influence of turnover on
equilibrium STI prevalence.
While changes to health state may affect the sexual behaviour of individuals,
its not clear what overall patterns emerge at the population level.
Fourth, we did not include individuals becoming re-susceptible
-- an important feature of many STIs such as syphilis and gonorrhoea
\citep{Fenton2008}.
As shown by \citet{Fenton2008} and \citet{Pourbohloul2003},
the re-supply of susceptible individuals following STI treatment
will fuel an epidemic, and so the influence of turnover on
STI prevalence and tPAF may be different, and warrants future study.
Fifth, our analyses were restricted to equilibrium STI prevalence.
The influence of turnover at different phases of an epidemic
-- growth, mature, declining --
are expected to vary, and thus represents an important topic for future investigation.
Finally, our analyses reflected an illustrative STI epidemic
in a population with illustrative risk strata.
Future work should explore more realistic systems for specific STIs,
such as the work by \citet{Johnson2016}.
% --------------------------------------------------------------------------------------------------
\paragraph{Conclusion}
In conclusion, turnover in risk will influence
epidemic model outputs, including projected prevalence and
measures of the contribution of high risk groups to overall STI transmission.
Turnover should therefore be considered in
transmission models with heterogeneity in risk.
% SS: This seems to undercut the paper a bit – if the data needs are allowed to come off as too daunting.
%     I wonder if before highlighting what the failure would result in, if it’s worth noting that
%     suggestions were provided of how to estimate / parameterize these data points. Perhaps not necessary,
%     but I was thinking again about the contributions you want to highlight in this paper and the methods that this
%     is advancing, and by ending with this statement it seems to downplay the methodological advance a bit by
%     making it sound difficult to implement. Feel free to ignore.
% SB: Yep, this is the balance of the big data grant. Ie, data are weak but not so weak
%     that we cant do things. So while a balanced approach is needed, it is possible!
% JK: Great points! I've revised this to highlight the proposed methods as a solution
%     to the data challenge, and generally sound more positive
The methods presented here illustrate how
epidemiologic data can be used to parametrize
turnover in epidemic models.
We hope that this will support accurate estimation of
the importance of addressing the unmet needs of high risk populations
-- including
gay men and other men who have sex with men,
transgender women,
people who use drugs, and
people of all genders who sell sex
-- to achieve population-level transmission reduction.