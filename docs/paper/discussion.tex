Using a mechanistic modelling analyses, 
we found that turnover could be important 
when estimating the tPAF
of high risk groups to the overall epidemic.
Mechanistic insights include disentangling 
three key phenomena by which turnover 
alters infection prevalence and incidence within risk groups and 
thereby, the level of inferred risk heterogeneity between groups.
Methodological contributions include an unified 
framework for modelling turnover 
using a flexible combination of data-driven constraints.
Taken together, the explanatory insights and framework
have mechanistic, public health, and methodological relevance 		%SM: re-ordered to match the order of the paragraph topics below
for the parameterization
and use of epidemic models to project the tPAF of high risk groups.

% --------------------------------------------------------------------------------------------------
\paragraph{Influence of turnover on prevalence}				% SM: did not see topic headings like this as subheadings in the papers I have reviewed for Scientific Reports or Epidemics. we sure they are allowed?

% TODO: treatment in our model ~~ early infection in Zhang2012:  %SM: ??
% decreasing infectiousness as time since infection increases         %SM: ??
Building on prior work by \citep{Stigum1994,Zhang2012,Henry2015}
which similarly found an inverted 
U-shaped relationship between
turnover and overall equilibrium STI prevalence, 
we identified three key phenomena that generated the relationship.
The turnover-driven phenomena include: 
shifts in proportion susceptible especially in the high risk group (and thus, reduction in herd immunity); 
changes in the characteristics of the sexual network (via reduction in infectious partnerships with greater turnover); 
and influx of individuals in the infectious health-state into the lower risk groups.
An examination of the above three phenomena revealed an important pattern 
of declining prevalence ratios between the highest and lowest risk groups at increasing rates of turnover. 
The implication of declining prevalence ratios with faster turnover could 
be described as a ``homogenizing effect'' of turnover because the 
net movement of infectious individuals was
from high to low risk - which, when taking place at faster and faster rates 
means individuals spend
very little time in any given risk group.			%SM: correct?
This `homogenizing effect'' is consistent with work by 
\citet{Henry2015} which showed that
 a reduction in risk heterogeneity through turnover
decreases the basic reproductive number.
Indeed, epidemiological and transmission modelling studies have shown that prevalence ratios 
are an important marker of risk heterogeneity and in turn, 
the impact of interventions focused on high risk groups. %cite: PMID: 22424777 (epi studies); PMID: 23226357 (review modeling studies)


% --------------------------------------------------------------------------------------------------
\paragraph{Implications for interventions}
Our comparison of fitted models with and without turnover
showed that if turnover exists in a given setting
but is ignored in a model, its projections could 
underestimate 
the tPAF of high risk groups.		
The tPAF was underestimated because heterogeneity in risk
had to be higher in the presence of turnover
to reproduce the same group-specific STI prevalence as the model without turnover.
Although we examined a single parameter to capture risk
(number of partners per year),
the findings would hold for any combination of factors
that alter the risk per susceptible individual (force of infection)  
which includes
biological transmission probabilities and rates of partner change \citep{Anderson1991}.
That is, the level of inferred risk heterogeneity may vary depending on whether
the epidemic model includes or excludes turnover.
The public health implication of models ignoring turnover
which is present in reality is that
the tPAF of high risk groups could be systematically underestimated.
Underestimating the tPAF of high risk groups - and underestimating
risk heterogeneity itself - could potentially misguide resources away from high risk groups.
Indeed, underestimating risk heterogeneity could underestimate the resources 
required to achieve local epidemic control, as suggested by \citet{Henry2015}. %SM: add  PMID: 24167449    
Important next steps surrounding the potential bias in tPAF projections attributable to inclusion/exclusion
of turnover include quantifying its magnitude, 
and characterizing the epidemiologic conditions under which the bias
would be meaningfully large in the context of public health programmes.

% --------------------------------------------------------------------------------------------------
\paragraph{Turnover framework}
We developed a unified framework 
to parameterize risk group turnover
using available epidemiologic data and/or assumptions.
There are five potential benefits of using the framework to model turnover.
First, the framework defines how specific epidemiologic data and assumptions
could be used as constraints to help define rates of turnover.
%(Table~\ref{tab:constraints}). %please do not refer to appendix, tables, or figures in the discussion (see journal instructions and discussion write-up guidance docs?).
Second, the framework allows flexibility in how and which
constraints (taking the form $b_k = A_k \bm{\theta}$)
can be chosen and combined, so that constraints best reflect
locally available data and/or plausible assumptions.
While it is necessary that constraints do not conflict one another,
it is not necessary that a complete set of $G^2$ constraints be defined
(where $G$ is the number of risk groups),
since optimization techniques can be used to calculate, for example,
the smallest possible values of $\phi$ which satisfy the given constraints. %SM: hard time following the sentence starting with "While...". suggst edit for clarity / simplify language - already mentionted the technical in appendix, so try to keep discussion as free of jargon as possible.  no need to fully repeat technical deets.
Third, this flexible approach also allows the framework to scale
to any number of risk groups, $G$.
Fourth, the framework can adequately reproduce 
several prior implementation of 
turnover in various epidemic models \citep{Stigum1994,Eaton2014,Henry2015}, 	%didn't Eaton paper have HIV-attributable mortality? Does this not then contradict what is said in limitations re: infection-attributable mortality? %please do not refer to appendix, tables, or figures in the discussion (see journal instructions and discussion write-up guidance docs?).
which means.....??? %SM: discussion section should put things into context or dive into the 'so what' not just repeat what is in results (or appendix). 

%SM: I did not follow the following sentence in relation to the sentence above. not sure i get the 'so what'?
%In so doing, we highlight which specific assumptions are
%the same and which are different across the different implementations.

Finally, the framework avoids the need for a burn-in period
to establish a demographic steady-state before introducing infection,
which was required in some previous models \citep{Boily2015}.

\par
%SM: prior to editing, the framework component took up >50% of the discussion real estate and so I thought best to bring it back to the main research question, and simplify the data paragraph. The details (type 1, type 2, etc.) are available in the appendix if someone wants to use the framework as is and hopefully everything is clear in there and the discussion paragraph reflects an example 'source' context for each key data element.
As noted above, one benefit of the unified framework for modelling turnover	
is clarifying data priorities for parameterizing turnover. 
Absolute or relative population size estimates across
risk groups may be obtained from
population-based sexual behaviour surveys \citep{DHS}, and from
mapping and enumeration of marginalized persons
such as sex workers \citep{Abdul-Quader2014}.
The proportion of individuals who enter into each risk group
may be available from sexual behaviour surveys:
for example, among individuals who became sexually active for the first time in the past year,
the proportion also engaged in multiple partnerships within the past year.
The average duration of time spent within each risk group, such as
the duration in sex work,
may be drawn from
cross-sectional survey questions asked of sex workers such as
``for how many years have you been a sex worker?''\ %
albeit with the recognition that such data are censured \citep{Watts2010}.
Longitudinal, or cohort studies
that track the self-reported sexual behaviour over time can also provide 
estimates of duration of time spent within a given risk strata \citep{Fergus2007}, 
or provide direct transition rates between risk strata.
%SM: p.s. try to overcome the instinct of saying 'difficult/tricky to explain'. we have to find a way to explain it - even if it takes more time and lots of feedback from others to get there. communicating the science with words / diagrams is really important if we ever want science to cross disciplines :) and even if we don't.... its key to the greater good of science :). 

% --------------------------------------------------------------------------------------------------
\paragraph{Limitations}
Despite its methodological contribution, our unified 
framework for modelling turnover was developed specifically 
to answer the mechanistic questions surrounding the tPAF. 
As such, there are two key limitations of the framework in its current form.
First, the framework does not account for						%SM: by "approach" - do we mean the framework or the STI model? please clarify as pertains to both noh?
infection-attributable mortality, such as HIV-attributable mortality.
Modelling studies have shown that HIV-attributable mortality can reduce the 
relative size of higher risk groups who bear a disproportionate burden of HIV, which in turn 
can cause an HIV epidemic to decline \citep{Boily1997}.
As such, many models of HIV transmission
that include very small ($<3\%$ of the population)
high risk groups, such as female sex workers,
often do not constrain the relative size
of the sub-group populations to be stable over time
\citep{Pickles2013}. Thus, the unified framework would require 
additional modifications before it can be used within transmission models of 
pathogens with infection-attributable mortality.									
%SM: so what does this mean for folks who want to use the unified framework? we need to be explicit and say our framework **should not** be used for any pathogen where disease-attributable mortality, until it is later modified...?


% LW: Lacking a “however” statement following this limitation to justify in which case such
%     limitation is of less concern
% JK: I do mention below, after introducing both points,
%     that these should be subject of future work.
%     I know its kind of a cop-out but incorportating death in the current model
%     is difficult to explain. % SM: but still have to say 'so what to do' though---> right now, its not clear to the reader whether they need to modify further if they want to include disease-attributable mortality or not. We don't have to say how to incporate death, but rather --> we have to go one step beyond what we have now.

Second, our unified framework assumed a single-sex population and did not stratify by age.		%SM: are we referring to the framework or the STI model re: the tPAF analysis?
In the context of real-world STI epidemics,
the relative size of risk groups may differ
by both sex and age,
such as the often smaller number of females and/or males who sell sex,
versus the larger number of males who pay for sex (clients of female or male sex workers).
Future modifications of the proposed framework include capturing
infection-attributable mortality and relevant age-sex stratifications so that
the framework can be applied more broadly to pathogen-specific epidemics.

\paragraph{Limitations}		%SM: in reading the limitations, felt a bit confused at times re: whether talking about the transmission model or the framework, so suggest separating out the two into shorter paragraphs.

Our turnover and tPAF mechanistic analyses also has several limitations.
First, we did not capture the possibility that some individuals may become
re-susceptible to infection after treatment
-- an important feature of many STIs such as syphilis and gonorrhoea
\citep{Fenton2008}.
As shown by \citet{Fenton2008} and \citet{Pourbohloul2003},
the re-supply of susceptible individuals following STI treatment
could fuel an epidemic, and so the influence of turnover on
STI prevalence and tPAF may be different.
Second, our analyses were restricted to equilibrium STI prevalence.
The influence of turnover on tPAF at different phases of an epidemic
-- growth, mature, declining --									%cite terminology of the phases of an epidemic: PMID: 8843250
may vary. Finally, our analyses reflected an illustrative STI epidemic
in a population with illustrative risk strata.
Important next steps in the examination of the extent to which turnover influences the tPAF include
pathogen- and population-specific modeling (such as the comparisons of model structures 
by \citet{Johnson2016} and % Hontalez PMID: 24167449    			
and at different epidemic phases.

%SM: suggest leaving out assortative mixing in limitations for now, and if reviewer asks --> then put it back in. It just felt like too much (i.e. information overload) when I read the discussion out loud a few times. p.s. for later -- but reading sections out before sending to editors is a great practice! 

% --------------------------------------------------------------------------------------------------
\paragraph{Conclusion}
In conclusion, turnover in risk will influence
epidemic model outputs, including projected prevalence and
measures of the contribution of high risk groups to overall STI transmission.
Turnover should therefore be considered in
transmission models with heterogeneity in risk.
% SS: This seems to undercut the paper a bit – if the data needs are allowed to come off as too daunting.
%     I wonder if before highlighting what the failure would result in, if it’s worth noting that
%     suggestions were provided of how to estimate / parameterize these data points. Perhaps not necessary,
%     but I was thinking again about the contributions you want to highlight in this paper and the methods that this
%     is advancing, and by ending with this statement it seems to downplay the methodological advance a bit by
%     making it sound difficult to implement. Feel free to ignore.
% SB: Yep, this is the balance of the big data grant. Ie, data are weak but not so weak
%     that we cant do things. So while a balanced approach is needed, it is possible!
% JK: Great points! I've revised this to highlight the proposed methods as a solution
%     to the data challenge, and generally sound more positive
The methods presented here illustrate how
epidemiologic data can be used to parametrize
turnover in epidemic models.
We hope that this will support accurate estimation of
the importance of addressing the unmet needs of high risk populations
-- including
gay men and other men who have sex with men,
transgender women,
people who use drugs, and
people of all genders who sell sex
-- to achieve population-level transmission reduction.