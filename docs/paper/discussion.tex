Using a mechanistic modelling analysis,
we found that turnover could be important when
projecting the tPAF of high risk groups to the overall epidemic.
Mechanistic insights include disentangling
three key phenomena by which turnover
alters equilibrium STI prevalence within risk groups,
and thereby the level of inferred risk heterogeneity between groups via model fitting.
Methodological contributions include a  framework for modelling turnover
which uses a flexible combination of data-driven constraints.
Taken together, our explanatory insights and framework
have mechanistic, public health, and methodological relevance for
the parameterization and use of epidemic models
to project intervention priorities for high~risk~groups. % TEMP spacing
% --------------------------------------------------------------------------------------------------
\paragraph{Influence of turnover on prevalence}
Building on prior work by \citet{Stigum1994,Zhang2012,Henry2015}
which similarly found an inverted U-shaped relationship between
turnover and overall equilibrium STI prevalence,
we identified three key phenomena that generated this relationship.
These turnover-driven phenomena were:
1)~a net flow of infectious individuals from higher risk groups to lower risk groups;
2)~reduced herd effect in the higher risk groups due to net gain of susceptible individuals; and
3)~reduced incidence overall due to fewer partner numbers among infectious individuals.
The above three phenomena contributed to the pattern
of declining prevalence ratio between the highest and lowest risk groups
for increasing rates of turnover.
A decline in prevalence ratio due to turnover implies a reduction in risk heterogeneity.
Since risk heterogeneity is associated with epidemic emergence and persistence \citep{May1988}
-- i.e. the basic reproductive number --
our findings are thus consistent with \citet{Henry2015},
who demonstrated that turnover reduces the basic reproductive number by reducing heterogeneity.
Indeed, epidemiological and transmission modelling studies have shown that prevalence ratios
are an important marker of risk heterogeneity, and in turn
the impact of interventions focused on high risk groups \citep{Baral2012,Mishra2012}.
% --------------------------------------------------------------------------------------------------
\paragraph{Implications for interventions}
Our comparison of fitted models with and without turnover showed that
if turnover exists in a given setting but is ignored in a model,
the inferred heterogeneity in risk would be lower than in reality,
while reproducing the same STI prevalence in each risk group.
As a result, the projected tPAF of high risk groups
could be systematically underestimated by models that ignore turnover.
Although we examined a single parameter to capture risk
(number of partners per year),
the findings would hold for any combination of factors
that alter the risk per susceptible individual (force of infection), including
biological transmission probabilities and rates of partner change \citep{Anderson1991}.
The public health implications of models ignoring turnover,
and thereby underestimating risk heterogeneity and the tPAF of high risk groups,
is that resources could potentially be misguided away from high risk groups,
For example, epidemic models which fail to include or accurately capture
turnover may underestimate the importance of addressing the unmet
needs of key populations at disproportionate risk of HIV and other STIs, such as
gay men and other men who have sex with men, transgender women, people who use drugs, and sex workers.
Such needs may be addressed by interventions tailored to
the unique vulnerabilities experienced by key populations, interventions such as
enhanced and prioritized screening and testing, condom use programmes, and reductions in barriers to safer sex.
In many HIV epidemic models of regions with high HIV prevalence, such as in Southern Africa,
key populations have historically been subsumed into the overall modelled population;
which meant, by design, less risk heterogeneity \citep{Eaton2012,Cori2014,Mishra2016}.
Our findings suggest that even when key populations are included, it is important to
further capture within-person changes in risks over time (such as duration in sex work).
Underestimating risk heterogeneity could also underestimate the resources
required to achieve local epidemic control, as suggested by \citet{Henry2015,Hontelez2013}.
Important next steps surrounding the potential bias in tPAF projections
attributable to inclusion/exclusion of turnover include
quantifying the magnitude of bias,
and characterizing the epidemiologic conditions under which the bias
would be meaningfully large in the context of public health programmes.
% --------------------------------------------------------------------------------------------------
\paragraph{Turnover framework}
We developed a unified framework
to parameterize risk group turnover
using available epidemiologic data and/or assumptions.
There are four potential benefits of using the framework to model turnover.
First, the framework defines how specific epidemiologic data and assumptions
could be used as constraints to help define rates of turnover.
Second, the framework allows flexibility in which constraints can be chosen and combined,
so that the constraints best reflect locally available data and/or plausible assumptions.
In fact, the framework can adequately reproduce
several prior implementations of turnover
in various epidemic models \citep{Stigum1994,Eaton2014,Henry2015}.
Third, this flexible approach also allows the framework to scale
to any number of risk groups.
Finally, the framework avoids the need for a burn-in period
to establish a demographic steady-state before introducing infection,
which was required in some previous models \citep{Boily2015}.
\par
As noted above, one benefit of the unified framework for modelling turnover
is clarifying data priorities for parameterizing turnover.
Absolute or relative population size estimates across risk groups
may be obtained from population-based sexual behaviour surveys \citep{DHS},
and from mapping and enumeration of marginalized persons
such as sex workers \citep{Abdul-Quader2014}.
The proportion of individuals who enter into each risk group
may be available through sexual behaviour surveys:
for example, among individuals who became sexually active for the first time in the past year,
the proportion who also engaged in multiple partnerships within the past year.
The average duration of time spent within each risk group, such as
the duration in sex work, may be drawn from
cross-sectional survey questions such as
``for how many years have you been a sex worker?''\ %
albeit with the recognition that such data are censured \citep{Watts2010}.
Longitudinal, or cohort studies that track
self-reported sexual behaviour over time can also provide
estimates of duration of time spent within a given risk strata \citep{Fergus2007},
or provide direct estimates of transition rates between risk strata.
% --------------------------------------------------------------------------------------------------
\paragraph{Limitations}
Our framework for modelling turnover was developed
specifically to answer mechanistic questions about the tPAF;
as such, there are two key limitations of the framework in its current form.
First, the framework did not stratify the population by sex or age.
In the context of real-world STI epidemics,
the relative size of risk groups may differ by both sex and age,
such as the often smaller number of females and/or males who sell sex,
versus the larger number of males who pay for sex (clients of female or male sex workers).
Second, the framework does not account for
infection-attributable mortality, such as HIV-attributable mortality.
However, modelling studies have shown that HIV-attributable mortality can reduce the
relative size of higher risk groups who bear a disproportionate burden of HIV,
which in turn can cause an HIV epidemic to decline \citep{Boily1997}.
As such, many models of HIV transmission that include
very small ($<3\%$ of the population) high risk groups, such as female sex workers,
often do not constrain the relative size
of the sub-group populations to be stable over time \citep{Pickles2013}.
By ignoring infection-attributable mortality,
the proposed framework would similarly allow risk groups to change relative size
in response to disproportionate infection-attributable mortality.
Future modifications of the proposed framework include methods to optionally re-balance
infection-attributable mortality, and relevant age-sex stratifications so that
the framework can be applied more broadly to pathogen-specific epidemics.
\par
Our analyses of turnover and tPAF also have several limitations.
First, we did not capture the possibility that some individuals may become
re-susceptible to infection after treatment
-- an important feature of many STIs such as syphilis and gonorrhoea \citep{Fenton2008}.
As shown by \citet{Fenton2008} and \citet{Pourbohloul2003},
the re-supply of susceptible individuals following STI treatment
could fuel an epidemic, and so the influence of turnover on
STI prevalence and tPAF may be different.
In fact, the herd effect underpinning phenomenon~2
relies on the existence of a treated/immune group,
since only net replacement of treated (versus infectious) individuals
with susceptible individuals via turnover can increase prevalence
in the high risk group as observed.
Second, our analyses were restricted to equilibrium STI prevalence.
The influence of turnover on prevalence and tPAF
may vary within different phases of an epidemic
-- growth, mature, declining \citep{Wasserheit1996}.
Finally, our analyses reflected an illustrative STI epidemic
in a population with illustrative risk strata.
Important next steps in the examination of
the extent to which turnover influences the tPAF include
pathogen- and population-specific modelling
-- such as the comparisons of model structures by \citet{Hontelez2013,Johnson2016} --
and at different epidemic phases.
% --------------------------------------------------------------------------------------------------
\paragraph{Conclusion}
In conclusion, turnover may influence prevalence of infection, and
thus influence inference on risk heterogeneity
when fitting risk-stratified epidemic models.
If models do not capture turnover,
the projected contribution of high risk groups, and thus,
the potential impact of prioritizing interventions to meet their needs, could be underestimated.
To aid the next generation of epidemic models
used to estimate the tPAF of high risk groups -- including key populations --
data collection efforts to parameterize risk group turnover should be prioritized.
