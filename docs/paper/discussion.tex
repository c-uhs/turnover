%Perhaps this is because it is seen as either infeasible or inconsequential to do so.
%In this work, we have hopefully shown that it is neither.
%\par
We developed a unified framework for modelling turnover in risk and
used the framework to explore the influence of turnover on 
the TPAF of the highest risk group to onward transmission in an
illustrative STI epidemic without STI-attributable mortality. 
We found that XXX % in original research papers, usually the summary opening paragraph should succintly (1-2 line at most) say what we found, not what we did. Otherwise reads as repetitive.



% --------------------------------------------------------------------------------------------------
\paragraph %{Turnover framework}	%this paragraph needs a lot of revision/work and I think I should definitely review/edit the next version before submission. I will not have time to edit before it goes to co-authors unfortuntely - so please make a note for them to tell you whether they understood the key messages here. Remember that a big part of this paragraph is written for epidemiologists/data folks who are going to give modelers the data to parameterize turnover.

%SM added --> can the framework be used for other bugs? not just STIs? can folks expand on the framework?
%would make that clear. Something like this but give example if possible (or reason re: non-STIs - i.e. how proportional contacts are similar to mass action but with heterogeneity and cite, etc.): we developed a unified framework for modeling turnover in risk that 
%may be used and expanded upon, to capture data-driven assumptions about various risk-strata
%in epidemic models of STIs, and indeed - other infectious diseases. 

Our framework %try to avoid writing 'this', 'these', etc. = generally considered less clear writing
provided a flexible way to establish constraints based on the available data and
key assumptions about the population size and duration of time spent in each risk group. 
The approach has XX advantages.	%if listing <5, give number
First, the formulation subsumed several of the previous implementations of turnover [REFS],		%formulation of framework. consistency check with terminology; also - again, please be careful how you present the work - we did not check that it can capture all implementations, since we did not review the full iiterature. we should put more humility in scientiific writing for papers please :) [reserve our "sell" for grants!].  Science is by definition cautious and self-critical so we should be careful not to over-sell findings in papers.
and thereby hopefully clarifies relationships among them.  % I do not understand this sentence. clarifies relationship among the various implementations? I don't think this is the first point you should start with. Order them the way they are presented in the system. First is parameterization and constraints. Last should be that it subsumes some of the prior implementations --> but why this is an advantage is not clear from the sentence.
Second, it facilitates parameterization of turnover among any number of risk groups
in a scalable, flexible way.									%how? this is really important and should be explained why you think (a) it is flexible; and (b) why it can be scalable. We are leaving too much for the reader to read between the lines. Be more precise and concrete.
Finally, the framework makes the parameterization of turnover systems more transparent,
through explicit incorporation of constraints as rows in the linear system.	% Give an example of what we mean here. What is an implicit way and what is an explicit way.

\par % what do you mean by epidemic context? and how does epidemic context change the constraints? not clear
The major data needs of this approach include the following four categories.			% i don't know what 'implicated by' means. List the data needs as categories and make sure same order/alignment as described in the System section.
First, the proportion of total individuals in each risk group $\bm{\hat{x}}$ is required.	%the System section would benefit from a table/flow-diagram of the constraints and their epidemiological meaning + data needs.
In the context of STIs, risk group size estimates may be 
obtained from representative surveys of the modelled population
such as demographic health surveys [ref] which collect data on self-reported behaviours or engagement in % Demographic Health Survey https://www.dhsprogram.com/
networks associated with variable risk. For marginalized populations, estimates of population size
are generated using various mapping and enumeration methods. %PMID: 24393694
%  try to give examples and citations as the edits above --> remember how tough it was to do this for the 'medium' risk for the ESwatini work? Not so simple so be careful with choice of verbs. Also that is not actually how we get size estimates for KP - we do mapping then surveys. we do not obtain the size from surveys.
Second, the risk behaviours of individuals upon entering the model		%if listing first, etc. - use second, et.c.
can be used to specify elements in $\bm{\hat{e}}$.
Such behaviours could be approximated using data from the youngest age group
in the above surveys,										%how? not sure this is clear - which data from the youngest age group? 
or using surveys among young people specifically.					%how? try to be more precise/concrete here...which data exactly? 
Third, it is useful to know the average duration of time spent within each risk group $\bm{\delta}$.	%"duration of individuals" does not sit well grammaically. what do you mean by useful? are all 4 of these needed? or just 1? unclear from current paragraph
Cross-sectional survey questions asked of female sex workers such as	
``for how many years have you been a sex worker?''			
may be used to obtain estimates of duration in sex work \citep{Watts2010}
with the recognition that the data are censured. Longitudinal, or cohort studies
that track the self-reported sexual behaviour over time can also provide 
estimates of duration within variable periods of risk.					%PMID: 31371449 (not the best citation for this b/c also looks at an intervention, but something like this)
														%avoid the term general population unless really clearly defined (was not previously defined i think).												
Finally, similar data, if available, can also be used to estimate
the rates of transition between specific risk groups ($\phi$).			%very vague. what kind of data? Cannot understand/follow this sentence. Also - is this "Fourth"?
% --------------------------------------------------------------------------------------------------
\paragraph %{Influence of turnover on incidence \& prevalence}			%check journal style if subheadings are common/used in Discussion section. usually not in most journals in the health-field, but might be ok for Epidemics & ID Modeling. I personally would remove the subheadings.
We found that, as expected and shown in XXX[refs],			% do not repeat the findings (already described in lots of detail in the results) but keep Discussion succint with 'so what' and 'how does it compare to other literature'. In modeling and other experimental papers, the 'why' is already explained in the results. In observational epidemiology, unless we are actually trying to make causal inference, the 'why' can only be postulated/hypothesized (not shown in results) and thus goes into the Discussion section.
turnover will change equilibrium STI incidence and prevalence 
Our findings are not new and indeed, are analogous to heat transfer mechanics [REF].				%please cite (even the foundational textbook, but ideally a paper that explains the mechanisms and terms conduction/convection. A note that PhD examiners are particularly picky about citations (i have seen students get major corrections for under-citing in the last few exams I have chaired). 
For example, incidence can be thought of as infection ``conduction'',		%if putting anything in quotation marks in Intro or Discussion, must be cited, unless you are using that term specifically for the study (e.g. turnover). Is there a reason for quotation marks? What other meaning could those terms in quotation marks have?
whereas turnover would represent infection ``convection'',
and both processes represent important mechanisms of infection spread.	%this is good, but I am worried that if people do not know heat transfer or fluid mechanics, then the 'so what' will go missing. I tried to edit to convey the more 'cross-disciplinary' nature of your work here by trying to frame it within the wider literature.

%As such, it would be important to simulate turnover in epidemic models	
%if such dynamics are known or assumed to be present in reality.		% this is not a clear conclusion that can be drawn from the findings discussed in the paragraph. What is defined as 'important'? Also - a more careful version of a sentence like this is better suited for after you discuss the TPAF stuff.
We also found that increasing turnover - by rapidly shifting individuals from one
risk group to another - essentially 'homogenizes'					%use quotation marks here b/c this term has many meanings, and you use it in a new way here. Perhaps as the whole group of co-authors but I wonder about using 'reduce heterogenetiy' instead of 'homogenizes' after this first time use. "homogenizes' has some negative connotations and I worry about folks working in LGBTQ and KP fields reading this and taking it out of context because the paper is so methodological that the term gets lost in the weeds. I would suggest that 'reduces/less heterogeneity' is equivalent here and less worrisome.
the variability in risk between individuals.							%will be glad to never read an instructive 'recall' from you again in an original research paper :) - only in your future lecture notes and if you write a seminar teaching paper or textbook! 
The result would be less heterogeneity as shown by							%did we show this in the result (less heterogeneity)? If not, then edit to reflect that we did not do this ourselvses..
 \citet{Henry2015} who demonstrated that 
such 'homogenization' of risk through turnover								%would change to 'reduced heterogeneity' - but could see what others think & leave a comment asking them to consider the alternative
decreases the basic reproductive number and thus, means an 
epidemic control could be easier to achieve. 
Our findings on the mechanisms by which increasing turnover			%we cannot say we corroborated their hypothesis if we did not say in your objectives will do that.
reduces the STI treatment rate to achieve zero prevalence 
furthers support the insights from \citet{Henry2015}.

% --------------------------------------------------------------------------------------------------
\paragraph% {Implications for interventions}						%separate out the implications as you did in the ISSTDR talk. clearly and with a concrete explanation of the 'so what'. mirror the Introduction. These two TPAF paragraps are the most important of the Discussion because that is how we framed the introduction and objectives. Even though the approach is a major contribution, we chose not to make that the only focus of the paper (lessons learned for me to consider splitting them up I think if we have future papers like this :)> 
We found that the TPAF of the highest-risk group					%check consistency: highest-risk vs. high-risk
would be lower in settings with turnover. This is important
to consider when comparing whom to prioritize interventions in 
different regions with 
different epidemiologic features such as STI prevalence. That is, a setting with		%explain the 'so what' and 'what does this mean for interventions' of your findings. don't just repeat what was in the results. 
turnover may require less of a focused approach to those at highest-risk.
Such an implication may be counterintuitive as a shorter period of higher risk among a 
relatively few means a smaller
person-years of intervention required - and indeed, goes against the traditional
concept of core-groups in STI epidemiology [REFS]. Core-groups are the relatively few
who are most at risk of STI acquistion and transmission, but in this case, and as discussed in the
paragraph above - turnover reduced heterogeneity. That is, settings without turnover had
a higher ratio of STI prevalence between the highest-risk and lowest-risk.
Public health implications of the above finding are two-fold. 
First, if epidemiologists and modelers were to generate
a geographic map of TPAF across different settings, some with and some 
without turnover or various levels of turnover, then turnover could be an 
important source of variability in the projected TPAF between settings. 
Second, if decision-makers
were allocating a fixed ceiling of STI prevention or treatment resources for a particular high-risk group (e.g. sex workers)
across settings based on the TPAF of sex work, 
then settings with 
zero or lower rates of turnover may benefit the most from STI interventions prioritized to sex workers.
This latter implication however, is a general
insight and does not account for transmission between sites; how costs of STI interventions vary over the sexual life-course of individuals;
and other practical and epidemiological considerations including sources of variability in the TPAF not explored in our study.

\par
In contrast to a comparison of settings with and without turnover, 
our comparison of two models calibrated to the same epidemic 
showed that ignoring turnover (if it exists in a given setting) 
will underestimate the TPAF. This is because heterogeneity in risk 
must be higher in the presence of turnover to produce the same epidemic features 
as a model without turnover. Although we examined a single parameter to capture
risk (frequency of partner change), %i can't remember which parameter :)
the insights would be generalizable to any other component of susceptibility or infectivity per contact 
because the risk per susceptible individual (or force of infection term)l is comprised of
the biological transmission probabilities and frequency of partner change. % cite Garnett paper or Anderson and May re: force of infection equations
in the context of models with assortative mixing (those at highest-risk are more likely to form partnerships
with those at highest risk), the difference between the TPAF in a model 
with vs. without turnover was further exacerbated. 					%have not reviewed results again, so make sure this edit is correct :). I also do want to review that part more carefully to see the checks on the mixing matrices :) but will do that when paper is with co-authors and after my Sep 6 many deadlines. Just make sure that what I wrote here is what you showed in results/appendix really clearly. I remember the figure and think yes, but did not review carefully yet. 
The public health implication of models ignoring turnover 
that is present is that we will systematically underestimate the TPAF 
of the highest-risk group
and potentially misguide resources away from the highest-risk group. 
Next steps would include quantifying the size of this potential bias
in TPAFs generated from models without turnover, and the 
epidemiologic conditions under which inclusion or exclusion of 
turnover have the largest influence on the size of the bias.

\paragraph
In both cases, the influence of turnover on TPAF signals the importance of 
collecting turnover-related data to parameterize models that are used to estimate the TPAF of 
high-risk and other groups. Data such as duration of time spent in a risk group, 
or rates of entry/exit from sex work, while difficult to collect - may be possible with
routinely collected longitudinal program data and novel methods of mapping and enumerating key populations using
capture re-capture methods over time[REF].							%will find a citation for you as Blanchard/CGPH did something like this for India (but i think unpublished to date)
Other approaches including estimating the proportion of a population formerly
engaged in a particular risk group, for example females formerly engaged in sex work [REF].	%Transitions study questionnaire has this - Cheuk et al. IJE submitted. I think Elizabeth also pulled this data for her paper.
%since will have listed data sources above, so no need to repeat, but I would highlight again here to make this point really clear and with a few examples like these edits.

% --------------------------------------------------------------------------------------------------
\paragraph %{Limitations}
There are five limitations of the study that are important when considering
the implications of our proposed unified approach to turnover, and the implications
of turnover on TPAF. First, our approach does not account for constraints in the 
presence of STI-attributable mortality, such as HIV-attributable mortality. 
It is well-established that HIV-attributable mortality will reduce the 
relative size of the highest-risk population, and that alone, can cause		%cite. eg. Boily 1997 paper
an epidemic to decline. As such, most models of HIV transmission, in particular
those that include very small (<3 percent of the population) 										
subsets of the population such as female sex workers, often do not 
constrain the relative size of the sex worker population to be stable over time.	%cite HIV modeling papers iwth sex work like Pickles Lancet GH Avahan modeling, etc.
Second, we assumed a single sex population and in the context of real-world
STI epidemics, the relative size of those engaged in sex work may between females who sell sex
and males who sell sex, and males who pay for sex (clients of sex workers). 
Additional work on the unified approach could support these two important limitations to the
generalizability of the approach across STI models.
Third, we did not include individuals becoming re-susceptible - an important feature of many STIs such
as syphilis and gonorrhea. % cite a syphilis and/or gonorhrea modeling paper
As shown by %Sally Blower and Fenton (did you see that paper? - hope I sent it to you) and Pourbohlol syphilis Vancouver rebound paper
the re-supply of susceptible individuals following STI treatment 
will fuel an epidemic, and so the patterns of STI incidence/prevalence 
and the effect of turnover on TPAF may be different and warrants future study.			%please do not try to do this in the next few days :). 
Fourth, our analyses were restricted to 
equilibrium STI prevalence and incidence. The influence of turnover 
at different phases of an STI epidemic - growth, mature, declining - 
are expected to vary and also comprise an important next step.
Fifth, our analyses reflected an illustrative STI and population risk strata, and future work
will require detailing a more realistic system for various STIs (including, but not limited to 
individuals who become re-susceptible) and building on the work by Stigum on 			%something like this as Stigum examined specific biology of STIs
comparing the influence of turnover on TPAF across various STIs.


\paragraph %1-3 sentence conclusion usually for most papers/journals. see Josh's paper perhaps as an example? Guidance docs may also be helpful here.

In conclusion, turnover in risk will influence 
measures of the contribution of high-risk groups to overall STI transmission. 
Turnover should be 
considered in transmission models with heterogeneity in risk. Although the 
data-needs remain potentially challenging, a failure to meet this challenge, 
could lead to misguided information on the importance of addressing the 
unmet needs of high-risk populations - such as those engaged in sex work - 
to achieve population-level transmission benefits.
%conclusion statement = can be a bit soap-boxy and editorial here. Not too much but better than in other places :). Stef is particularly good at conclusion statements!


%------------------- deleted below but left some notes as comments

%Turnover likely also has an important influence on
%temporal dynamics of infection transmission.
%For example, \citet{Henry2015} derived
%the analytical relationship between turnover
%and the basic reproduction number $R_0$,
%and showed that $R_0$ decreases monotonically with turnover.		% I cannot follow the logic of how Henry2015 citations supports this sentence re: different phase of the epidemic.
%These limitations and gaps in our analysis thus represent
%interesting opportunities for future work.

%However, following similar analyses in		
%Appendices~\ref{aaa:what-if-mort}~and~\ref{aaa:what-if-sirs}, respectively,			%I believe we discussed **not** including in the appendix at last PM. No right/wrong answer here but I remain convinced of that decision at this time unless you want to discuss further if you feel strongly that you want to include? You can use in the PhD, but I would strongly recommend removing from the results and appendix. If a reviewer asks, we can do - but the paper is starting to get diluted with too many 'and we did just a bit more' stuff that is showing up in the discussion and taking away from the messages and starting to read like a thesis chapter and not a tight paper. You did really great work Jesse - but can show / use later. Try to focus on a single story for a paper :)  - especially given the journals we are submitting to. Thesis chapters have a wider belt. 
%As an aside - I also want to go over and double-check the math and code re: the STI-attributable part much more carefully before you put into a paper...
%we found that inclusion of disease-attributable mortality or reinfection would
%also result in greater underestimation of high risk group TPAF
%by models without turnover.


