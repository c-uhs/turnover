% -------------------------------------------------------------------------------------------------- SM: Excellent! Reads really well / clear
\textsc{Background.}
Epidemic models of sexually transmitted infections (STIs) 
are often used to characterise										%SM: characterise/determine by estimating/projecting... instead of estimate by estimating/projecting might sound better?
the contribution of risk groups
to overall transmission 
by estimating the transmission population attributable fraction (tPAF)			%SM: I like these edits a lot but perhaps can remove 'over time' just to shorten that first (long) sentence :).
% SM: changed projection to estimate for consitency later in background section where we say "modelled estimates" instead of "modelled projections". 
of unmet prevention and treatment needs within risk groups.
However, evidence suggests that
STI risk is dynamic over an individuals' sexual life course which 
manifests as turnover between risk groups.
We sought to examine the mechanisms by which turnover could
influence modelled estimates of the tPAF of high risk groups.					%SM: decide re: projections or estimates once selected the same word for consistency		
% --------------------------------------------------------------------------------------------------
\textsc{Methods.}
A unifying, data-guided framework was developed to simulate risk group turnover		%SM: most prefer active voice (will forward a great tweet by Bill Miller I thnk), but note that Stef suggested passive voice ha ha :) am ok with that for now because rest of para is active.
in deterministic, compartmental, transmission models.
We applied the framework to an illustrative model of an STI
and examined the mechanisms by which
risk group turnover influenced equilibrium prevalence across risk groups.
We then fit a model with and a model without turnover to the same
risk-stratified STI prevalence targets
and compared the inferred level of risk heterogeneity and
the tPAF of the highest risk group projected by the two models.						%SM: decide re: projected or estimated once selected the same word for consistency
% --------------------------------------------------------------------------------------------------
\textsc{Results.}
The influence of turnover on group-specific prevalence and incidence
was mediated by three main phenomena:
1) changes to the level of herd immunity in the highest risk group;					%SM: I think helpful to say changes, movement, reduction, etc. to show that it was 'shifts' in these by varying rates of turnover that mediated... not the presence/absence of these phenomena.
2) movement of previously high risk individuals with the infection into the low risk group;
3) changes in the number of partnerships where transmission can occur.				%SM: or reduction, but since we do not say which direction (no->slow->fast vs. fast--> slow-->no) turnover... better to say change i think
Faster turnover led to
a smaller ratio of STI prevalence between the highest and lowest risk groups.
Compared to the fitted model without turnover,
the fitted model with turnover inferred greater risk heterogeneity
and consistently projected a larger tPAF of the highest risk group over time.			%SM: decide re: projected or estimated once selected the same word for consistency (I kind of like projected but am easy either way!)
% --------------------------------------------------------------------------------------------------
\textsc{Implications.}
If turnover is not captured in epidemic models,
the projected contribution of high risk groups, and thus,
the potential impact of prioritizing interventions to address their needs could be underestimated.
To aid the next generation of tPAF models,
data collection efforts to parameterize risk group turnover should be prioritized.