% --------------------------------------------------------------------------------------------------
\textsc{Background.}
Epidemic models are often used to estimate
the contribution of risk groups
to overall transmission of sexually transmitted infections (STIs)
by projecting the transmission population attributable fraction (tPAF) over time
based on unmet prevention and treatment needs.
However, evidence suggests that
STI risk is dynamic, that individuals may turnover among risk groups
within their sexual life course.
We sought to examine the mechanisms by which turnover could
influence modelled estimates of the tPAF of high risk groups.
% --------------------------------------------------------------------------------------------------
\textsc{Methods.}
A unifying, data-guided framework was developed to simulate risk group turnover
in deterministic compartmental transmission models.
We applied the framework to an illustrative, model of a STI
and examined the mechanisms by which
risk group turnover influenced equilibrium prevalence across risk groups.
We then fit a model with and a model without turnover to the same
risk-stratified STI prevalence targets
and compared the inferred level of risk heterogeneity and
the tPAF of the highest risk group projected by the two models.
% --------------------------------------------------------------------------------------------------
\textsc{Results.}
The influence of turnover on group-specific prevalence and incidence
was mediated by three main phenomena:
1) herd immunity in the highest risk group;
2) movement of previously high risk individuals with the infection into the low risk group;
3) number of partnerships where transmission can occur.
Faster turnover led to
a smaller ratio of STI prevalence between the highest and lowest risk groups.
Compared to the fitted model without turnover,
the fitted model with turnover inferred greater risk heterogeneity
and consistently projected a larger tPAF of the highest risk group over time.
% --------------------------------------------------------------------------------------------------
\textsc{Implications.}
If turnover is not captured in epidemic models,
the projected contribution of high risk groups, and thus,
the potential impact of prioritizing interventions to address their needs could be underestimated.
To aid the next generation of tPAF models,
data collection efforts to parameterize risk group turnover should be prioritized.