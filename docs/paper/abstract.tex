%\textsc{Background.}
%While many epidemic models stratify populations by heterogeneity in risk,
%few consider movement of individuals between risk groups,
%which we call turnover.
%\textsc{Methods.}
%In this work, we develop a formal framework for modelling risk group turnover.
%The framework subsumes previous approaches to this task,
%and explicitly incorporates available data and assumptions as constraints
%to resolve appropriate rates of turnover.
%We then applied this framework to an illustrative STI/HIV model
%to examine the influence of turnover on model outputs, including:
%the equilibrium infection prevalence and incidence across risk groups,
%the inferred level of risk heterogeneity, and
%the importance of interventions reaching the highest risk group.
%\textsc{Results.}
%Equilibrium infection prevalence in each risk group
%was maximized at moderate but different rates of turnover.
%However, the equilibrium prevalence ratio
%between the highest and lowest risk groups
%consistently decreased with increasing rates of turnover.
%The level of risk heterogeneity inferred
%via model fitting to the same prevalence targets
%was then higher with turnover than without.
%Similarly, after model fitting,
%the estimated importance of interventions reaching the highest risk group
%was higher with turnover than without.
%\textsc{Implications.}
%Equilibrium prevalence predicted by epidemic models
%is influenced by turnover
%\\