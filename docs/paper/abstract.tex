% SM: what is the format & word count restrictions for Epidemics?
%     Can you put here and then I will edit the abstract - thanks!
%     suggest starting with the overarching objective
%     re: implications of turnover on contribution of high risk groups (like in intro) as 
%     motivation/rationale, and then framework --> and then answer.
% JK: Sounds good. How is this looking?
%     re. format: I don't see any restrictions in the Guide for Authors,
%     but most Epidemics articles are 100 - 200 words with no headings/structure.
%     This is at about 250 right now with headings, so we might have to cut a bit,
%     but I'm not sure.
%     May end up cutting the bits marked *
\textsc{Background.}
Epidemic models are often used to estimate
the contribution of high risk groups to the overall epidemic,
helping to inform intervention priorities.
While many epidemic models stratify populations by heterogeneity in risk,
few consider movement of individuals between risk groups,
which we call turnover.
Its not clear
how turnover can be modelled based on epidemiologic data, or % *
how inclusion of turnover in epidemic models will influence
the projected importance of interventions reaching high risk groups.
\textsc{Methods.}
We developed a framework for modelling risk group turnover
in deterministic compartmental transmission models
which incorporates available data and assumptions as constraints.
We applied this framework to an illustrative STI/HIV model
to examine the influence of turnover on:
the equilibrium infection prevalence and incidence across risk groups; % *
the inferred level of risk heterogeneity; and % *
the importance of interventions reaching the highest risk group.
\textsc{Results.}
The equilibrium prevalence ratio
between the highest and lowest risk groups
decreased with increasing rates of turnover. % *
The level of risk heterogeneity inferred
via model fitting to the same prevalence targets
was then higher with turnover than without. % *
After model fitting,
the estimated importance of interventions reaching the highest risk group
was also higher with turnover than without.
\textsc{Implications.}
Equilibrium prevalence predicted by epidemic models
can be strongly influenced by turnover.
The projected importance of interventions prioritizing on high risk groups
could be underestimated if fitted models do not simulate turnover
which is present in reality.
Collection of data which can be used to
parameterize risk group turnover in epidemic models should be prioritized. % *
\\