% SM: what is the format & word count restrictions for Epidemics?
%     Can you put here and then I will edit the abstract - thanks!
%     suggest starting with the overarching objective
%     re: implications of turnover on contribution of high risk groups (like in intro) as 
%     motivation/rationale, and then framework --> and then answer.
% JK: Sounds good. How is this looking?
%     re. format: I don't see any restrictions in the Guide for Authors,
%     but most Epidemics articles are 100 - 200 words with no headings/structure.
%     This is at about 250 right now with headings, so we might have to cut a bit,
%     but I'm not sure.
%     May end up cutting the bits marked *
\textsc{Background.}
Epidemic models are often used to estimate
the contribution of high risk groups to overall transmission 
by projecting the transmission population attributable fraction (tPAF) over time 
of the unmet prevention and treatment needs of high risk groups. 
In the context of sexually transmitted infections, evidence suggests considerable
turnover in risk within the sexual life-course of individuals, especially among those at 
highest risk. We sought to examine the mechanisms by which turnover could 
influence modelled estimates of the tPAF of high risk groups.
\textsc{Methods.}
First, we developed a unifying, data-guided framework to capture risk group turnover 
in deterministic compartmental transmission models.
We then applied the framework to an illustrative, risk-stratified model of 
a sexually transmitted infection and examined: the mechanisms by which 
turnover modifies equilibrium prevalence and incidence across risk groups. % *
We then fit two models (one with and one without turnover) to the same 
risk-stratified prevalence targets
and compared the the inferred level of risk heterogeneity and % *
the tPAF of the highest risk group projected by the two models.
\textsc{Results.}
As turnover varied, there was a shift in three main phenomena which drove variability in
within-group prevalence and incidence at equilibrium:
the relative proportion susceptible within risk-groups, especially the high risk group (and thus, level of herd immunity); 
the sexual network characteristics via overall reduction in number of partnerships where infection could be passed on; 
and influx of previously high-risk individuals with the infection into the low risk group.
Overall, a faster turnover rate led to a smaller ratio 
of infection prevalence between the highest and lowest risk.
Compared to the fitted model without turnover, the fitted model
with turnover inferred (i.e. required) greater risk heterogeneity
and consistently projected a larger tPAF of the highest risk group 
to overall transmission over time.
\textsc{Implications.}
If turnover is not captured in epidemic models, 
the projected contribution of high risk groups, and thus, 
the potential impact of prioritizing interventions to reach
high risk groups, could be underestimated. To aid the next generation of tPAF models, 
data collection efforts to parameterize risk group turnover should be prioritized. % *
\\