% --------------------------------------------------------------------------------------------------
\textsc{Background.}
Epidemic models of sexually transmitted infections (STIs)
are often used to characterize
the contribution of risk groups
to overall transmission
by projecting the transmission population attributable fraction (tPAF)
of unmet prevention and treatment needs within risk groups.
However, evidence suggests that
STI risk is dynamic over an individual's sexual life course,
which manifests as turnover between risk groups.
We sought to examine the mechanisms by which turnover
influences modelled projections of the tPAF of high risk groups.
% --------------------------------------------------------------------------------------------------
\textsc{Methods.}
We developed a unifying, data-guided framework to simulate risk group turnover
in deterministic, compartmental transmission models.
We applied the framework to an illustrative model of an STI
and examined the mechanisms by which
risk group turnover influenced equilibrium prevalence across risk groups.
We then fit a model with and without turnover to
the same risk-stratified STI prevalence targets
and compared the inferred level of risk heterogeneity and
tPAF of the highest risk group projected by the two models.
% --------------------------------------------------------------------------------------------------
\textsc{Results.}
The influence of turnover on group-specific prevalence
was mediated by three main phenomena:
movement of previously high risk individuals with the infection into lower risk groups;
changes to herd immunity in the highest risk group; and
changes in the number of partnerships where transmission can occur.
Faster turnover led to
a smaller ratio of STI prevalence between the highest and lowest risk groups.
Compared to the fitted model without turnover,
the fitted model with turnover inferred greater risk heterogeneity
and consistently projected a larger tPAF of the highest risk group over time.
% --------------------------------------------------------------------------------------------------
\textsc{Implications.}
If turnover is not captured in epidemic models,
the projected contribution of high risk groups, and thus,
the potential impact of prioritizing interventions to address their needs, could be underestimated.
To aid the next generation of tPAF models,
data collection efforts to parameterize risk group turnover should be prioritized.