% SM: what is the format & word count restrictions for Epidemics?
%     Can you put here and then I will edit the abstract - thanks!
%     suggest starting with the overarching objective
%     re: implications of turnover on contribution of high risk groups (like in intro) as 
%     motivation/rationale, and then framework --> and then answer.
% JK: Sounds good. How is this looking?
%     re. format: I don't see any restrictions in the Guide for Authors,
%     but most Epidemics articles are 100 - 200 words with no headings/structure.
%     This is at about 250 right now with headings, so we might have to cut a bit,
%     but I'm not sure.
%     May end up cutting the bits marked *
\textsc{Background.}
Epidemic models are often used to estimate
the contribution of high risk groups to overall transmission 
by projecting the transmission population attributable fraction (tPAF) over time 
of the unmet prevention and treatment needs of high risk groups. 
In the context of sexually transmitted infections, data suggest considerable
turnover in risk within the sexual life-course of individuals, especially among those at 
highest risk. We sought to examine the mechanisms by which turnover could 
influence modelled estimates  of the tPAF of high risk groups.
\textsc{Methods.}
First, we developed a unifying, data-guided framework to capture risk group turnover 
in deterministic compartmental transmission models.
We then applied the framework to an illustrative, risk-stratified model of 
a sexually transmitted infection and examined the mechanisms by which 
turnover modifies equilibrium prevalence and incidence across risk groups; % *
and the influence of including versus excluding turnover in fitted models with respect to
the inferred level of risk heterogeneity and % *
the tPAF of the highest risk group.
\textsc{Results.}

The equilibrium prevalence ratio
between the highest and lowest risk groups
decreased with increasing rates of turnover. % *
The level of risk heterogeneity inferred
via model fitting to the same prevalence targets
was then higher with turnover than without. % *
After model fitting,
the estimated importance of interventions reaching the highest risk group
was also higher with turnover than without.
\textsc{Implications.}
Equilibrium prevalence predicted by epidemic models
can be strongly influenced by turnover.
The projected importance of interventions prioritizing on high risk groups
could be underestimated if fitted models do not simulate turnover
which is present in reality.
Collection of data which can be used to
parameterize risk group turnover in epidemic models should be prioritized. % *
\\