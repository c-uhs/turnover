** R O U G H **
\begin{itemize}
  \item key factors in risk of HIV acquisition
  \item intro HIV modelling, prior work showing importance of heterogeneity
  \item comments on applicability of these concepts to non-HIV transmissible diseases
        (other STI, non-S TI)
  \item prior work on turnover, comment on need for ``equilibration'' / ``burn-in'' period
\end{itemize}
What do we mean by ``risk group dynamics''?
\begin{enumerate}
  \item inclusion of risk groups at all (yes / no)
  \item inclusion of turnover among these groups (yes / no, how?)
  \item consideration of how groups are re-balanced given differential attributable death
  -- subject of future work
\end{enumerate}
\par
From \citet{Eaton2014}:
\textit{Two behavioral parameters
-- the rate of transition from higher- to lower-risk groups and \textup{[\dots]} -- 
were particularly important for simulating the observed prevalence trend in many different ways,
as well as determining the intervention impact.}
\par
From SR by \citet{Mishra2012}:
N = 107 models included behavioural heterogeneity, while
N = 88 did not.
\par
From SR by \citet{Ronn2017}:
N = 34 included risk heterogeneity, while
N = 11 models had none.
\par
Some papers to include in \tab{tab:prior-work}:
\cite{Barnighausen2012,
      Cremin2013,
      Eaton2014,
      Estill2012,
      Granich2009,
      Hallett2008,
      Johnson2006,
      Phillips2011,
      Rosenberg2004,
      Shah2016}