\documentclass[a4]{article}
\usepackage[margin=3cm]{geometry}
\usepackage{graphicx}
\graphicspath{{logo/}}
\setlength{\parindent}{0pt}
\setlength{\parskip}{1em}
\pagenumbering{gobble}
\begin{document}
\begin{minipage}{0.5\linewidth}
  \includegraphics[height=0.8cm]{map-cuhs.eps}\\[1em]
  \includegraphics[height=1.2cm]{smh.eps}\\[1em]
  \includegraphics[height=1cm]{uoft.eps}
\end{minipage}%
\begin{minipage}{0.5\linewidth}
  \begin{flushright}
    Sharmistha Mishra\\
    MAP Centre for Urban Health Solutions\\
    Li Ka Shing Knowledge Institute\\
    St Michael's Hospital,
    Unity Health Toronto\\
    University of Toronto\\
    \texttt{sharmistha.mishra@utoronto.ca}
  \end{flushright}
\end{minipage}
\\[2em]
Prof. Neil Ferguson \& Prof. Dr. Hans Heesterbeek\\
Editors-in-Chief of \textit{Epidemics}\\[1em]
% ==================================================================================================
\textbf{Re. Submission of a research paper to Epidemics}\\[1em]
% ==================================================================================================
Dear Editors,
\par
We are pleased to submit to you the attached manuscript entitled
\textit{Modelling Epidemic Risk Group Dynamics}%
for publication as a (regular issue) research paper.
\par
Epidemic models of sexually transmitted infections (STI) such as HIV
are increasingly used to help guide intervention priorities
by quantifying the contribution of high risk groups to overall transmission.
However, movement of individuals between risk groups
-- which we call ``turnover'' --
is not often considered, and it is not clear how
estimates of ``contribution'' would change if turnover was considered.
Moreover, methods to implement turnover which incorporate epidemiological data
appear to be lacking.
\par
In this paper, we present new methods to design and parameterize
systems of turnover among risk groups in epidemic models,
based on epidemiological data.
We then leverage these methods to examine the influence of turnover
on several epidemic model outputs in a representative STI,
including the contribution of the highest risk group to overall transmission.
We find that this contribution could be underestimated in models without turnover
following model fitting,
which suggests that models without turnover may underestimate
the importance of reaching high risk groups with interventions.
\par
While our study is framed around a representative STI,
risk group turnover is applicable to a wide range of 
epidemic contexts, infections, and hosts.
We therefore aimed to reach a broad audience in the readers of \textit{Epidemics},
who we expect will benefit from the proposed framework for parameterizing turnover,
and our mechanistic insights as to the influence of turnover on epidemic model outputs.
\par
Researchers who we think would provide excellent review of this work include:
James S. Koopman (University of Michigan);
Michael Pickles (University of Manitoba);
Marie-Claude Boily (Imperial College London);
Jeffrey Eaton (Imperial College London);
Hein Stigum (Norwegian Institute of Public Health); and
Geoff Garnett (Bill and Melinda Gates Foundation).
\par
We look forward to hearing from you about how we can share our contributions
with the readers of \textit{Epidemics}.
\\[2em]
Sincerely,\\[1em]
Sharmistha Mishra
\end{document}
