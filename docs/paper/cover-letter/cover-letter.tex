\documentclass[a4]{article}
\usepackage[margin=3cm]{geometry}
\usepackage{graphicx}
\graphicspath{{logo/}}
\setlength{\parindent}{0pt}
\setlength{\parskip}{1em}
\pagenumbering{gobble}
\begin{document}
\begin{minipage}{0.5\linewidth}
  \includegraphics[height=0.8cm]{map-cuhs.eps}\\[1em]
  \includegraphics[height=1.2cm]{smh.eps}\\[1em]
  \includegraphics[height=1cm]{uoft.eps}
\end{minipage}%
\begin{minipage}{0.5\linewidth}
  \begin{flushright}
    Sharmistha Mishra\\
    MAP Centre for Urban Health Solutions\\
    Li Ka Shing Knowledge Institute\\
    St Michael's Hospital,
    Unity Health Toronto\\
    University of Toronto\\
    \texttt{sharmistha.mishra@utoronto.ca}
  \end{flushright}
\end{minipage}
\\[2em]
Prof. Neil Ferguson \& Prof. Dr. Hans Heesterbeek\\
Editors-in-Chief of \textit{Epidemics}\\[1em]
% ==================================================================================================
\textbf{Re. Submission of a research paper to Epidemics}\\[1em]
% ==================================================================================================
Dear Editors,
\par
We are pleased to submit the attached manuscript entitled
\textit{Modelling Epidemic Risk Group Dynamics}%
for consideration in  \textit{Epidemics}.							%SM: I do not see instructions to say whether it is a regular issue or not in the cover letter. I think that goes into the online submission drop-downs maybe. Usually do not say regualr or not unless submitting to a special call/special issue.
\par
Epidemic models of sexually transmitted infections (STI)
are increasingly used to 
quantify the contribution of high risk groups to overall transmission.
Contributions are often measured from epidemic models 
by calculating the transmission population attributable 
fraction of unmet prevention and treatment needs of specific risk groups.
However, movement of individuals between risk groups
-- i.e. ``turnover'' --					%SM: i do not think we invented the term :) 
is not often considered, despite epidemiological evidence to suggest 
that STI risk is dynamic over an individual's sexual life course.
Moreover, there exist various implementations of turnover without
a unified approach of implementing turnover using epidemiological data.		
\par
In this paper, we examined the mechanisms by which turnover could
influence modelled estimates of the transmission population 
attributable fraction of high risk groups. First, we developed
a new, unified framework to design and parameterize
systems of turnover among risk groups in epidemic models when
guided by epidemiological data. We then 
used the new framework in an illustrative, risk-stratified model of an STI
and identified for the first time: (a) that fitted models without turnover underestimate
the contribution of high risk groups to overall transmission; and (b) 
the underestimate stems from inferred 
risk heterogeneity in the presence/absence of turnover which in turn is mediated by
three phenomena that shift with varying turnover rates: herd immunity 
in the highest risk group, influx of infectious individuals into the low risk group,
and changes in number of partnerships where transmission can occur.

\par
To our knowledge, this is the first study to examine and thus generate new insights 
into the influence of turnover
on the transmission population attributable fraction of high risk groups; and the first
to detail the mechanisms underlying the relationship. The findings have important
implications for projecting not only the contribution of high risk groups to overall transmission 
but also when projecting the potential transmission impact of interventions priortiized to high risk groups.

\par
While our study is framed around an illustrative STI,
risk group turnover is applicable to a wide range of 
epidemic contexts, infections, and hosts. 
We therefore believe the new framework for parameterizing turnover, 
the findings, and the mechanistic insights will be of interest to the broad readership of
 \textit{Epidemics},


%Researchers who we think would provide excellent review of this work include:
%James S. Koopman (University of Michigan); 
%Michael Pickles (University of Manitoba);
%Marie-Claude Boily (Imperial College London); 
%Jeffrey Eaton (Imperial College London);
%Hein Stigum (Norwegian Institute of Public Health); and
%Geoff Garnett (Bill and Melinda Gates Foundation).
\par
Thank you for your consideration and we look forward to hearing from you.
\\[2em]
Sincerely,\\[1em]
Sharmistha Mishra and Jesse Knight, on behalf of co-authors
\end{document}
