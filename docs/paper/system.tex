In this section we introduce a system of
risk groups, flows between them, and equations
which can be used to describe turnover
in deterministic compartmental epidemic models.
% ==================================================================================================
\subsection{Notation}
Consider a population divided into $G$ risk groups.
We denote the number of individuals in risk group $i \in [1, \dots, G]$ as $x_i$
and the set of all risk groups as $\bm{x} = \{x_1, \dots, x_G\}$.
The total population size is $N = \sum_i {x_i}$,%
\footnote{Here, as in many models, ``total population'' actually represents
  a subset of the population with a given duration in the model --
  e.g. an age-constrained range.}
and the proportion of the total population that each group represents
is denoted as $\hat{x}_i = x_i / N$.
We do not consider stratification of age groups.
Individuals enter the model at a rate $\nu$ per year,
and exit at a rate $\mu$ per year.
However, the proportion of individuals entering into group $i$ from outside the model
may be different from
the proportion of individuals currently in group $i$ in the model ($\hat{x}_i$).
Therefore, we distinguish these proportions
and denote the proportion entering into group $i$ as $\hat{e}_i$.
For example, a higher proportion of youth (individuals entering the model, $\hat{e}_i$)
may engage in high-risk sexual behaviour,
as compared to the population overall ($\hat{x}_i$).
It will later be shown how rates of turnover can maintain such a system at equilibrium.
% The number of entrants into group $i$ is therefore given by
% $\nu N \hat{e}_i$.
% This assumption then implies then implies the existence of
% a ``source'' population $\bm{e} = \{e_1, \dots, e_G\}$,
% from which model entrants originate.
% The size of this population is then assumed to be the same as $\bm{x}$ ($\sum_i e_i = N$)
% so that the entry rate $\nu$ can be used in the usual way,
% as a proportion of $N$.
\par
Turnover transitions may occur between any two groups, in either direction;
therefore we denote the turnover rates as a $G \times G$ matrix $\phi$.
The element $\phi_{ij}$ corresponds to the proportion of individuals in group $x_i$
who move from group $x_i$ to group $x_j$ each year.
An example matrix is given in Eq.~(\ref{eq:phi}),
where we write the diagonal elements as $*$ since they represent
transitions from a group to itself, which are inconsequential.
\begin{equation}\label{eq:phi}
\phi = \left[\begin{array}{cccc}
	         *          & x_1  \rightarrow x_2 & \cdots & x_1 \rightarrow x_G \\[0.5em]
	x_2 \rightarrow x_1 &          *           & \cdots & x_2 \rightarrow x_G \\[0.5em]
	      \vdots        &        \vdots        & \ddots &       \vdots        \\[0.5em]
	x_G \rightarrow x_1 & x_G \rightarrow x_2  & \cdots &          *
\end{array}\right]
\end{equation}
These transition flows and the associated rates
are also shown for $G = 3$ in Figure~\ref{fig:system}.
\begin{figure}
  \centering
  \includegraphics[width=0.5\linewidth]{turnover-system}
  \caption{System of states and flows between them for $G = 3$}
  \label{fig:system}
\end{figure}
% ==================================================================================================
\subsection{Parameterization}\label{ss:params}
Next, we consider the goal of constructing a system like the one introduced above
which reflects the risk group dynamics observed in a specific context.
We assume that the relative sizes of the risk groups in the model ($\bm{\hat{x}}$)
are already known, and should remain constant over time.
Thus, what remains is to estimate the values of the parameters:
$\nu$, $\mu$, $\bm{\hat{e}}$, and $\phi$,
using commonly available sources of data.
% -------------------------------------------------------------------------------------------------- %% please see the system.tex edited document I sent via email for comments before the Total Population Size subsection

\subsubsection{Total Population Size}\label{sss:params-nu-mu}
The total population size $N(t)$ is related to
the rates of exogenous population entry $\nu(t)$ and exit $\mu(t)$.  % suggest clarifying what is meant here (be more precise) - how is it related? does one define the other? or does the value of N(t) change based on whether nu and mu have the same value vs. different values?
We allow the proportion entering the modeled system to vary by risk group via $\bm{\hat{e}}$,
while the exogenous exit rate has the same value for each group. % risk vs. activity group? ensure a consistency check for terminology
We assume that there is no disease-attributable death.
Because the values of $\nu$ and $\mu$ are the same for each risk group, 
they can be estimated independent of
$\bm{\hat{x}}$, $\bm{\hat{e}}$, and $\phi$.
\par
The difference between entry and exit rates
defines the rate of population growth:
\begin{equation}\label{eq:growth-G}
\mathcal{G}(t) = \nu(t) - \mu(t) 
\end{equation}
The total population may then be defined using an initial population size $N_0$ as:
\begin{equation}\label{eq:growth-vary}
  N(t) = N_0 \exp{\left(\int_{0}^{\,t}{\log\big(1+\mathcal{G}(\tau) \big)d\tau}\right)}
\end{equation}
which, for constant growth, simplifies to:   %when you say 'familiar' do you mean it can be cited? suggest including citation for equation (3) and (4) as they have been shown/used before from demography
\begin{equation} \label{eq:growth-const}
  N(t) = N_0 {(1 + \mathcal{G})}^{t}
\end{equation}
Census data can be used to source the total population size in a given geographic setting over time $N(t)$,  %can cite an census source - eg. World Bank Population Size data
thus allowing Eqs.~(\ref{eq:growth-vary})~and~(\ref{eq:growth-const})
to be used to estimate $\mathcal{G}(t)$.
\par
If the total population size is assumed to be constant, then $\mathcal{G}(t) = 0$ and $\nu(t) = \mu(t)$.
If population growth occurs at a stable rate, then 
$\mathcal{G}$ is fixed at a constant value
which can be estimated via Eq.~(\ref{eq:growth-const})
using any two values of $N(t)$, separated by a time interval $\tau$:
\begin{equation}
\mathcal{G}_{\tau} = {\frac{N(t+\tau)}{N(t)}}^{\frac{1}{\tau}} -1
\end{equation}

If the rate of population growth $\mathcal{G}$ varies over time,
this % which process? - refer to the relevant equation...in general, try to limit use of 'this', 'these', etc. and instead use the specific bookmark 
process can be repeated for consecutive time intervals,
and the complete function $\mathcal{G}(t)$ approximated piecewise by constant values.
The piecewise approximation is generally more feasible than exact solutions using Eq.~(\ref{eq:growth-vary}),
and can reproduce $N(t)$ accurately for small enough intervals $\tau$,
such as one year.  %the above sentence would benefit from a citation for 'feasibility' and 'reproducing accurately' otherwise it may be considered editorializing
\par

The value of $\mathcal{G}(t)$ does not imply any particular
values of $\nu(t)$ or $\mu(t)$,
since any choice of $\mu(t)$ can be compensated by an appropriate choice of $\nu(t)$,
and vice versa, as in Eq.~(\ref{eq:growth-G}).										%hard to follow this sentence, suggest rephrasing for clarity

Most modeled systems assume a constant duration of time that individuals spend in the model $\delta(t)$,  %should be cited (Anderson & May textbook, etc.)
which is related to the exit rate by:
\begin{equation} \label{eq:duration-model}
\delta(t) = \mu^{-1}(t)
\end{equation}

in the context of sexually transmitted infections, the duration of time usually reflects
the average sexual life-course of individuals from age 15~to~50 years such that
$\delta = 35$ years. 
The duration $\delta may also vary with time to reflect changes in life expectancy.
$\mu(t)$ can then be defined as $\delta^{-t}(t)$
following Eq.~(\ref{eq:duration-model}),
and $\nu(t)$ defined as $\mathcal{G}(t) - \mu(t)$
following Eq.~(\ref{eq:growth-G}).

% --------------------------------------------------------------------------------------------------
\subsubsection{Turnover}\label{sss:params-turnover}
Next, we present methods for resolving
the distribution of individuals entering the risk model $\bm{\hat{e}}(t)$ and
the rates of turnover $\phi(t)$,
assuming that entry and exit rates $\nu(t)$ and $\mu(t)$ are known.
First we formulate the problem as a system of equations.
Then we explore the data and assumptions required			%suggest precise language/terminology especially in the methods/results of papers
to solve for the values of parameters in the system.
The $(t)$ notation is omitted throughout this section for clarity,
though time-varying parameters can be estimated by
repeating the necessary calculations for each $t$.
\par
The number of risk groups $G$ specifies the number of 
$G$ unknown elements in $\bm{\hat{e}}$ and $G(G-1)$ unknown elements in $\phi$.
We collect these unknowns in the vector
$\bm{\theta} = \left[\bm{\hat{e}}, \bm{y}\right]$,
where $\bm{y} = \mathrm{vec}_{i \ne j}(\phi)$.
For example, for $G = 3$, the vector $\bm{\theta}$ is defined as:  %nice! i like how you provide a clear example
\begin{equation}
\bm{\theta} = \left[
\begin{array}{ccccccccc}
\hat{e}_1 & \hat{e}_2 & \hat{e}_3 & \phi_{12} & \phi_{13} & \phi_{21} & \phi_{23} & \phi_{31} & \phi_{32}
\end{array}\right]
\end{equation}
We then define a linear system of equations		%can use nice phrasing like 'crux of this framework' in the discussion section or in the opening of a section, but suggest "we-do-this-do-that" language here
which uniquely determine the elements of $\bm{\theta}$:
\begin{equation}\label{eq:system-matrix}
\bm{b} = A \thinspace \bm{\theta}
\end{equation}  
where $A$ is a $M \times G^2 $ matrix
and $\bm{b}$ is a $M$-length vector.
That is, each row in $A$ and $\bm{b}$ specifies
an assumed relationship involving elements of
$\bm{\hat{e}}$ and $\phi$.							%suggest rephrase the sentence for clarity ("species and assumed relationship" --> simpler words)
Given a sufficient number $M = G^2$ of unique constraints, % the sentence is not clear. Give examples of unique constraints in words here an then go into it in the next section
the parameters can be calculated algebraically using $\bm{\theta} = A^{-1}\bm{b}$.
Some examples of constraints are explored below: constant group size;....  % list these here for the reader --> is each of the following sections describing a constraint? - make that really clear in the opening what the constraint is
% --------------------------------------------------------------------------------------------------
\paragraph{Constraint 1: Constant Group Size}			%number the example constraints. Similar to Box 1 --> a box or flow diagram to show the various approaches to constraints will be useful - e.g. modeler can either ensure constant group size or duration spent in each risk group. alternatively present the options as a subset of the third point in Box 1 (under Turnover) --> would lead nicely into decisions based on data and assumptions. Can discuss at next PM if suggestion not clear from my written comments.
One epidemiological feature that epidemic models consider
is whether or not the relative population size of risk groups are constant over time. %citation = modeling papers where they want to keep constant vs. where the relative pop. size changes over time (e.g. in models with behaviour change over time)
Enabling constant group size assumes a stable level of heterogeneity over time.   %can mention this with respect to sex work population size in the discussion section
First, we define the ``conservation of mass'' equation for group $x_i$,
wherein the rate of change of the group
is defined as the sum of flows in\,/\,out of the group:
\begin{equation}\label{eq:mass-balance}
\frac{d}{dt}x_i
= \nu \thinspace e_i + \sum_{j}{\phi_{ji} \thinspace x_j}
- \mu \thinspace x_i - \sum_{j}{\phi_{ij} \thinspace x_i}
\end{equation}

Eq.~(\ref{eq:mass-balance}) is written in terms of
absolute population sizes $\bm{x}$ and $\bm{e}$, 
but can be written 
as proportions $\bm{\hat{x}}$ and $\bm{\hat{e}}$ 
when dividing by $N$. Working with proportions are
useful as $N$ need not be known.			
% not sure if there is additional normalization needed, actually  % SM: explain this to me?
If we assume that the average proportions of each group $\hat{x}_i$ is constant over time,  %why average? we are dealing with a deterministic set of equations noh? or do you mean at steady state?
then the desired rate of change for risk group $i$
will be equal to the rate of population growth of risk group, $\mathcal{G} x_i$.
Substituting this into Eq.~(\ref{eq:mass-balance}),  %specify 'this'? - its ok to be repetitive to avoid any possible misunderstandings among those of us with short attention spans! (yup - i've been schooled in limiting use of 'this' and 'these'! 
and simplifying yields:
\begin{equation}\label{eq:system}
\nu \thinspace x_i
= \nu \thinspace e_i + \sum_{j}{\phi_{ji} \thinspace x_j}
- \sum_{j}{\phi_{ij} \thinspace x_i}
\end{equation}
Factoring the left and right hand sides in terms of $\bm{\hat{e}}$ and $\phi$,
we obtain $G$ unique constraints.
For $G = 3$, this yields the following 3 rows as the basis of $\bm{b}$ and $A$:
\begin{equation}\label{eq:b-A-basis}
\bm{b} = \left[\begin{array}{c}
\nu x_1 \\ \nu x_2 \\ \nu x_3
\end{array}\right];\qquad
A = \left[\begin{array}{ccccccccc}
 \nu  & \cdot & \cdot & -x_1  & -x_1  &  x_2  & \cdot &  x_3  & \cdot \\
\cdot &  \nu  & \cdot &  x_1  & \cdot & -x_2  & -x_2  & \cdot &  x_3  \\
\cdot & \cdot &  \nu  & \cdot &  x_1  & \cdot &  x_2  & -x_3  & -x_3  \\
\end{array}\right] 
\end{equation}
The $G$ constraints ensure risk groups do not change size over time.
However, an unique solution requires
an additional $G(G-1)$ constraints.
For $G = 3$, 6 additional constraints are required: [...list them/examples here...].  % what are these constraints? the specified elements? its not clear until after reading the next section - suggest introducing/listing things before moving to subsequent paragraph/section. Also - suggest including the 'rationale' or epidemiological 'reason' for each constraint. I put some edits to that effect, but may need to discuss at PM if not clear what I meant by this...
% --------------------------------------------------------------------------------------------------
\paragraph{Specified Elements}
The simplest type of additional constraint is to
directly specify elements in $\bm{\hat{e}}$ or $\phi$.								%by specifying, do you mean provide input values directly from the data? and to every element as independent values?
This constraint may be appended to $\bm{b}$ and $A$ using indicator encoding:		%suggest defining "indicator encoding" with a citation for the rest of us...; suggest precision re: what 'this' refers to (i.e. a value for every element)?
with $b_k$ as the specified value and $A_k = [0,\dots,1,\dots,0]$ as the indicator vector.	%what does subscript k refer to? the value of an element in $\bm{\hat{e}}$ ?
For example, for $G = 3$, if it is known that 20\% of individuals
enter directly into risk group $x_1$ upon entry into the model ($\hat{e}_1 = 0.20$),
then $\bm{b}$ and $A$ can be augmented with:
\begin{equation}
\bm{b}' = \left[\begin{array}{c} 0.20 \end{array}\right];\qquad
A' = \left[\begin{array}{ccccccccc}
1 & \cdot & \cdot & \cdot & \cdot & \cdot & \cdot & \cdot & \cdot \\
\end{array}\right] 
\end{equation}																% nice! I like when you provide examples - really helps re: making it concrete & increasing and clarity 
If the data suggest zero turnover from group $i$ to group $j$,
then the above approach can also be used to set $\phi_{ij} = 0$.						%if 'above approach' refers to equation 12, then say that otherwise - just be careful when using terms like 'this approach' - etc. as there are >10 times in one large section when approaches are discussed. 
Note that the elements of $\bm{\hat{e}}$ must sum to 1.
Therefore, specifying all elements in $\bm{\hat{e}}$
will only provide $G-1$ constraints (in addition to constraint 1 defined in section XXX),		%when talking about 'additional' constraints, try to refer back to what it is in addition to
as the last element is redundant.												%redundant? or is it because it sums to 1 and therefore is a 'known' value?... not sure redundant is the correct term?
This relationship is implicit in Eq.~(\ref{eq:b-A-basis}),								%which relationship? lots of 'this', etc. and reader is losing track...
so it is not necessary to supply a constraint $1 = \sum_{i} \hat{e}_i$.						%am not sure I follow why it is not necessary to supply the summation constraint? - is it a check? 
Similar redundancies or inconsistencies can emerge for constraints on $\phi$, as noted below.	%similar re: sentence above? specify which sections 'below'-- but only mention redundancies in sentence above, where are the inconsitencies? rephrase sentence for clarity and precision
% --------------------------------------------------------------------------------------------------
\paragraph{Constraint 3: Duration in each Risk Group}						% can discuss at next PM, but a brief table of what type of data and assumptions for each constraint would be useful for epidemiology audience (thinking about readers like Sheree, etc.)
Constraint 1 assumes that the relative population size of each group remains constant.
Another epidemiological feature that epidemic models consider
is whether or not the duration of time spent within a given risk group remains constant. 
For example, in STI transmission models that include formal sex work, 
one may assume that the duration of formal sex work remains stable over time.  %citation = modeling paper(s) where they want to keep duration of sex work constant?
Thus, another practical constraint may be derived from
the average duration of time individuals spend in a risk group.
The duration $\delta_i$ is defined as the inverse of all efferent flow rates:		%define efferent in brackets (like exogenous, etc. - we will want to define all of these early and use consistently)
\begin{equation}\label{eq:duration-group}
\delta_i = {\bigg(\mu + \sum_{j}{\phi_{ij}}\bigg)}^{-1}
\end{equation}
Estimates of the duration in a given risk group can be 
sourced from cross-sectional survey data where participants are asked about 
how long they have engaged in a particular practice - such as sex in exchange for money. % citation
Data on duration may also be sourced from longitudinal data where repeated measures of self-reported sexual behaviour or 
proxy measures of sexual risk data are collected. %citation

% would not introduce key populations here (i.e. limit # of new terms that need to be defined). if talking about key populations, reasonable to do so in Discussion +/- Introduction.

Data on duration in each risk group can be used to define constraints on $\phi$ by
rearranging Eq.~(\ref{eq:duration-group}) to yield:
${\delta_{i}}^{-1} - \mu = \sum_{j}{\phi_{ij}}$.
For example, if for $G = 3$,
and the average duration in group $x_1$ is known to be $\delta_1 = 5$ years,
then $\bm{b}$ and $A$ can be augmented with:
\begin{equation}
\bm{b}' = \left[\begin{array}{c}
{5}^{-1} - \mu
\end{array}\right];\qquad
A' = \left[\begin{array}{ccccccccc}
\cdot & \cdot & \cdot & 1 & 1 & \cdot & \cdot & \cdot & \cdot \\
\end{array}\right]
\end{equation}
As noted above, redundancies can emerge							%flowery sentence alert. simplify for clarity. what is meant by redundancy? as per above comment - is 'redunancy' the most accurate and precise term? and why do they emerge? do they not simply exist? the term 'emerge' has some specific scientific meaning from the literature on emergent properties of complex systems. I worry we are using a lot of different words and losing precision and clarity so lets keep the language really direct and simple as we can
when constraining turnover rates $\phi$ via duration $\delta_i$.				%rephrase sentences here for clarity
Namely, specifying all efferent flow rates $\phi_{ij}$ for one group $i$
will fully determine the duration in the group $\delta_i$.					%define what is meant by 'fully determine' and include a citation
In general, each constraint which is not redundant will increase the rank of $A$ by one,		%rephrase for clarity
and $A$ must be full rank ($G^2$) to uniquely determine the parameters in $\bm{\theta}$.
% --------------------------------------------------------------------------------------------------
\paragraph{Turnover Balance} 	%define what is meant by turnover balance? found this section difficult to follow/understand re: rationale for the constraint...
A final constraint is.... %fill in with an explicit link to epidemiology or mechanism (what is the 'real world' link here?) - see above sections.

If we assume that
the absolute number of individuals moving between two risk groups is
related by a ratio $\frac{r_i}{r_j}$ so that:
$\phi_{ij} x_{i} r_{i} = \phi_{ji} x_{j} r_{j}$. we can
constrain $\phi_{ij}$ and $\phi_{ji}$.				%is constrain the right word or do we mean constrain the values of...? or is "fixed" a better phrasing here? 
For example, for $G = 3$,
if we assume that the number of individuals moving between groups $x_1$ and $x_2$
is related by $\frac{3}{2}$,
then $\bm{b}$ and $A$ can be augmented with:
\begin{equation}
\bm{b}' = \left[\begin{array}{c}
0
\end{array}\right];\qquad
A' = \left[\begin{array}{ccccccccc}
\cdot & \cdot & \cdot & 3 x_1 & \cdot & -2 x_2 & \cdot & \cdot & \cdot \\
\end{array}\right] 
\end{equation}
If we assume that the absolute number of individuals moving between the groups is equal,		
then $\frac{r_i}{r_j} = 1$.		%it is recommended to avoid adjectives as much as possible in most scientific writing
Again, care should be taken to avoid redundancies and inconsistencies with other constraints. %found this sentence to read like a textbook instead of a scientific paper. and kind of condescending in tone? should be rephrased.
%% JK: I feel this should be elaborated on,			
%% but its honestly very difficult to summarize the situations where this occurs.			%I found this section hard to follow re: "why"/"rationale" for the constraint. Suggest getting input from HM and LW about how to talk about the constraint and what it could 'mean' as we do in the above section on constraints. That is, for every constraint - why is it a constraint? what does it mean?
% --------------------------------------------------------------------------------------------------
\paragraph{Minimization}
Lastly, we can avoid specifying a complete set of constraints on $\bm{\theta}$		%give an example rationale of why / under what conditions one may want to do this
($A$ can be less than full rank)
if the problem is posed as a minimization problem, namely:
\begin{equation}\label{eq:system-optimize}
\bm{\theta}^{*} = {\arg \min}
\thinspace f(\bm{\theta}),
\quad \textrm{subject to:}
\enspace\bm{b} = A\thinspace\bm{\theta};
\enspace\bm{\theta} \ge 0
\end{equation}
where $f$ is a function which globally constrains $\bm{\theta}$,
such as: ${\left|\left| \,\cdot\, \right|\right|}_2$.
That is, when a sufficient number of assumptions cannot be made about $\bm{\hat{e}}$ and $\phi$		%what is meant by 'sufficient' - perhaps we can be more explicit/precise here?
to yield a unique solution,
the minimization approach can be used to find the smallest values of $\bm{\hat{e}}$ and $\phi$
which satisfy the given constraints.														%as you had done nicely above, suggest including an example scenario. this sentence i had to re-read a few times to try and understand...perhaps it can be broken up a bit (or increase precision re: terminology)?
Numerical solutions to minimization problems are widely available,
such as the Non-Negative Lease Squares solver \citep{Lawson1995}.%
\footnote{The \textsc{nnls} solver is available in Python:
   \href{https://docs.scipy.org/doc/scipy/reference/generated/scipy.optimize.nnls.html}
{\texttt{https://docs.scipy.org/doc/scipy/reference/generated/scipy.optimize.nnls.html}}.}
%% JK: may want to add a paragraph on inconsistent systems (no true solution):					%suggest framing as 'conditions' under which there may be no true solution, and what this means - practically - for including turnover in epidemic models.
%% how to identify them and what to do about them.											%agree - and see above comment re: framing
% ==================================================================================================
\subsection{Previous Approaches}														%this section can be strengthened a lot more by more directly showing how it may be subsumed into the proposed unified approach. 	and more importantly - rather than being 'subsumed' ---> can the goals of the implementation used in prior studies be replicated with the unified approach we propose? -- the goals/objectives/constraints are what matter. 

The three major features pertaining to risk group dynamics and the associated assumptions			%terminology consistency check
are summarized in Box~\ref{box:assumptions}. They include: population growth; heterogeneity in risk; and turnover in risk.

Few epidemic models of sexually transmitted infections with 
heterogeneity in risk 
have simulated turnover among risk groups.												%could benefit from a citation is possible without actually doing a systematic review specifically for this. are there any systematic review or comprehensive reviews you can cite? e.g. Watts paper? 
And they have implemented turnover in various ways. In this section, we review
X prior implementation of turnover and each study's objectives for turnover (e.g. constant relative population size of risk groups over time)
; and highlight how the unified approach
proposed in Section X could be used to achieve the same objectives.			%for each paper/model --> define their objectives/constraints for turnover, and then talk about who they went about; then link to your unified approach explicitly and clearly.
																	
In a $G = 2$ system,								
\citet{Stigum1994} use a parameter $\kappa$ to define
the rate of movement of individuals from a core group into the remaining population,
which is balanced by an equal number of individuals moving in the opposite direction.
Considered through the proposed framework,
this defines a unique system through assuming:
a \emph{specified element} $\kappa = \phi_{12}$,
\emph{constant group size}, and \emph{balanced turnover}.
\citet{Henry2015} employ the same assumptions for another $G = 2$ system,
but define a ``re-selection rate'' $\omega$ to control
the relative rate of turnover, so that
$\phi_{12} = \omega \hat{x}_{2}$ and $\phi_{21} = \omega \hat{x}_{1}$.
In a $G = 3$ system,
\citet{Eaton2014} use a parameter $\Psi$ to define the rate of movement from
high-to-low, high-to-medium, and medium-to-low risk groups each year,
but assume no transitions in the opposite direction.
Thus, all six elements of $\phi$ are \emph{specified},
while \emph{constant group sizes} are ensured by computing
the required distribution of model entrants $\bm{\hat{e}}$
via equation [S12] in the Supplemental Information.
\par
The $G = 7$ system used by
\citet{Boily2015} is contextual, and							%what is meant by contextual?
includes four high-risk groups which transition to specific low-risk groups
(e.g. from ``high-/low-volume sex work'' to ``formerly engaged in sex work'').
The turnover matrix $\phi$ is then completely \emph{specified} (though notably sparse)
using several assumptions about \emph{group durations}.
The distribution of risk groups among individuals entering the model $\bm{\hat{e}}$
is also \emph{specified},
leaving the distribution of risk groups among individuals in the model $\bm{\hat{x}}$
as the only free parameters.
For this reason, a 100-year ``burn-in'' period is required
to equilibrate risk-group sizes before introduction of HIV in the model.
Note that this approach is not perfectly compatible with the framework presented here,			%not clear what this means
since we assume \emph{constant group sizes}
as the basis for the system in Eq.~(\ref{eq:b-A-basis}),
and exclude $\bm{\hat{x}}$ from the vector of free parameters $\bm{\theta}$.
However, by relaxing these constraints,
it should be possible to formulate even the implementation by \citet{Boily2015}					%'should be' is unclear and vague. Also is the goal to formulate the implementation of other approaches, or is it to reproduce the goals of the other approaches -- for each paper, what was the goal? link to the constraints you outlined above. is it to keep group sizes constant or duration in risk constant based on data? 
in terms of the proposed framework.													%but what about HIV-attributable mortality? I think across all models discussed above except for Stigum, need to discuss inclusion of HIV-attributable mortality
\par
In sum, the framework for modelling turnover presented here aims to generalize				%where is 'here'? the paragraph above?
all previous implementations.
