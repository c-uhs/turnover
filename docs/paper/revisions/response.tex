\newcommand{\comment}[1]{\par\textbf{#1}\par}
\newcommand{\reply}[1]{\par{#1}\par}
\renewcommand{\quote}[1]{\par\emph{#1}\par}
\newcommand{\added}[1]{\textcolor{green!70!gray}{#1}}
% ==================================================================================================
\textbf{Reviewer \#1}
% -------------------------------------------------------------------------------------------------- %SM: show all of the text from the reviewer (not just the suggestion) - makes it easier for the editor to read and decide if the response is sufficient.
\comment{%
  This is a well design study which used turnover between risk groups to do modelling for projecting the transmission population attributable fraction.
  There included three phenomena that drove shifts in equilibrium sexually transmitted infections (STIs) prevalence across risk groups at variable rates of turnover:
  1) net flows of infectious individuals from high risk groups into low risk groups;
  2) changes to herd immunity, especially within the high risk group; and
  3) changes to the number of partnerships available with infectious individuals.
  My suggestions are the following:}
% --------------------------------------------------------------------------------------------------
%\comment{%
  Due to have no vaccination for STIs, or persons are not immune to STIs after infected with sexually transmitted diseases,
  it should consider that STIs have no herd immunity.}

\reply{%
  We agree that the choice of terminology, herd ``immunity'' is confusing,
  especially since we have simulated a ``treated'', rather than ``immune'' group as the final health state.
  The ``treated'' group are mechanistically considered an ``immune'' group, in that, following 
  treatment - individuals are no longer susceptible to infection or re-infection. The ``treated''  group therefore, contribute to herd effects as they would in a 
  classic S-I-R model (susceptble-infectious-recovered) since the R in this case is mechanistically equivalent to the treated compartment.
  Thus, for clarity, we have replaced the phrase ``herd immunity'' with ``\added{herd effect}'' throughout the paper.
  
  % JK: @SM, I forgot this important detail in my rough thoughts in the email. %SM: exactly! no we are thinking the same thing - the model is an SIR model (hence, herd effects from R - or in this case, treated). Take out the rest of the explanation as not needed. The only thing to clarify here was using term 'herd effects'.
  %     The treated group really is fundamental to the trends we have observed here
  %     (for better or for worse!)
  To clarify this point, we have also added to the discussion:
  \quote{%
    First, we did not capture the possibility that some individuals may become
    re-susceptible to infection after treatment
    -- an important feature of many STIs such as syphilis and gonorrhoea [Fenton2008].
    As shown by [Fenton2008] and [Pourbohloul2003],
    the re-supply of susceptible individuals following STI treatment
    could fuel an epidemic, and so the influence of turnover on
    STI prevalence and tPAF may be different.
    \added{In fact, the herd effect underpinning phenomenon~2
      relies on the existence of a treated/immune health state,
      since only net replacement of treated (versus infectious) individuals
      with susceptible individuals via turnover can increase prevalence
      in the high risk group as observed.}}
  \par
%  We chose to simulate a ``treated'' health state after this study grew out of HIV modelling work,
%  but we did not call the simulated infection ``HIV'' since we did not model infection-attributable mortality. %SM: No would not include. Reviewer did not ask for this. Would keep responses really focused.
  % JK: @SM, not sure if you feel this is relevant, but I feel it helps contextualize our decisions.
  Although we specifically examine the question of turnover in the context of an STI, 
  the findings are relevant for a variety of non-STI pathogens
  where risk heterogeneity is present, many of which may have natural or treatment-induced immunity. 
  We have added the following to the Discussion section to help clarify the transferability of insights to biological 
  systems in general which follow an S-I-R natural or treatment-mediated biological course.
  % JK: @SM, do you think this comment about useful insights beyong STI pathogens is too strong?
  %     I really do think this is true, and would even consider adding it to the paper
  %     (last sentence in the limitations section). What do you think?  %SM: agree with including.
}
% --------------------------------------------------------------------------------------------------
\comment{%
  The most effective prevention is a substantial reduction of sexual partners and/or expanded condom use for STI.
  This should be considered in the manuscript.}
\reply{%
  In the discussion, we have added the following to provide further context with respect to interventions, and cited the 2016 WHO guidelines for HIV prevention and 
  treatment interventions for key populations.
  \quote{%
    For example, epidemic models which fail to include or accurately capture
    turnover may underestimate the importance of addressing the unmet
    needs of key populations at disproportionate risk of HIV and other STIs, such as
    gay men and other men who have sex with men, transgender women, people who use drugs, and sex workers.
    \added{That is, epidemic models without turnover may underestimate the potential impact 
    of prioritizing, allocating, and tailoring interventions for key populations: interventions such as
    enhanced and prioritized screening and testing, early diagnoses and treatment,
    condom use programmes, and reductions in structural barriers to safer sex.} %SM: include reference here for interventions https://www.who.int/hiv/pub/guidelines/keypopulations-2016/en/
  }
  % JK: @SM, 100% agree with your comment in the email that we should not recommend specific interventions
  %     as we did not explore, and stay conscientious of the specific language/framing involved.
}
% ==================================================================================================
\textbf{Reviewer \#2}\par
% --------------------------------------------------------------------------------------------------
%\comment{%
%  The authors explored the role of turnover in evaluating the transmission
%  dynamics of infectious diseases, especial STI. A multi-group model
%  incorporate with data-guided framework is used to exam this effect. The
%  authors numerically show the importance of considering turnover to evaluate
%  the disease transmission in different risk groups via three study cases. This
%  topic is interesting. I have the following comments:}
% --------------------------------------------------------------------------------------------------
\comment{%
  1) How is the turnover effect on the medium risk group? As it may have
  various effects on this group which usually has the largest population
  and could impact the overall disease pattern.}
\reply{%
  
  First, the largest group .... %SM: need to address that part of the comment more explicitly in the response (cannot ignore it).
  We have thus clarified that the relative size of the groups are as follows: ..... % need to say this in the response.
 
  Second, our rationale for focusing on the highest-risk group was guided by the framing of 
  research questions around key populations who are often represented in HIV/STI 
  transmission modelling as the highest risk group, and on whom tPAF are now increasingly being measured. 
  The highest-risk group therefore 
  represents a well-demarcated group (epidemiologically and with respect to programmes) 
  that can be identified for population-specific programming, and on whom tPAF are now
  increasingly measured. We therefore focused the mechanistic examination (Experiment 1)
  on the highest risk group as the explanation for the main research question on the influence of 
  highest-risk group turnover on tPAF of the highest-risk group.

  We agree with the reviewer that it will be important to further study the mechanisms of influence of turnover
  in the middle-risk group on the tPAF of the medium risk group. However, we feel that 
  such an examination is beyond the scope of the 
  current study answers would require a new research question and 
  specific experiments wherein the focus is on turnover in
  the medium-risk group (vs. our current experiments which are grounded in turnover in the highest-risk group) - but examination which would benefit from future study.

  In the meantime, we have added the following in the revised manuscript to address the comment and suggestion:
  \par
   Results \S~3.1 section:
  \quote{%
    To explain the inverted U-shape and different turnover thresholds by group,
    we examined the processes contributing to prevalence,
    first in the high risk group, and then in the low risk group.
    \added{Prevalence in the medium risk group is then driven by a combination of these processes.}}  %SM: try to avoid using contractions in scientific writing ("we've" should be "we have", etc.)
  We also added to Figure 5 the three panels for the medium risk group,
  and the following sentence to the end of \S~3.1.2:
  \quote{%
    \added{The effects of these phenomena on the medium risk group were
    a mixture of the effects on the high and low risk groups,
    as illustrated in Figure~5d--5f,
    leading to the unique profile shown in Figure~4b.}}
 
  % JK: @SM, you mentioned in the email to perhaps also mention that tPAF of the medium risk group
  %     might be similarly underestimated without turnover.  %SM: not similarly but rather what we think the effect might be.... just as you laid out below?
  %     I'm not confident enough in that conclusion to add to the paper for 2 reasons:
  %     1) the prevalence ratios involving the medium risk group were not strictly monotonic (Figure C.2 b,c);
  %     2) while the contribution of the medium risk group to onwrad transmission would likely
  %     increase relative to the low risk group, the contribution would likely decrease relative to the high risk group;
  %     thus we cannot reliably conclude that the medium risk group tPAF would increase or decrease, overall
  %     in the presence of turnover. Does that make sense?
  
  % SM: what about a tPAF figure for medium risk group just to show that it is a mixture of effects, and not consistently over or under-estimated? Could do this since it is not a new set of experiments
   %       per se, but an output that can be put in appendix, and given one line in results and point to appendix? 
}
% --------------------------------------------------------------------------------------------------
\comment{%
  2) Please illustrate the Figure 5 in more details, i.e. what are the
  different scenarios for panels in each column. Formulas for the curves
  may help readers to understand more.}
\reply{%
  To clarify what is being plotted in Figure 5,   %SM: lets leave out the closed form / analytic solution part. if that is what the reviewer meant, then editor (and/or reviewer) can write again and we can respond.
  we have revised the methodology for Experiment 1 (\S~2.3.1)
  to update the text and to include formulae for the plotted quantities, new Eq.~(5).
  To further improve the ease of understanding Figure 5,
  we also made the following changes:
  \begin{itemize}[itemsep=0pt,topsep=0pt]
    \item added panel row and column headings at the left and top, respectively
    \item removed the legend from each panel and replaced it with a single legend
          at the top of the entire figure
    \item added a red shaded area below zero labelled ``net loss''
          and labelled the non-shaded area above zero ``net gain''
  \end{itemize}
  We hope that these changes address the limitations noted by the reviewer.
  % JK: @SM, I think you also mentioned a scatter plot instead of a line graph,
  %     I would prefer to stick with the line graph,
  %     since I think it's reasonable to infer that the relationship is continuous,
  %     even though we obtain it from discrete sampling of the (intractable) solution. %SM: sounds good.
}