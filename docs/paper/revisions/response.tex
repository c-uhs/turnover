\newcommand{\comment}[1]{\par\textbf{#1}\par}
\newcommand{\reply}[1]{\par{#1}\par}
\renewcommand{\quote}[1]{\begin{quotation}\noindent\emph{#1}\end{quotation}}
\newcommand{\added}[1]{\textcolor{green!70!gray}{#1}}
% ==================================================================================================
\textbf{Reviewer \#1}
% --------------------------------------------------------------------------------------------------
\comment{%
  This is a well design study which used turnover between risk groups to do modelling for projecting the transmission population attributable fraction.
  There included three phenomena that drove shifts in equilibrium sexually transmitted infections (STIs) prevalence across risk groups at variable rates of turnover:
  1) net flows of infectious individuals from high risk groups into low risk groups;
  2) changes to herd immunity, especially within the high risk group; and
  3) changes to the number of partnerships available with infectious individuals.
  My suggestions are the following:}
% --------------------------------------------------------------------------------------------------
\comment{%
  Due to have no vaccination for STIs, or persons are not immune to STIs after infected with sexually transmitted diseases,
  it should consider that STIs have no herd immunity.}
\reply{%
  We agree that the choice of terminology, herd ``immunity'' is confusing,
  especially since we have simulated a ``treated'', rather than ``immune'' group as the final health state.
  The ``treated'' group are mechanistically considered an ``immune'' group.
  That is, once individuals in the infectious state receive treatment,
  they are no longer susceptible to infection or re-infection.
  The ``treated''  group therefore, contribute to herd effects as they would in a 
  classic S-I-R model (susceptible-infectious-recovered) since the R in this case
  is mechanistically equivalent to the treated compartment.
  Thus, for clarity, we have replaced the phrase ``herd immunity'' with ``\added{herd effect}'' throughout the paper.
  To clarify this point, we have also added to the discussion:
  \quote{%
    First, we did not capture the possibility that some individuals may become
    re-susceptible to infection after treatment
    -- an important feature of many STIs such as syphilis and gonorrhoea [Fenton2008].
    As shown by [Fenton2008] and [Pourbohloul2003],
    the re-supply of susceptible individuals following STI treatment
    could fuel an epidemic, and so the influence of turnover on
    STI prevalence and tPAF may be different.
    \added{In fact, the herd effect underpinning phenomenon~2
      relies on the existence of a treated/immune health state,
      since only net replacement of treated (versus infectious) individuals
      with susceptible individuals via turnover can increase prevalence
      in the high risk group as observed.}}
  \par
  Although we specifically examine the question of turnover in the context of an STI,
  our findings may also be useful for non-STI pathogens
  which follow an S-I-R natural or treatment-mediated biological course.
  We have added the following to the limitations section to help clarify the transferability of insights:
  \quote{%
    \added{Moreover, our findings are likely relevant to non-STI pathogens when risk heterogeneity is present,
      many of which have natural or treatment-induced immunity.}}
}
% --------------------------------------------------------------------------------------------------
\comment{%
  The most effective prevention is a substantial reduction of sexual partners and/or expanded condom use for STI.
  This should be considered in the manuscript.}
\reply{%
  In the discussion, we have added the following to provide further context with respect to interventions,
  and cited the 2016 WHO guidelines for HIV prevention and treatment interventions for key populations.
  \quote{%
    For example, epidemic models which fail to include or accurately capture
    turnover may underestimate the importance of addressing the unmet
    needs of key populations at disproportionate risk of HIV and other STIs, such as
    gay men and other men who have sex with men, transgender women, people who use drugs, and sex workers.
    \added{That is, epidemic models without turnover may underestimate the potential impact 
      of prioritizing, allocating, and tailoring interventions for key populations: interventions such as
      enhanced and prioritized screening and testing, early diagnoses and treatment,
      condom use programmes, and reductions in structural barriers to safer sex [WHO2016].}}
}
\clearpage
% ==================================================================================================
\textbf{Reviewer \#2}\par
% --------------------------------------------------------------------------------------------------
\comment{%
  The authors explored the role of turnover in evaluating the transmission
  dynamics of infectious diseases, especial STI. A multi-group model
  incorporate with data-guided framework is used to exam this effect. The
  authors numerically show the importance of considering turnover to evaluate
  the disease transmission in different risk groups via three study cases. This
  topic is interesting. I have the following comments:}
% --------------------------------------------------------------------------------------------------
\reply{%
  We thank the reviewer for the positive feedback.
}
\comment{%
  1) How is the turnover effect on the medium risk group? As it may have
  various effects on this group which usually has the largest population
  and could impact the overall disease pattern.}
\reply{%
  We address and consider the comment using three points of clarification.
  First, we have clarified that in our analyses, the largest risk group was the low-risk group.
  We have thus clarified in two locations the relative size of the groups:
  in Methods 2.2:
  \quote{high~$H$ \added{(smallest)}, medium~$M$, and low~$L$ \added{(largest)}.}
  and in Results 3.1:
  \quote{first in the \added{smallest} high risk group, and then in the \added{largest} low risk group.}
  \par
  Second, our rationale for focusing on the highest-risk group was guided by
  the framing of research questions around key populations who are often represented in
  HIV/STI transmission modelling as the highest risk group,
  and for whom tPAF are now increasingly being measured.
  The highest-risk group therefore represents a well-demarcated group
  (epidemiologically and with respect to programmes) 
  that can be identified for population-specific programming,
  and on whom tPAF are now increasingly measured.
  We therefore focused the mechanistic examination (Experiment 1)
  on the highest risk group as the explanation for the main research question on
  the influence of turnover on tPAF of the highest-risk group.
  \par
  We agree with the reviewer that it will be important to further study
  how turnover mechanistically influences the prevalence and tPAF of the medium risk group.
  However, we feel that such an examination is beyond the scope of the current study, because
  answers would require a new research question and specific experiments wherein the focus is
  on the influence of turnover in the medium-risk group.  
  For example, the prevalence ratios involving the medium risk group were not monotonic,
  so the potential influence of turnover on the tPAF of the medium risk group would
  depend on the rate of turnover. Similarly, while
  the contribution of the medium risk group to onward transmission could increase relative to the low risk group,
  the contribution could decrease relative to the high risk group.
  \par
  Thus we have included a recommendation for further study in the limitations section:
  \quote{%
    \added{Second, we mainly examined how turnover influences STI prevalence and tPAF of the highest risk group.
    Future work could explore the mechanisms underlying the influence of turnover on
    STI prevalence and tPAF of other risk groups.}}
  and added the following in the revised manuscript to address the comment.
  Results 3.1:
  \quote{%
    To explain the inverted U-shape and different turnover thresholds by group,
    we examined the processes contributing to prevalence,
    first in the high risk group, and then in the low risk group.
    \added{Prevalence in the medium risk group is then driven by a combination of these processes.}}
  We also added to Figure 5 the three panels for the medium risk group,
  and the following sentence to the end of Results 3.1.2:
  \quote{%
    \added{The effects of these phenomena on the medium risk group were
    a mixture of the effects on the high and low risk groups,
    as illustrated in Figure~5d--5f,
    leading to the unique profile shown in Figure~4b.}}
}
% --------------------------------------------------------------------------------------------------
\comment{%
  2) Please illustrate the Figure 5 in more details, i.e. what are the
  different scenarios for panels in each column. Formulas for the curves
  may help readers to understand more.}
\reply{%
  To clarify what is being plotted in Figure 5,
  we have revised the methodology for Experiment 1 (Methods 2.3.1)
  to update the text and to include formulae for the plotted quantities: new Eq.~(5).
  To further improve the ease of understanding Figure 5,
  we also made the following changes:
  \begin{itemize}[itemsep=0pt,topsep=0pt]
    \item added panel row and column headings at the left and top, respectively
    \item removed the legend from each panel and replaced it with a single legend
          at the top of the entire figure
    \item added a red shaded area below zero labelled ``net loss''
          and labelled the non-shaded area above zero ``net gain''
  \end{itemize}
  We hope that these changes address the limitations noted by the reviewer.
}