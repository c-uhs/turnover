\subsection{Experiment~1: Mechanisms by which turnover influences equilibrium prevalence}
\label{ss:res-prevalence}
Figure~\ref{fig:prevalence} shows the trends in equilibrium STI prevalence  			
among the high, medium, low, risk groups, at different rates of turnover which 		%SM: suggest refer to Figure panels in text ".... the high (Figure 4a), medium (Figure 4b), ... etc. 
are depicted on the x-axis as
duration of time spent in the high risk group.
First, Figure~\ref{fig:prevalence} reveals an inverted U-shaped relationship
between STI prevalence and turnover in all three risk groups.
That is, equilibrium STI prevalence is higher in systems with slow turnover 
versus those with no turnover 
(region~A in Figure~\ref{fig:prevalence} ). Equilibrium STI prevalence then peaked						%SM: when talking about Region in the text, must refer to the figure.
at slightly faster turnover before declining in systems with even faster turnover 
(region~B in Figure~\ref{fig:prevalence} ).
Second, comparison of group-specific prevalence in Figure~\ref{fig:prevalence} shows that
the threshold turnover rate at which group-specific prevalence peaked
varied by risk group.
Prevalence in the high risk group peaked at the lower turnover threshold
(Figure~\ref{fig:prevalence-high}),
while prevalence in low risk group peaked at a higher turnover threshold								
(Figure~\ref{fig:prevalence-low}).

To explain the inverted U-shape and different turnover thresholds by group,							%SM: good
we examined the components contributing to prevalence,
first in the high risk group, and then in the low risk group.
\begin{figure}
\begin{minipage}[t]{0.475\linewidth}
  \centering
  \begin{subfigure}{0.85\linewidth}
    \centering
    \includegraphics[width=\linewidth]{{1d-prevalence-high-tau=0.1}.pdf}
    \caption{High risk}
    \label{fig:prevalence-high}
  \end{subfigure}\\
  \begin{subfigure}{0.85\linewidth}
    \centering
    \includegraphics[width=\linewidth]{{1d-prevalence-med-tau=0.1}.pdf}
    \caption{Medium risk}
    \label{fig:prevalence-med}
  \end{subfigure}\\
  \begin{subfigure}{0.85\linewidth}
    \centering
    \includegraphics[width=\linewidth]{{1d-prevalence-low-tau=0.1}.pdf}
    \caption{Low risk}
    \label{fig:prevalence-low}
  \end{subfigure}
  \caption{Relationship between equilibrium STI prevalence
    in high, medium, and low risk groups versus turnover rate.
    Regions A~and~B denote where equilibrium prevalence is increasing and decreasing
    with different rates of turnover, respectively.
    See Figure~\ref{fig:dur-group} for x-axis definition.}
  \label{fig:prevalence}
\end{minipage}\hspace{0.05\linewidth}%
\begin{minipage}[t]{0.475\linewidth}
  \centering
  \begin{subfigure}{0.85\linewidth}
    \centering
    \includegraphics[width=\linewidth]{dX-abs-basic-high-I.pdf}
    \caption{High risk}
    \label{fig:dX-I-high}
  \end{subfigure}\\
  \begin{subfigure}{0.85\linewidth}
    \centering
    \includegraphics[width=\linewidth]{dX-abs-basic-med-I.pdf}
    \caption{Medium risk}
    \label{fig:dX-I-med}
  \end{subfigure}\\
  \begin{subfigure}{0.85\linewidth}
    \centering
    \includegraphics[width=\linewidth]{dX-abs-basic-low-I.pdf}
    \caption{Low risk}
    \label{fig:dX-I-low}
  \end{subfigure}\\
  \caption{Absolute rates of change at equilibrium
    (number of individuals gained/lost per year)
    among infectious individuals in each risk group,
    broken down by type of change:
    loss/gain via turnover, % ($+ \sum_j \phi_{ji} \mathcal{I}_j - \sum_j \phi_{ij} \mathcal{I}_i$),
    gain via incident infections, % ($+ \lambda_i \mathcal{S}_i$),
    loss via treatment, % ($- \tau \mathcal{I}_i$),
    loss via death, % ($- \mu \mathcal{I}_i$),
    and net change.
    Based on Eq.~(\ref{eq:model-I}).
    Rates of change do not sum to zero due to population growth.
    See Figure~\ref{fig:dur-group} for x-axis definition.}
  \label{fig:dX-I}
\end{minipage}
\end{figure}
\begin{figure}[!tbp]
  \centering
  \includegraphics[width=0.6\linewidth]{flows-legend.pdf}\\
  \includegraphics[width=0.11\linewidth]{flows-labels.pdf}
  \begin{subfigure}[t]{0.22\linewidth}
    \centering
    \includegraphics[width=\linewidth]{flows-none.pdf}
    \caption{No turnover}
    \footnotesize $\delta_H^{-1} = \input{\datapath/flows/none/phi.tex}$
    \label{fig:flows-none}
  \end{subfigure}%
  \begin{subfigure}[t]{0.22\linewidth}
    \centering
    \includegraphics[width=\linewidth]{flows-low.pdf}
    \caption{Slow turnover}
    \footnotesize $\delta_H^{-1} = \input{\datapath/flows/low/phi.tex}$
    \label{fig:flows-low}
  \end{subfigure}%
  \begin{subfigure}[t]{0.22\linewidth}
    \centering
    \includegraphics[width=\linewidth]{flows-high.pdf}
    \caption{Fast turnover}
    \footnotesize $\delta_H^{-1} = \input{\datapath/flows/high/phi.tex}$
    \label{fig:flows-high}
  \end{subfigure}%
  \begin{subfigure}[t]{0.22\linewidth}
    \centering
    \includegraphics[width=\linewidth]{flows-extr.pdf}
    \caption{Very fast turnover}
    \footnotesize $\delta_H^{-1} = \input{\datapath/flows/extr/phi.tex}$
    \label{fig:flows-extr}
  \end{subfigure}\\[0.5em]
  $\null\hspace{0.11\linewidth}
  \underbrace{\hspace{0.44\linewidth}}_{\textrm{Region~A}}
  \underbrace{\hspace{0.44\linewidth}}_{\textrm{Region~B}}$
  \caption{Depiction of health states of individuals in each risk group		
    and of individuals moving between risk groups,
    obtained from models at equilibrium
    under four overall rates of turnover.
    Circle sizes are proportional to risk group sizes.
    Circle slices and arrow widths are also proportional to
    the proportion of health states within risk groups and
    among individuals moving between risk groups, respectively.
    However, circle sizes and arrow widths do not have comparable scales.}	%SM: tell reader that Appendix Figure 7 provides the numerical support for the proportions, etc.
  \label{fig:flows}
\end{figure}

\par
Figure~\ref{fig:dX-I} shows the four components which contributed to 
gain/loss of infectious individuals in each risk group:
1)~net gain/loss via turnover,
2)~gain via incident infections,
3)~loss via treatment, and
4)~loss via death,
for each risk group in the model,
at equilibrium under different rates of turnover.
Figure~\ref{fig:flows} also illustrates
the distribution of health states in each risk group
and among individuals moving between risk groups
under four different rates of turnover.
% --------------------------------------------------------------------------------------------------
\subsubsection{Influence of turnover on equilibrium prevalence in the high risk group}		%SM: becuase there are so many figures, i think will have to refer to the figure in almost every sentence. Before sending me the penulitmate version, please read out loud and check if any sentences seem unclear as to which figure they are referring to, etc.
\label{sss:res-prev-high}
As shown in Figure~\ref{fig:flows} at all four rates of turnover,
the proportion of individuals 
who are infectious state is largest in the high risk group. 
As individuals in the infectious state left
the high risk group via turnover, they were largely 
replaced by susceptible individuals from lower risk groups (Figure 7b). 		%SM: refer here to Figure 7b in Appendix so that the "numbers" can be corroborated.
For example, in the context of slow turnover (Figure 6b, Figure 7b), 			%SM for each statement --> give clear example with numbers to support the statement.
X percent of individuals in the high risk group were in the infectious state,
and the absolute rate of change of the infectious state 
attributable to turnover was -XX (Figure 7b) while the
absolute rate of change in the susceptible state attributable to 
turnover was +XX (Figure 7a). The pattern of net loss of 
infectious individuals in the high risk group via turnover
persisted across the range of turnover rates (Figure~\ref{fig:dX-I-high}, yellow).			%SM: and Figure 7b?

%%%%%% SM below to review

which acted to decrease prevalence in the group.
However, turnover similarly caused a net replacement of
treated individuals with susceptible individuals in the high risk group
(Figure~\ref{fig:flows}).
Loss of treated individuals acted to reduce herd immunity,
thereby increasing the rate of incident infections,
which we observed for slow rates of turnover versus no turnover
(Figure~\ref{fig:dX-I-high}, red).
Under slow rates of turnover, increasing turnover
added more infections to the high risk group via incidence
than were lost via turnover
(Figure~\ref{fig:dX-I-high});
thus equilibrium prevalence in the high risk group
increased with turnover in region~A
(Figure~\ref{fig:prevalence-high}).
\par
% JK: In order to help distinguish between high/low risk groups and high/low turnover,
%     I replaced "high" turnover with "fast", and "low" with "slow"
%     What do you think?
Under faster rates of turnover,
the rate of incident infections in the high risk group did not keep pace with
the loss of infectious individuals via turnover
(Figure~\ref{fig:dX-I-high});
thus prevalence declined with increasing turnover
(Figure~\ref{fig:prevalence-high}, region~B).
Two phenomena contributed to decreasing rate of incident infections
in the high risk group under faster rates of turnover.
First, as herd immunity in the high risk group was reduced via turnover,
the additional infections attributable to
further reducing this barrier to transmission decreased
(diminishing returns).
Second, the net movement of infectious individuals
from high to low risk (Figure~\ref{fig:flows}) reduced
the average number of partners per year among infectious individuals
(Figure~\ref{fig:C-I}).
As shown in Appendix~\ref{aa:eqs-incidence},
under our modelling assumptions,
incidence in all risk groups was proportional to
the average number of partners per year among infectious individuals
(Figure~\ref{fig:C-I}),
and overall prevalence
(Figure~\ref{fig:prevalence-all}).
Thus, as the average number of partners per year among infectious individuals
decreased with faster turnover, incidence also decreased
(Figure~\ref{fig:incidence-all}).
Moreover, as equilibrium incidence decreased across all groups
with faster turnover for the same reason,
equilibrium prevalence then decreased, which in turn reduced incidence,
and so on in a mutually reinforcing exponential decline
(Figure~\ref{fig:prevalence}, region~B).
\begin{figure}
  \centering
  \begin{subfigure}[t]{0.32\linewidth}
    \centering
    \includegraphics[width=\linewidth]{{1d-C-I-tau=0.1}.pdf}
    \caption{Average number of partners per year among infectious individuals}
    \label{fig:C-I}
  \end{subfigure}%
  \begin{subfigure}[t]{0.32\linewidth}
    \centering
    \includegraphics[width=\linewidth]{{1d-prevalence-all-tau=0.1}.pdf}
    \caption{Overall prevalence}
    \label{fig:prevalence-all}
  \end{subfigure}%
  \begin{subfigure}[t]{0.32\linewidth}
    \centering
    \includegraphics[width=\linewidth]{{1d-incidence-all-tau=0.1}.pdf}
    \caption{Overall incidence}
    \label{fig:incidence-all}
  \end{subfigure}%
  \caption{Overall incidence and the non-constant factors of incidence versus turnover.
    The product of factors (\subref{fig:C-I}) and (\subref{fig:prevalence-all})
    is proportional to (\subref{fig:incidence-all}) overall incidence.
    See Appendix~\ref{aa:eqs-incidence} for proof.
    See Figure~\ref{fig:dur-group} for x-axis definition.}
  \label{fig:incidence-factors}
\end{figure}
% --------------------------------------------------------------------------------------------------
\subsubsection{Influence of turnover on equilibrium prevalence in the low risk group}
\label{sss:res-prev-low}
As shown in Figure~\ref{fig:flows}, the low risk group was
composed mainly of susceptible individuals at equilibrium,
so increasing turnover yielded
a net replacement of susceptible individuals
with infectious and treated individuals from higher risk groups.
The net influx of infectious individuals acted to
increase equilibrium prevalence in the low risk group
(Figure~\ref{fig:dX-I-low}, yellow).
The net influx of treated individuals
contributed to a small increase in herd immunity,
but the herd immunity effect in the low risk group
remained negligible due to
the small proportion of treated relative to susceptible individuals.
With slow versus no turnover,
incident infections also increased in the low risk group
(Figure~\ref{fig:dX-I-low}, red),
driven by increasing overall incidence
(Figure~\ref{fig:incidence-all}),
ultimately attributable to reduced herd immunity in the high risk group
(see Section~\ref{sss:res-prev-high}).
The net gain of infectious individuals via turnover and increasing incident infections
then both increased equilibrium prevalence in the low risk group
under slow versus no turnover
(Figure~\ref{fig:prevalence-low}, region~A).
\par
Faster rates of turnover still contributed a net gain of infectious individuals
via movement of already infected individuals.
However, as in the high risk group,
the rate of incident infections decreased
in the low risk group under faster rates of turnover
(Figure~\ref{fig:dX-I-low}, red),
due to decreasing overall incidence
(Figure~\ref{fig:incidence-all})
via decreasing number of partners per year among infectious individuals
(Figure~\ref{fig:C-I}).
As a result, equilibrium prevalence decreased in the low risk group as turnover increased
(Figure~\ref{fig:prevalence-low}, region~B),
following the same exponential decline as in the high risk group
under very fast rates of turnover.
% --------------------------------------------------------------------------------------------------
\subsubsection{Prevalence ratios and different turnover rates for peak prevalence among risk groups}
% JK: not really sure if this is the best title for this section.
The movement of individuals between risk groups via turnover yielded
a net loss of infectious individuals from the high risk group, but
a net gain of infectious individuals in the low risk group
(Figure~\ref{fig:dX-I}, yellow).
By comparison, the influence of turnover on
the rate of incident infections was relatively similar
among high versus low risk groups (Figure~\ref{fig:dX-I}, red).
Thus, the net effect of increasing turnover in the high risk group
\emph{tended towards} decreasing prevalence,
whereas in the low risk group,
the net effect \emph{tended towards} increasing prevalence.
Since these two tendencies act to bring prevalence in the two groups
closer together, we describe the overall tendency as
the ``homogenizing effect'' of turnover on group-specific prevalence.%
\footnote{The level of risk heterogeneity experienced by individuals in the model
  also decreases with faster turnover;
  this decrease in risk heterogeneity
  could also be described as a ``risk homogenizing effect''.}
% JK: I remember we had agreed on ``reducing risk heterogeneity'' versus ``risk homogenizing''
%     
The homogenizing effect is demonstrated in Figure~\ref{fig:ratio-prevalence},
which shows the equilibrium prevalence ratios
in pair-wise comparisons of all three risk groups.
The prevalence ratio between highest and lowest risk groups
monotonically decreased with turnover
(Figure~\ref{fig:ratio-prevalence-high-low}).
% JK: I don't talk about prevalence ratios between high/med or med/low.
%     They're difficult to explain and I'm not sure it add anything,
%     especially considering we haven't discussed the medium risk group
%     at all in this section...
The homogenizing effect is also why, in the high risk group,
lower rates of turnover were required
to have a net decreasing effect on prevalence,
as compared to in the low risk group.
That is, the threshold rate of turnover at which
prevalence in the high risk group peaked and then declined
was lower than in the low risk group
(Figure~\ref{fig:prevalence}).
\begin{figure}
  \centering
  \begin{subfigure}{0.31\linewidth}
    \centering\includegraphics[width=\linewidth]{{1d-ratio-prevalence-high-low-tau=0.1}.pdf}
    \caption{High vs Low risk}
    \label{fig:ratio-prevalence-high-low}
  \end{subfigure}
  \begin{subfigure}{0.31\linewidth}
    \centering\includegraphics[width=\linewidth]{{1d-ratio-prevalence-high-med-tau=0.1}.pdf}
    \caption{High vs Medium Risk}
    \label{fig:ratio-prevalence-high-med}
  \end{subfigure}
  \begin{subfigure}{0.31\linewidth}
    \centering\includegraphics[width=\linewidth]{{1d-ratio-prevalence-med-low-tau=0.1}.pdf}
    \caption{Medium vs Low Risk}
    \label{fig:ratio-prevalence-med-low}
  \end{subfigure}
  \caption{Equilibrium prevalence ratios between risk groups
    under different rates of turnover.
    See Figure~\ref{fig:dur-group} for x-axis definition.}
  \label{fig:ratio-prevalence}
\end{figure}
% ==================================================================================================
\subsection{Experiment~2: Inferred risk heterogeneity with vs without turnover}
\label{ss:res-infer}
Next, we compared the fitted parameters in models with versus without turnover
($\delta_H = 5$ vs $\delta_H = 33$ years).
Before model fitting, the predicted prevalence ratio
between high and low risk groups was lower with turnover versus without:
$\input{\datapath/values/turnover-prevalence-ratio-high-low.txt}$~vs~%
$\input{\datapath/values/no-turnover-prevalence-ratio-high-low.txt}$,
reflecting the homogenizing effect of turnover on group-specific prevalence
(Figure~\ref{fig:ratio-prevalence-high-low}).
Thus, when fitting the model to target prevalence values,
the fitted numbers of partners per year $C$ would have to compensate for
this difference in prevalence ratio with versus without turnover.
\par
After fitting the number of partners per year, both models predicted
the target equilibrium infection prevalence values of 20\%,~8.75\%,~3\%,~and~5\%
among the high, medium, low risk groups, and overall
(Figure~\ref{fig:tpaf-prevalence}).
However, in order to do so, the ratio of fitted partners
between high and low risk groups ($C_H~/~C_L$)
was higher with turnover than without:
$\input{\datapath/values/turnover-[fit]-C-ratio-high-low.txt}$~vs~%
$\input{\datapath/values/no-turnover-[fit]-C-ratio-high-low.txt}$
(Table~\ref{tab:fitting}).
That is, the inferred level of risk heterogeneity was higher
in the model with turnover than in the model without turnover.
In order to observe the same prevalence ratio in a system with turnover,
the homogenizing effect of turnover must be overcome by
greater heterogeneity in risk, as compared to a system without turnover.
\begin{table}
  \centering
  \caption{Equilibrium partnership formation rates and prevalence
    among the high and low risk groups
    predicted by the models with and without turnover,
    before and after model fitting.}
  \label{tab:fitting}
  \begin{tabular}{rcccccc}
	\toprule
  & \multicolumn{3}{c}{Contact Rate} & \multicolumn{3}{c}{Prevalence} \\
  \cmidrule(lr){2-4}\cmidrule(lr){5-7}
  Context & High & Low & High/Low & High & Low & High/Low \\\midrule
  No Turnover &
  \input{\datapath/values/no-turnover-C-high.txt}
  & \input{\datapath/values/no-turnover-C-low.txt}
  & \input{\datapath/values/no-turnover-C-ratio-high-low.txt}
  & \input{\datapath/values/no-turnover-prevalence-high.txt}
  & \input{\datapath/values/no-turnover-prevalence-low.txt}
  & \textbf{\input{\datapath/values/no-turnover-prevalence-ratio-high-low.txt}}\\
  Turnover &
    \input{\datapath/values/turnover-C-high.txt}
  & \input{\datapath/values/turnover-C-low.txt}
  & \input{\datapath/values/turnover-C-ratio-high-low.txt}
  & \input{\datapath/values/turnover-prevalence-high.txt}
  & \input{\datapath/values/turnover-prevalence-low.txt}
  & \textbf{\input{\datapath/values/turnover-prevalence-ratio-high-low.txt}}\\
  No Turnover [fit] &
    \input{\datapath/values/no-turnover-[fit]-C-high.txt}
  & \input{\datapath/values/no-turnover-[fit]-C-low.txt}
  & \textbf{\input{\datapath/values/no-turnover-[fit]-C-ratio-high-low.txt}}
  & \input{\datapath/values/no-turnover-[fit]-prevalence-high.txt}
  & \input{\datapath/values/no-turnover-[fit]-prevalence-low.txt}
  & \input{\datapath/values/no-turnover-[fit]-prevalence-ratio-high-low.txt}\\
  Turnover [fit] &
  \input{\datapath/values/turnover-[fit]-C-high.txt}
  & \input{\datapath/values/turnover-[fit]-C-low.txt}
  & \textbf{\input{\datapath/values/turnover-[fit]-C-ratio-high-low.txt}}
  & \input{\datapath/values/turnover-[fit]-prevalence-high.txt}
  & \input{\datapath/values/turnover-[fit]-prevalence-low.txt}
  & \input{\datapath/values/turnover-[fit]-prevalence-ratio-high-low.txt}\\
  \bottomrule
\end{tabular}

\end{table}
% ==================================================================================================
\subsection{Experiment~3: Influence of turnover on the tPAF of the high risk group}
\label{ss:res-tpaf}
Finally, we compared the predicted tPAF of the high risk group
with and without turnover, after fitting to the same prevalence data
(Figure~\ref{fig:tpaf-fit}).
The tPAF approaches 1 for both models over the 50 year period,
indicating that unmet treatment needs of the high risk group
are central to epidemic persistence in both scenarios.
% SB: Crucial.
The estimated tPAF of the high risk group is also higher
in the model with turnover versus
in the model without turnover
over all time horizons,
implying that models which fail to capture turnover dynamics which are present in reality
may underestimate the tPAF of high risk groups.
This increase in tPAF of the high risk group can be attributed to
a higher ratio of fitted partners $C_H~/~C_L$
in the model with turnover is higher than in the model without
(Experiment~2).
The increased partner ratio affords
a higher risk of onward transmission to the high risk group
in the model with turnover, and thus an increase in tPAF.
\begin{figure}[!tbp]
  \centering
  \includegraphics[width=0.5\linewidth]{sit-tpaf-tpaf-high-all-vs=fit.pdf}
  \caption{Transmission population attributable fraction (tPAF)
    of the high risk group in models with and without turnover,
    after fitting the number of partners per year to group-specific prevalence.}
  \label{fig:tpaf-fit}
\end{figure}
% LW: I think this is the key message and valid.
% SS: This section perhaps could be a bit clearer as clearly very important
%     Is there a way to more explicitly talk about new infections
%     in the low / medium risks coming from prevalent infections
%     entering into these states from high risk groups and that
%     subsequent infections emanating from these individuals are thus still
%     linked back to prior history in the high risk group?
% JK: Great point and I realized this was not emphasized anywhere in the paper.
%     However, I think this section is not the right place for it,
%     as the increase in TPAF is really attributable to the higher estimated
%     partners ratio from Experiment 2.
%     However, I highlight that infections in low risk groups can originate
%     from higher risk period in the discussion,
%     since it kind of gets lost in Experiment 1.
