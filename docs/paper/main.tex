\documentclass{article}
\usepackage{amsmath,bm,upgreek}
\usepackage{graphicx}
\usepackage{enumitem}
\usepackage[margin=2cm]{geometry}
\usepackage[colorlinks,linkcolor=blue]{hyperref}
% --------------------------------------------------------------------------------------------------
\graphicspath{{figs/}{figs/tikz/}}
\usepackage{tikz}
\usepackage{tikz-qtree}
\usetikzlibrary{positioning,arrows.meta,quotes,calc}
\tikzset{%
  arrow/.style    = { ->, >=Latex,  line width = 0.4mm, rounded corners, draw = black, every edge/.style={arrow} },
  arrow/.style    = { ->, >=Latex,  very thick, rounded corners,
                      fill = #1, draw = #1 },
  tbox/.style     = { fill = #1!20!white, draw = #1!80!white, thick, align = center,
                      minimum width = 0.8cm, minimum height = 0.6cm, node distance = 1.5cm },
  leaf/.style     = { fill = #1!20!white, draw = #1!80!white, thick, align = center, rounded corners = 0.3cm, grow = down,
                      minimum width = 1.2cm, minimum height = 0.6cm, node distance = 1.5cm }
}
\newcommand{\connectall}[2]{
  \foreach \s in {#1} {
    \foreach \t in {#2} {
      \draw[arrow,<->] (\s) -- (\t);
    }
  }
}
\setlength{\parindent}{0pt}
\setlength{\parskip}{6pt}
\numberwithin{equation}{section}
\setlist[enumerate,1]{label=\arabic*.}
\setlist[enumerate,2]{label*=\arabic*.}
\renewcommand{\zeta}{\upzeta}
\newcommand{\x}{\hat{x}}
\newcommand{\z}{\hat{z}}
\newcommand{\N}{\textsc{n}}
\newcommand{\G}{\textsc{g}}
% --------------------------------------------------------------------------------------------------
\title{Disease Convection:\\
  Modelling Population Turnover in Simulated Epidemics}
\author{Jesse Knight}
%%%%%%%%%%%%%%%%%%%%%%%%%%%%%%%%%%%%%%%%%%%%%%%%%%%%%%%%%%%%%%%%%%%%%%%%%%%%%%%%%%%%%%%%%%%%%%%%%%%%
\begin{document}
%%%%%%%%%%%%%%%%%%%%%%%%%%%%%%%%%%%%%%%%%%%%%%%%%%%%%%%%%%%%%%%%%%%%%%%%%%%%%%%%%%%%%%%%%%%%%%%%%%%%
\maketitle
\tableofcontents
\clearpage
%%%%%%%%%%%%%%%%%%%%%%%%%%%%%%%%%%%%%%%%%%%%%%%%%%%%%%%%%%%%%%%%%%%%%%%%%%%%%%%%%%%%%%%%%%%%%%%%%%%%
\section{Background}

% ==================================================================================================
\subsection{Notation}
We denote the variable representing
the size of group $i \in [1 \dots \G]$ as $x_i$
and the vector of all $x_i$ as $\bm{x}$.
The total population size is denoted $\N$,
and the proportions represented by each group by $\x_i = x_i \N^{-1}$.
The rate of population entry into group $i$ is denoted by $\nu_i$, and
the rate of exit by $\mu_i$
(which may include group-specific disease-attributable death).
The proportion of the entering population who are in group $i$,
which may not be equal to the proportion of the current population in group $i$,
is denoted $\z_i$.
Since the rate of entry $\nu_i$ is typically expressed as
a function of the total population size $\N$,
we model each group in the entering population $\bm{z}$
as having size $z_i = \z_i \N$.
%This somewhat awkwardly implies the existence of a population $\bm{z}$
%which is not included in the model, but which can be thought of as
%the stream of individuals who enter the model.
\par
Turnover transitions can occur between any two groups, in either direction;
therefore we denote the turnover rates as a $\G \times \G$ matrix $\zeta$,
where $\zeta_{ij}$ corresponds to the transition $x_i \rightarrow x_j$.
An explicit definition is given in Eq.~(\ref{eq:zeta}),
where the diagonal elements are denoted $*$ since they represent
transitions from a group to itself, which is inconsequential.
\begin{equation}\label{eq:zeta}
\zeta = \left[\begin{array}{cccc}
         *           & x_1  \rightarrow x_2 & \cdots & x_1 \rightarrow x_\G \\[0.5em]
x_2  \rightarrow x_1 &          *           & \cdots & x_2 \rightarrow x_\G \\[0.5em]
      \vdots         &       \vdots         & \ddots &       \vdots         \\[0.5em]
x_\G \rightarrow x_1 & x_\G \rightarrow x_2 & \cdots &          *
\end{array}\right]
\end{equation}
These transition flows and the associated rates are summarized for $\G = 3$ in Figure~\ref{fig:system}.
\begin{figure}[h]
  \centering
\newcommand{\colorx}{blue}
\newcommand{\colori}{\colorx!40!white}
\newcommand{\coloro}{\colorx!60!black}
\newcommand{\colort}{\colorx!50!green}
\newcommand{\colord}{\colorx!50!red}

\begin{tikzpicture}
\node(x1) [tbox=\colorx] at (0.0,0.0)     {$x_1$};
\node(x2) [tbox=\colorx, below = of x1]   {$x_2$};
\node(x3) [tbox=\colorx, below = of x2]   {$x_3$};
\node(i1) [tbox=\colori, left  = 2cm of x1] {$z_1$};
\node(i2) [tbox=\colori, left  = 2cm of x2] {$z_2$};
\node(i3) [tbox=\colori, left  = 2cm of x3] {$z_3$};
\node(o1) [,   right = 2cm of x1] {};
\node(o2) [,   right = 2cm of x2] {};
\node(o3) [,   right = 2cm of x3] {};
%\node(d1) [,   above right = 0.5cm and 2.5cm of x1] {};
%\node(d2) [,   above right = 0.5cm and 2.5cm of x2] {};
%\node(d3) [,   above right = 0.5cm and 2.5cm of x3] {};

\draw[arrow=\colort] (x1) edge["$\zeta_{12}$", swap, bend right=20] (x2); 
\draw[arrow=\colort] (x2) edge["$\zeta_{21}$", swap, bend right=20] (x1);
\draw[arrow=\colort] (x3) edge["$\zeta_{32}$", swap, bend right=20] (x2); 
\draw[arrow=\colort] (x2) edge["$\zeta_{23}$", swap, bend right=20] (x3);
\draw[arrow=\colort] (x1) edge["$\zeta_{13}$", swap, bend right=55, near end] (x3); 
\draw[arrow=\colort] (x3) edge["$\zeta_{31}$", swap, bend right=55, near end] (x1);
\draw[arrow=\colori] (i1) edge["$\nu_1$", near start] (x1);
\draw[arrow=\colori] (i2) edge["$\nu_2$", near start] (x2);
\draw[arrow=\colori] (i3) edge["$\nu_3$", near start] (x3);
\draw[arrow=\coloro] (x1) edge["$\mu_1$", near end]   (o1);
\draw[arrow=\coloro] (x2) edge["$\mu_2$", near end]   (o2);
\draw[arrow=\coloro] (x3) edge["$\mu_3$", near end]   (o3);
%\draw[arrow=\colord] (x1) edge["$\phi_1$", near end]  (d1);
%\draw[arrow=\colord] (x2) edge["$\phi_2$", near end]  (d2);
%\draw[arrow=\colord] (x3) edge["$\phi_3$", near end]  (d3);

\end{tikzpicture}
  \caption{System of compartments and flows among them for $\G = 3$}
  \label{fig:system}
\end{figure}
% ==================================================================================================
\subsection{Data Sources}
No doubt, the system in Figure~\ref{fig:system} is pretty,
but how can this model be parametrized
-- which data can be used to define the values of
$\N$, $\bm{x}$, $\bm{z}$, $\bm{\nu}$, $\bm{\mu}$, and $\zeta$\thinspace?
% --------------------------------------------------------------------------------------------------
\subsubsection{Population Size $N$}
% --------------------------------------------------------------------------------------------------
\subsubsection{Population Proportions $x$}
% --------------------------------------------------------------------------------------------------
\subsubsection{Entering Proportions $z$}
% --------------------------------------------------------------------------------------------------
\subsubsection{Entry $\nu$ and Exit $\mu$}
% --------------------------------------------------------------------------------------------------
\subsubsection{Turnover $\zeta$}
% ==================================================================================================
\subsection{Approaches to Turnover}
In this section, we will examine previous approaches to modelling turnover
and highlight their assumptions.
% --------------------------------------------------------------------------------------------------
\subsubsection{Assumptions}
The major assumptions which might be made when modelling turnover are summarized as follows:
\begin{enumerate} % TODO: work in progress
  \item \label{ass:zeta=0} There is no turnover: $\zeta_{ij} = 0$
  \item \label{ass:const} Parameter values are constant:
  \begin{enumerate}
    \item \label{ass:const-N} The population size $\N$ does not change with time
    \item \label{ass:const-x} The proportions of each group $\x_i$ do not change with time
    \item \label{ass:const-z} The proportions of the entering population $\z_i$ to not change with time
    \item \label{ass:const-nu-mu} The rates of entry $\nu_i$ and exit $\mu_i$ do not change with time
    \item \label{ass:const-zeta} The rates of turnover $\zeta_{ij}$ do not change with time
  \end{enumerate}
  \item \label{ass:know} Parameter values are known:
  \begin{enumerate}
    \item \label{ass:know-N} The population size $\N$ is known
    \item \label{ass:know-x} The proportions of one or more groups $\x_i$ are known
    \item \label{ass:know-z} The proportions of the entering population $\z_i$ are known
    \item \label{ass:know-nu-mu} The rates of entry $\nu_i$ and exit $\mu_i$ are known
    \item \label{ass:know-zeta-i} The proportion of group $i$ who transfer to group $j$ in a given year is known: $\zeta_{ij}$
    \item \label{ass:know-dur-i} The average duration of an individual in group $i$ is known: $D_i$
  \end{enumerate}
  \item \label{ass:equal} Some parameters are equal:
  \begin{enumerate}
    \item \label{ass:z=x} The group proportions in the entering population are equal to those in the general population: $\z_i = \x_i$
    \item \label{ass:one-nu} The rate of entry into the population is the same for all groups: $\nu_i = \nu, \enspace\forall i$
    \item \label{ass:one-mu} The rate of exit from the population is the same for all groups: $\mu_i = \mu, \enspace\forall i$
  \end{enumerate}
\end{enumerate}
%%%%%%%%%%%%%%%%%%%%%%%%%%%%%%%%%%%%%%%%%%%%%%%%%%%%%%%%%%%%%%%%%%%%%%%%%%%%%%%%%%%%%%%%%%%%%%%%%%%%
\section{Methods}
% ==================================================================================================
\subsection{Equations}
In this work, we propose a framework for implementing turnover,
namely methods for estimating the parameters outlined in Figure~\ref{fig:system},
given a that some are already known from data.
This framework is rooted in a set of equations, which we can summarize as follows:
\begin{enumerate}
  \item \label{sys:mass-balance}\textbf{Mass Balance:}
  the rate of change of group $i$ is equal to the net sum of the flows in and out of the group:
  \begin{equation}\label{eq:mass-balance}
    \frac{d}{dt}x_i
    = \nu_i \thinspace z_i + \sum_{j}{\zeta_{ji} \thinspace x_j}
    - x_i \left( \mu_i + \sum_{j}{\zeta_{ij}} \right)
  \end{equation}
% TODO: this needs review due to nonlinearities of attributable death
%  \item \label{sys:growth}\textbf{Population Growth:}
%  the total growth rate of the population $G$ is simply entry minus exit:
%  \begin{equation}
%    G = \sum_i \nu_i z_i - \sum_i \mu_i x_i
%  \end{equation}
  \item \label{sys:duration}\textbf{Duration:}
  the average duration in group $i$, $D_i$, is the inverse of all efferent flow rates:
  \begin{equation}
    D_i = {\left(\mu_i + \sum_{j}{\zeta_{ij}}\right)}^{-1}
  \end{equation}
\end{enumerate}

% ==================================================================================================
\subsection{Solving the System}
% TODO: from idea/turnover.tex
% ==================================================================================================
\subsection{Experiment}
In this section, we introduce a simple epidemic model.
With this model, we aim to answer the following questions:
\begin{enumerate}
  \item What are the differences in the simulated epidemic using different turnover implementations?
\end{enumerate}
%%%%%%%%%%%%%%%%%%%%%%%%%%%%%%%%%%%%%%%%%%%%%%%%%%%%%%%%%%%%%%%%%%%%%%%%%%%%%%%%%%%%%%%%%%%%%%%%%%%%
\section{Results}
%%%%%%%%%%%%%%%%%%%%%%%%%%%%%%%%%%%%%%%%%%%%%%%%%%%%%%%%%%%%%%%%%%%%%%%%%%%%%%%%%%%%%%%%%%%%%%%%%%%%
\section{Discussion}
%%%%%%%%%%%%%%%%%%%%%%%%%%%%%%%%%%%%%%%%%%%%%%%%%%%%%%%%%%%%%%%%%%%%%%%%%%%%%%%%%%%%%%%%%%%%%%%%%%%%
\section{Conclusions}
%%%%%%%%%%%%%%%%%%%%%%%%%%%%%%%%%%%%%%%%%%%%%%%%%%%%%%%%%%%%%%%%%%%%%%%%%%%%%%%%%%%%%%%%%%%%%%%%%%%%
\end{document}
%%%%%%%%%%%%%%%%%%%%%%%%%%%%%%%%%%%%%%%%%%%%%%%%%%%%%%%%%%%%%%%%%%%%%%%%%%%%%%%%%%%%%%%%%%%%%%%%%%%%
