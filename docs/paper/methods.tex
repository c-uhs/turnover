First we developed a new, unified framework for implementing turnover. 
We then simulated a deterministic, compartmental model of
an illustrative STI to carry out the experiments.

% SM: consistency & double-checks
% modelliing vs. modeling
% avoid using the words 'this', 'these', etc. as much as possible
% ==================================================================================================
\subsection{A unified framework for implementing turnover} %we do not use the term 'turnover system' before, so even though more words - would suggest we try to use same phrasing throughout. we introduce a lot of new terminology in the paper, and so the fewer new 'terms/phrases' that need to be defined --> the eaiser it will be for the reader
%     (PS not addressing other comments yet in this edit)
\label{ss:system}
\begin{figure}
  \centering
  \includegraphics[width=0.5\linewidth]{turnover}
  \caption{System of risk groups and flows between them for $G = 3$}
  \label{fig:system}
\end{figure}
We developed a framework for implementing turnover,	
as depicted in  Figure~\ref{fig:system}.
and detailed in the Appendix~\ref{a:system}. 
In the framework, the simulated population is divided into $G$ risk groups.
The number of individuals in group $i \in [1, \dots, G]$ is denoted $x_i$,
and the relative size of each group is denoted $\hat{x}_i = x_i / N$,
where $N$ is the total population size.
Individuals enter the population at a rate $\nu$ and exit at a rate $\mu$ per year.
The distribution of risk groups at entry into the model
is denoted $\hat{e}_i$, which may be different from $\hat{x}_i$.
The total number of individuals entering group $i$ per year
is therefore given by $\nu \hat{e}_i N$.
Turnover rates are collected in a $G \times G$ matrix $\phi$,
where $\phi_{ij}$ is the proportion of individuals in group $i$
who move from group $i$ into group $j$ each year.

The framework only considers demographic transitions between risk groups, and does not
include any health-states. That is, we assume that rates of turnover $\phi$
do not vary by the health state of individuals. 		%But health-states have not been described yet. Seems funny to write it into the unified system noh? Perhaps should say that the unified system is independent of the disease model, and thus, independent of the health-states, or something like that?
\par
For our specific research questions,			%Clarify that the 'we assume' part is specific to the research question - i.e. for the set of experiments in the current paper. vs. the unified framework assumes that the relative size of risk groups remain stable, etc. 
we assumed that the following: (1) the relative sizes of risk groups		%consider list/number the assumptions to help the reader navigate and follow the thought process	
$\bm{\hat{x}} = [\hat{x}_1, \dots, \hat{x}_G]$
are known and should remain constant over time; and (2) 
the rates of population entry $\nu$ and exit $\mu$
are known, but that they may vary over time.
The approach to estimate $\nu$ and $\mu$ is detailed in Appendix~\ref{aaa:params-nu-mu}.
Thus, what remained was to estimate $\bm{\hat{e}}$ and $\phi$,
representing $G + G(G-1) = G^2$ unknown values.

In the framework,
the parameters (%SM: list parameters) 
are collected in the vector   %change variable to parameter. 'variables' have a very specific meaning in ID modeling (state variable and force of infection terms). What does 'these parameters' refer to? 
$\bm{\theta} = \left[\bm{\hat{e}}, \bm{y}\right]$,
where $\bm{y} = \mathrm{vec}_{i \ne j}(\phi)$.
To uniquely determine the elements of $\bm{\theta}$, we
we constructed a set of linear constraints.
Each constraint $k$ took the form		%SM: remove 'this' and 'these' from every sentence in paper, and then put back in with a clear definition of the subject :) 
$b_k = A_k \bm{\theta}$,
% JK: present tense for notation
where $b_k$ is a constant and $A_k$ is a vector with the same length as $\bm{\theta}$.
The values of $\bm{\theta}$ were then obtained by solving
%\begin{equation}\label{eq:system-matrix}
$\bm{\theta} = A^{-1}\bm{b}$,
%\end{equation}
using existing algorithms~\citep{LAPACK}. %SM: specify what type of algorithm: eg. optimization algorithms for soliving linear systems? 
\par
The framework defines four types of constraints which can used to
solve for the values of $\bm{\hat{e}}$ and $\phi$ via $\bm{\theta}$.
The frameworks supports flexibility with respect to selecting and combining
constraints, guided by data availability and the underlying assumptions.
However, a minimum of $G^2$ non-redundant constraints had to be specified
to produce a unique solution such that one value of $\bm{\theta}$ satisfies all constraints. %SM: using -- word -- is for emphasis in place of commas usually. don't think aposthrophes needed.
Table~\ref{tab:constraints} summarizes 
the four types of constraints, with their underlying assumptions and the type of data which
can be used to inform the parameters.									%SM: the term 'define' takes on various meanings in the text. review for consistency.
Additional details, including
constraint equations, examples, and considerations for combining constraints,
are in Appendix~\ref{aaa:params-turnover}.%
\begin{table}
  \centering
  \caption{Summary of constraint types for defining risk group turnover}
  \label{tab:constraints}
  \footnotesize%
\setlength{\tabcolsep}{3pt}%
\begin{tabular}{llcl}
	\toprule
	Name & Description & Variables & Data sources \& Examples \\
	\midrule
	1.~Constant group size
  & \cellbox{0.3\linewidth}{
    the relative sizes of groups are known or assumed,
    and assumed to not change over time}
  & $\hat{x}_i$
  & \cellbox{0.36\linewidth}{
    demographic health surveys \citep{DHS},
    key population mapping and enumeration \citep{Abdul-Quader2014}}\\
	2.~Specified elements
  & \cellbox{0.3\linewidth}{
    group sizes in the population following sexual debut
    and/or individual rates of turnover
    are known or assumed}
  & $\hat{e}_i$, $\phi_{ij}$
  & \cellbox{0.36\linewidth}{
    demographic health surveys \citep{DHS},
    key population surveys \citep{Baral2014}}\\
	3.~Group duration
  & \cellbox{0.3\linewidth}{
    the average durations of individuals in each group
    are known or assumed}
  & $\delta_i$
  & \cellbox{0.36\linewidth}{
    cohort studies of sexual behaviour over time \citep{Fergus2007},
    key population surveys \citep{Baral2014}}\\
	4.~Turnover rate ratios
  & \cellbox{0.3\linewidth}{
    ratios between different rates of turnover are known or assumed}
  & $\phi_{ij}$
  & \cellbox{0.36\linewidth}{
    demographic health surveys \citep{DHS},
    key population surveys \citep{Baral2014}}\\
	\bottomrule
\end{tabular}\\[1em]
\footnotesize\flushleft
$\nu$:~rate of population entry;
$\phi_{ij}$:~rate of turnover from group $i$ to group $j$;
$\hat{x}_i$:~proportion of individuals in risk group $i$;
$\hat{e}_i$:~proportion of individuals entering into risk group $i$;
$\delta_i$:~average duration spent in risk group $i$.
\end{table}
% ==================================================================================================
\subsection{Transmission model}\label{ss:model-sim}
We developed a deterministic, compartmental model of an illustrative 
sexually tranmitted infection with 3 risk groups. We did not
simulate a specific pathogen, but rather selected a biological system
that included susceptible, infectious, and treated (or recovered/immune) health-states. 
The transmission model thereofore was mechannistically representative of sexually transmitted infections like
HIV (where effective antiretroviral treatment reflects a health-state where individuals
are no longer susceptible nor infectious) or hepatitis B virus (where among the large proportion
of individuals who clear their acute infection, develop life-long protective immunity). %SM: find a citation for the natural history of HBV infection (e.g. a major review paper)

Thus, the model included three health states:
susceptible~$\mathcal{S}$, infectious~$\mathcal{I}$, and treated~$\mathcal{T}$
(Figure~\ref{fig:health-states}),
and $G = 3$ levels of risk:
high~$H$, medium~$M$, and low~$L$.
\begin{figure}
  \centering
  \includegraphics[width=0.4\linewidth]{health-states}
  \caption{Modelled health states.
    $\mathcal{S}$: susceptible;
    $\mathcal{I}$: infected;
    $\mathcal{T}$: treated;
    $\lambda$: force of infection;
    $\tau$: treatment.}
  \label{fig:health-states}
\end{figure}
Risk strata were defined by different number of contacts per year
so that individuals in risk group $i$ were assumed to
form contacts at a rate $C_{i}$ per year.
The probability of contact formation $\rho_{ik}$ between individuals in group $i$
and individuals in risk group $k$ was assumed to be
proportionate to the total number of available contacts within each group:
\begin{equation}
\rho_{ik} = \frac
{C_k x_k}
{\sum_{\mathrm{k}}C_{\mathrm{k}} x_{\mathrm{k}}}
\label{eq:rho}
\end{equation}
\par
The biological probability of transmission was defined as $\beta$ per contact.
Individuals transitioned from the
susceptible $\mathcal{S}$ to infectious $\mathcal{I}$ health-state
via a force of infection $\lambda$ per year, per susceptible in risk group $i$:
\begin{equation}
\lambda_{i} =
C_{i} \sum_k \rho_{ik} \thinspace  \beta \thinspace \frac{\mathcal{I}_k}{x_k}
\label{eq:foi}
\end{equation}
Individuals were assumed to transition from the
infectious $\mathcal{I}$ to treated $\mathcal{T}$ health-state
at a rate $\tau$ per year, reflecting diagnosis and treatment.
The treatment rate did not vary by risk group.
Individuals in the treated $\mathcal{T}$ health-state were neither infectious nor susceptible,
and individuals could not become re-infected.
% --------------------------------------------------------------------------------------------------
\subsubsection{Implementing turnover within the transmission model}
As described in Section~\ref{ss:system}, individuals
entered the model at a rate $\nu$,
exited the model at a rate $\mu$,
and transitioned from risk group $i$ to group $j$ at a rate $\phi_{ij}$.
The turnover rates $\phi$ and
distribution of individuals entering the model by risk group $\bm{\hat{e}}$
were computed using the methods outlined in
Appendix~\ref{aaa:params-turnover}, based on the following assumptions.
% SB: Do we need to explain why we made these assumptions? Or provide refs?
% JK: Since we are constructing a simulated system, I'm not sure what refs might be appropriate,
%     but, you make a great point about justifying.
%     I've added a sentence after all 3 assumptions explaining.
%     I think I was avoiding this because it was hard to explain,
%     but let me know how it reads.
First, we assumed that
the proportion of individuals entering each risk group $\bm{\hat{e}}$
was equal to the proportion of individuals across risk groups in the model $\bm{\hat{x}}$.
% LW: And another strong assumption which was implicit here is that u assume
%     the rate of turn over to be the same irrespective of disease status.
%     And I think it is critical to make it explicit.
%     So S, I, T all have the same turn over rate,
% JK: So, I did now include this in the Turnover System section (last sentence).
%     Do you think it needs to be restated here?
Second, we assumed that
the average duration spent in each risk group $\bm{\delta}$ was known.
Third, we assumed that
the absolute number of individuals moving between two risk groups in either direction was balanced.
% LW: I would think this is a very strong assumption.
%     It is essentially saying the turn-over system consistently
%     *swap* individuals in two risk groups.
% HM: Not fully understand how does this assumption work?
%     If high risk group has 10 individuals move to medium group,
%     medium group will move 10 individuals to high risk? This is independent from population entry?
% JK: @HM: yes, this is what this assumption means. I've edited hopefully to be more clear.
%     @LW: see new sentence below. We have to make at least one more assumption,
%     else the system will not have a unique solution,
%     and by "balancing" the turnover rates, it is a way of avoiding "bias":
These assumptions were chosen to avoid any dominant direction of turnover,
so that the effects of turnover on model outputs would be
attributable to movement of people between risk groups in general,
rather than in one specific direction.
The system of equations which resulted from these assumptions
is given in Appendix~\ref{aa:eqs-turnover}.
To meet all three conditions, there was only one possible value
for each element in $\phi$ and $\bm{\hat{e}}$.
In other words, by specifying these three conditions,
we ensured that a unique set of $\phi$ and $\bm{\hat{e}}$ was computed.
\par
Using the above three assumptions,
we needed to specify the values of $\bm{\hat{x}}$, $\bm{\delta}$, $\nu$, and $\mu$.
Such parameters could be derived from data as described in Appendix~\ref{aaa:params-turnover};
however, in this experiment, we used the illustrative values summarized in
Table~\ref{tab:params}.
% LW: After read the whole results section,
%     I think experiment 2 and 3.2 followed values specified in Table 3.
%     But experiment 1 and 3.1 explored a wide range of
%     turn-over and treatment rates.
%     it is unclear what do u refer to when u said “this experiment”.
%     Given the organization and length – by the time I reached results of Experiemnt 2,
%     I almost forgot under which values they were done – as the section preceed it
%     explored a range of values of turn-over rate and treatment rate.
% JK: Hopefully now with the paper much much shorter, this is resolved?
After resolving the system of equations,
$\bm{\hat{e}}$ was equal to $\bm{\hat{x}}$ (assumed), and $\phi$ was:
\begin{equation}
\label{eq:phi-values}
\phi = \left[\begin{array}{ccc}
* & 0.0833 & 0.0867\\
0.0208 & * & 0.0158\\
0.0058 & 0.0042 & *
\end{array}\right]

\end{equation}
\begin{table}
  \centering
  \caption{Model parameters}
  \label{tab:params}
  \begin{tabular}{clc}
	\toprule
	    Symbol     & Description                                                     &        Default value         \\
	\midrule
	 $\bm{\beta}$  & transmission probability per partnership                        &            $0.03$            \\
	    $\tau$     & rate of treatment initiation among infected                     &            $0.1$             \\
	    $N_0$      & initial population size                                         &            $1000$            \\
	\midrule
	$\bm{\hat{x}}$ & proportion of system individuals by risk group                  & $[ 0.05 \es 0.20 \es 0.75 ]$ \\
	$\bm{\hat{e}}$ & proportion of entering individuals risk by risk group           & $[ 0.05 \es 0.20 \es 0.75 ]$ \\
	$\bm{\delta}$  & average duration spent in each risk group                       &    $[ 5 \es 15 \es 25 ]$     \\
	     $C$       & number of partners per year by individuals in each risk group   &     $[ 25 \es 5 \es 1 ]$     \\
	    $\nu$      & rate of population entry                                        &            $0.05$            \\
	    $\mu$      & rate of population exit                                         &            $0.03$            \\
	\bottomrule
\end{tabular}\\[1em]
\footnotesize\raggedright
All rates have units $\mathrm{year}^{-1}$; durations are in $\mathrm{years}$;
parameters stratified by risk group are written [high, medium, low] risk.
  % SS: All of this seems a little hard to follow in terms of the infection,
  %     b/c it seems to indicate that there is an underlying infection of interest,
  %     but it is not being stated.
  % JK: Same as above, not sure what to say here...
  %     We did start out with HIV, but since we're not considering STI-mortality,
  %     we really cannoy call it HIV.
\end{table}
\par
We then simulated epidemics using these parameters.
The model was initialized with $N_0 = 1000$ individuals
who were distributed across risk groups according to $\bm{\hat{x}}$.
We seeded the epidemic with
one infectious individual in each risk group at $t = 0$.
There were no treated individuals at the start of the epidemic,
and so all individuals except the 3 infectious individuals were susceptible.
We numerically solved the system of ordinary differential equations
in Python%
\footnote{Code for all aspects of the project is available at:
  \href{https://github.com/c-uhs/turnover}{\texttt{https://github.com/c-uhs/turnover}}}
using Euler's method with a time step of $dt = 0.1$ years.
The full system of model equations is given in Appendix~\ref{aa:eqs-model}.
All comparative analyses were then conducted at equilibrium,
defined as a steady state at 500 years with $<1\%$ difference in incidence per year.
% ==================================================================================================
\subsection{Experiments}
\label{ss:exp}
The experiments used to examine
the influence of risk group turnover on model outputs
were as follows.
% --------------------------------------------------------------------------------------------------
\subsubsection{Experiment~1: Influence of turnover on equilibrium incidence and prevalence}
\label{sss:exp-prev-inc}
Experiment~1 examined the influence of turnover on
equilibrium incidence and prevalence, across each risk group and overall.
Incidence was defined as $\lambda_i$ from Eq.~(\ref{eq:foi}), and
prevalence was defined as $\hat{\mathcal{I}}_i = \dfrac{\mathcal{I}_i}{\mathcal{X}_i}$.
% SM: start by something like this to orient the reader
%     and help with interpretation before results section
% JK: I agree this could be useful, but then we go back-and forth between
%     implementation ("controlled by a single parameter") and
%     an overview of the experiment ("influence of turnover on ...")
%     I'd rather introduce how turnover is controlled via duration in high risk group
%     after noting why we cannot simply scale the rates proportionally with a single parameter.
As in similar experiments \citep{Zhang2012,Henry2015},
the rates of turnover were scaled by a single parameter.
However, because the model had $G = 3$ risk groups,
multiplying a set of base rates $\phi$ by a scalar factor
would have resulted in changes to the relative population size of risk groups $\bm{\hat{x}}$.
Thus, we controlled the rates of turnover using
the duration of individuals in the high risk group $\delta_H$,
such that a shorter $\delta_H$ implied higher rates of turnover among all groups.
The duration of individuals in the medium risk group $\delta_M$
was then defined as a value between $\delta_H$ and the maximum duration $\mu^{-1}$
which scaled with $\delta_H$ following:
$\delta_M = \delta_H + \kappa \left(\mu^{-1} - \delta_H\right)$, with $\kappa = 0.3$.
The duration of individuals in the low risk group $\delta_L$
similarly scaled with $\delta_H$,
but due to existing constraints,
specification of $\delta_H$ and $\delta_M$
resulted in only one possible value of $\delta_L$.
In this way, each value of $\delta_H$ was used to define a unique set of turnover rates $\phi$
whose elements all scaled inversely with the duration in the high risk group $\delta_H$.
The value of $\delta_H$ was then varied from 33~to~3 years
to examine the influence of different turnover rates.
The resulting durations in each group are shown in Figure~\ref{fig:dur-group}.
\begin{figure}
  \centering\includegraphics[width=0.45\linewidth]{{1d-dur-all-tau=0.1}.pdf}
  \caption{Average duration in each risk group as turnover rates vary.}
  \label{fig:dur-group}
\end{figure}
\par
We plotted the STI prevalence versus turnover for each risk group,
and explored mechanisms which explained the observed trends,
including movement of individuals between risk groups, and incidence.
We also plotted the STI prevalence ratios between each combination of risk groups,
to help form a basis for understanding the results of subsequent experiments.
% --------------------------------------------------------------------------------------------------
\subsubsection{Experiment~2: Inferred risk heterogeneity with vs without turnover}
\label{sss:exp-infer}
Next, we examined the influence of turnover on
the parameter values inferred via model fitting.
Specifically, we fit the model, with and without turnover, to:
20\% infection prevalence among the high risk group,
8.75\% among the medium risk group,
3\% among the low risk group,
and 5\% overall.
We fit the contact rates $C$ of all risk groups
by minimizing the negative log-likelihood of each predicted prevalence versus the target.%
\footnote{Sample sizes of 500, 2000, 7500, and 10,000 were assumed to generate binomial distributions
  for the high, medium, low, and overall prevalence targets respectively,
  and the minimization was performed using
  the SLSQP method~\citep{Kraft1988} from the SciPy Python package
  (\href{https://docs.scipy.org/doc/scipy/reference/generated/scipy.optimize.minimize.html}
  {\texttt{scipy.optimize.minimize}}).}
We then compared the inferred contact rates $C$
in the model with versus without turnover.
The ratio of fitted (or posterior) contact rates $C_H~/~C_L$
represents a measure of risk heterogeneity in the population,
after fixing all other parameters,
which produces the given infection prevalence.
% --------------------------------------------------------------------------------------------------
\subsubsection{Experiment~3: Influence of turnover on the TPAF of the highest risk group}
\label{sss:exp-tpaf}
Finally, Experiment~3 examined how
the estimated contribution of highest risk group to overall transmission,
as measured by the transmission population attributable fraction (TPAF),
varied with versus without turnover.
% JK: changed this from "is defined as" to "was defined as"
%     I personally prefer the old "is" but let me know what you think @SM.
The TPAF of a risk group $i$ was defined as:
\begin{equation}
\textsc{tpaf}_i(t) = \frac{I_0(t) - I_i(t)}{I_0(t)}
\end{equation}
% JK: again, I think present tense belongs here for notation.
where $I_0(t)$ is the cumulative number of new infections
by time $t$ under usual conditions,
and $I_i(t)$ is the cumulative number of new infections
assuming no transmission from risk group $i$.
Both $I_0(t)$ and $I_i(t)$ were calculated
starting from a system at equilibrium.
\par
We compared the two fitted models from Experiment~2,
which were identical in structure except that
one model had no turnover and one model had turnover
(default parameterization described in Section~\ref{ss:model-sim}).
While prevalence was the same in both models,
the group-specific contact rates inferred via model fitting were different.
Following equilibration of both models,
the TPAF of the high risk group was then estimated over a continuous time horizon.
